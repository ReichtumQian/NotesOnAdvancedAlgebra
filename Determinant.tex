
\chapter{行列式}

\section{行列式基本概念与性质}

\begin{theorem}[行列式初等变换]
  矩阵初等变换相对行列式的变换为:
  \begin{itemize}
  \item 互换两行/列:取反号
  \item 同行/列可同时提出公因式
  \item 倍加:不改变行列式大小
  \end{itemize}
\end{theorem}

\begin{theorem}[行列式分拆]
  行列式可以关于一行或者一列进行分拆,即
  \begin{equation*}
    \left|\begin{array}{cccc}
            a_{11} & a_{12} & \cdots & a_{1 n} \\
            \vdots & \vdots & \ddots & \vdots \\
            b_{s 1}+c_{s 1} & b_{s 2}+c_{s 2} & \cdots & b_{s n}+c_{s n} \\
            \vdots & \vdots & \ddots & \vdots \\
            a_{n 1} & a_{n 2} & \cdots & a_{n n}
          \end{array}\right|=\left|\begin{array}{cccc}
                                     a_{11} & a_{12} & \cdots & a_{1 n} \\
                                     \vdots & \vdots & \ddots & \vdots \\
                                     b_{s 1} & b_{s 2} & \cdots & b_{s n} \\
                                     \vdots & \vdots & \ddots & \vdots \\
                                     a_{n 1} & a_{n 2} & \cdots & a_{n n}
                                   \end{array}\right|+\left|\begin{array}{cccc}
                                                              a_{11} & a_{12} & \cdots & a_{1 n} \\
                                                              \vdots & \vdots & \ddots & \vdots \\
                                                              c_{s 1} & c_{s 2} & \cdots & c_{s n} \\
                                                              \vdots & \vdots & \ddots & \vdots \\
                                                              a_{n 1} & a_{n 2} & \cdots & a_{n n}
                                                            \end{array}\right| .
  \end{equation*}
\end{theorem}



\section{行列式展开}

\begin{definition}[余子式与代数余子式]
  将矩阵$A$去除第$i$行,第$j$列后矩阵的行列式称为$a_{ij}$的余子式$M_{ij}$,
  代数余子式$A_{ij} := (-1)^{i+j}M_{ij}$
\end{definition}


\begin{theorem}[行列式展开]
  固定列$j$,则按列展开为$|A| = \sum\limits_{i = 1}^n a_{ij}A_{ij}$。
  固定行$i$,则按行展开为$|A| = \sum\limits_{j = 1}^n a_{ij}A_{ij}$
\end{theorem}


\begin{theorem}[不同行相乘为0]
  $A$的第$i$行元素与$k$行的代数余子式乘积和为$0$:
  $\sum\limits_{j = 1}^n a_{ij}A_{kj} = 0$
\end{theorem}



\section{特殊的行列式}

\subsection{爪型行列式}

\begin{theorem}[正爪型行列式]
  正爪型行列式通过对角消去一条直边,再按三角行列式展开
\end{theorem}

\begin{theorem}[异爪型行列式]
  异爪型行列式(一横两斜)分以下几种情况:
  \begin{itemize}
  \item 行和为$0$型:全部加到第一行,按第一列展开
  \item 两对角各自相等型:按斜直夹边展开,获得递推公式
  \end{itemize}
\end{theorem}

\begin{exercise}[异爪型行列式]
  (1)计算$|A| = \left|
    \begin{array}{cccccc}
      0&-1&-1&\cdots&-1&-1 \\
       1&1&0&\cdots&0&0 \\
       0&1&1&\cdots&0&0 \\
       \vdots&\vdots&\vdots&&\vdots&\vdots \\
       0&0&0&\cdots&1&1
    \end{array}
  \right|$

  (2)设$s(x) =
  \begin{cases}
    \frac{x}{|x|}, & x \neq 0\\
    0, & x = 0
  \end{cases}
  $,已知$A = (a_{ij})$,
  $a_{ij }= s(i-j)$,求$|A|$
\end{exercise}

\begin{solution}
  (1)按最后一列展开,得到$D_n = (-1)^{n+2} + D_{n-1}$,因此$n$为奇数时$D_n = 0$,$n$为偶数时$D_n = 1$

  (2)上下相减后变为(1)
\end{solution}

\subsection{大对角型(三斜线)}


\subsection{行和相同型}

\begin{theorem}[行和相同型]
  行和相同型行列式分以下两种情况:
  \begin{itemize}
  \item 内部大量相同:加到第一列,提出公因式,消去内部大量元素
  \item 循环型:上下相减(例题见循环型一节)
  \end{itemize}
\end{theorem}

~

\begin{exercise}[内部大量相同型]
  计算行列式$\left|
    \begin{array}{ccccc}
      a&b&b&\cdots&b \\
       b&a&b&\cdots&b \\
       b&b&a&\cdots&b \\
       \vdots&\vdots&&&\vdots \\
       b&b&b&\cdots&a
    \end{array}
  \right|$
\end{exercise}

\begin{solution}
  加到第一列,提出$[a + (n-1)b]$,用第一列消去其他元素,最终得到答案$(a-b)^{n-1}[a + (n-1)b]$
\end{solution}




\subsection{循环型}

循环型一般无论形态都是上下相减,一般只有行和相同型和反对称型两种。

\begin{exercise}[循环型]
  计算$|A|,|B|$
  \begin{equation*}
    |A| = \left|
      \begin{array}{cccccc}
        0&1&2&\cdots&n-2&n-1 \\
        1&0&1&\cdots&n-3&n-2 \\
        2&1&0&\cdots&n-4&n-3 \\
        \vdots&\vdots&\vdots&&\vdots&\vdots \\
        n-2&n-3&n-4&\cdots&0&1 \\
        n-1&n-2&n-3&\cdots&1&0
      \end{array}
    \right| \quad
    |B| = \left|
      \begin{array}{cccccc}
        1&2&3&\cdots&n-1&n \\
         n&1&2&\cdots&n-2&n-1 \\
         n-1&n&1&\cdots&n-3&n-2 \\
         \vdots&\vdots&\vdots&&\vdots&\vdots \\
         3&4&5&\cdots&1&2 \\
         2&3&4&\cdots&n&1
      \end{array}
    \right|
  \end{equation*}
\end{exercise}

\begin{solution}
  (1)依次上面一行减下面一行,变成如下行列式,再每一列加上第一列即可
  \begin{equation*}
    \left|
      \begin{array}{cccccc}
        -1&1&1&\cdots&1&1\\
        -1&-1&1&\cdots&1&1\\
        -1&-1&-1&\cdots&1&1\\ 
          \vdots&\vdots&\vdots&&\vdots&\vdots \\
        -1&-1&-1&\cdots&-1&1\\ 
        n-1&n-2&n-3&\cdots&1&0
      \end{array}
    \right|
  \end{equation*}

  (2)依次上面一行减下面一行,然后将最后一列的$-1$倍加到每一列。
\end{solution}


\subsection{行列式加边法}

\begin{theorem}[行列式加边法]
  计算矩阵$A$的行列式$|A|$时,可加上一边:
  \begin{equation*}
    |A| = \left|
      \begin{array}{cccc}
        a_{11}&a_{12}&\cdots&a_{1n}\\
        a_{21}&a_{22}&\cdots&a_{2n} \\
              \vdots&\vdots&&\vdots \\
              a_{n1}&a_{n2}&\cdots&a_{nn}
      \end{array}
    \right| = \left|
      \begin{array}{ccccc}
        1&1&1&1&1\\
        0&a_{11}&a_{12}&\cdots&a_{1n}\\
        0&a_{21}&a_{22}&\cdots&a_{2n} \\
        \vdots&\vdots&\vdots&&\vdots \\
        0&a_{n1}&a_{n2}&\cdots&a_{nn}
      \end{array}
    \right|
  \end{equation*}
\end{theorem}

\begin{note}
  行列式加边法往往在内部元素大量相同,但行和不同的情况。
\end{note}


\section{Cramer法则}

\begin{theorem}[Cramer法则]
  $n$个方程的$n$元线性方程组$AX = B$,系数矩阵行列式$|A| \neq 0$,
  则其唯一解为
  \begin{equation*}
    \left( \frac{|B_1|}{|A|}, \frac{|B_2|}{|A|},\cdots, \frac{|B_n|}{|A|} \right)
  \end{equation*}
  其中$B_i$为$A$将第$i$列换为$B$组成的矩阵。
\end{theorem}







