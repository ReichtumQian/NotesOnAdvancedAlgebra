

\chapter{对偶空间}

\section{对偶空间与对偶基}

\begin{definition}[对偶空间]
  $V$是$K$上的线性空间,
  将定义于$V$上的所有有界线性泛函组成的空间称为对偶空间$V^{*}$。
\end{definition}

\begin{definition}[对偶基]
  给定$V$的一组基$\epsilon_1,\cdots,\epsilon_n$,
  定义线性泛函$f_i$满足:
  \begin{equation*}
    f_i(\epsilon_j) =
    \begin{cases}
      1, & i = j\\
      0, & i \neq j
    \end{cases}
  \end{equation*}
  则$\{f_1,\cdots,f_n\}$是$V^{*}$的一组基,称其为$\epsilon_1,\cdots,\epsilon_n$的对偶基。
\end{definition}

\begin{note}
  计算对偶基时,
  将$f_i$视为一个向量,
  即$f_i(\alpha) = f_{i1}x_1 + \cdots + f_{in}x_n = f_i^T \alpha$。
  计算$\epsilon_1, \cdots, \epsilon_n$的对偶基只需要满足:
  \begin{equation*}
    \left[
      \begin{array}{c}
        f_1^T\\
        f_2^T\\
        \vdots\\
        f_n^T
      \end{array}
    \right] \left[
      \begin{array}{cccc}
        \epsilon_1&\epsilon_2&\cdots&\epsilon_n
      \end{array}
    \right] = E
  \end{equation*}
\end{note}

~

\begin{exercise}
  给定$\epsilon_1 = [1,1,0]^T, \epsilon_2 = [0,1,1]^T, \epsilon_3 = [0,1,-1]^T$,
  求其对偶基
\end{exercise}

\begin{solution}
  显然$f_1,f_2,f_3$都是三个系数组成的线性函数,
  不妨记$A = [f_1, f_2,f_3]^T, P = [\epsilon_1,\epsilon_2,\epsilon_3]$,则
  \begin{equation*}
    AP = \left[
      \begin{array}{c}
        f_1^T\\
        f_2^T\\
        f_3^T
      \end{array}
    \right] \left[ \epsilon_1, \epsilon_2, \epsilon_3 \right] = E
  \end{equation*}
  因此$A = P^{-1}$,只需求出$P^{-1}$即可。
  最终求出:
  \begin{equation*}
    P^{-1} = \left[
      \begin{array}{ccc}
        1&0&0 \\
         - \frac{1}{2}& \frac{1}{2}& \frac{1}{2} \\
         - \frac{1}{2}& \frac{1}{2}&- \frac{1}{2}
      \end{array}
    \right]
    \Rightarrow
    \begin{cases}
      f_1^T = (1,0,0)\\
      f_2^T = (- \frac{1}{2}, \frac{1}{2}, \frac{1}{2})\\
      f_3^T = (- \frac{1}{2}, \frac{1}{2}, - \frac{1}{2})
    \end{cases}
  \end{equation*}
\end{solution}








