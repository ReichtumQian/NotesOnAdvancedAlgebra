
\section{线性映射}


\subsection{线性映射的概念}

\begin{definition}[线性映射]
  $V,W$是$\mathbb{P}$上的线性空间,$\varphi : V \rightarrow W$若满足:
  \begin{itemize}
  \item $\varphi(\alpha + \beta) = \varphi(\alpha) + \varphi(\beta)$
  \item $\varphi(c\alpha) = c \varphi(\alpha)$
  \end{itemize}
  则称$\varphi$为$V$到$W$的线性映射,
  将$V$到$W$的线性映射全体记为集合$\mathrm{Hom}_{\mathbb{P}}(V,W)$
\end{definition}

\subsection{核与像}

核与像是最常见的不变子空间,因此在进一步研究线性映射之前,要先研究这两个特殊的集合。

\begin{definition}[核与像]
  $U,V$是$\mathbb{P}$上的线性空间,
  $f:U \rightarrow V$,
  定义$\mathrm{Ker}(f) = \{\alpha \in U: f(\alpha ) = 0\}$,
  $\mathrm{Im}(f) = \{f(\alpha):\alpha \in U\}$
\end{definition}

\begin{note}
  像的维度对应着$(\alpha_1,\cdots,\alpha_n)A$的秩,
  核的维度对应着线性方程组$AX = 0$的解空间维数。
\end{note}

\begin{theorem}[核像维数公式]
  $f \in \mathrm{Hom}(U,V)$其核与像满足:
  \begin{equation*}
    \mathrm{dim} ~ \mathrm{Ker}(f) + \mathrm{dim} ~ \mathrm{Im}(f) = \mathrm{dim}(U)
  \end{equation*}
\end{theorem}

\begin{proof}
  设$\alpha_1,\cdots,\alpha_r$是$\mathrm{Ker}f$的基,
  $\alpha_1,\cdots,\alpha_n$是$U$的基,
  而$\mathrm{Im}f = L(f(\alpha_1),\cdots,f(\alpha_n)) = L(f(\alpha_{r+1}),\cdots,f(\alpha_n))$,
  下面证明线性无关。
  假设$k_{r+1}f(\alpha_{r+1}) + \cdots + k_n f(\alpha_n) = 0$,
  根据线性性可知$f(k_{r+1}\alpha_{r+1} + \cdots + k_n\alpha_n) = 0$,
  即$k_{r+1}\alpha_{r+1} + \cdots + k_n \alpha_n \in \mathrm{Ker}(f)$,
  根据基的线性无关性,
  显然$k_{r+1},\cdots,k_n$均为$0$
\end{proof}

\begin{note}
  该定理的证明会在题目中出现,主要方式是先设原空间的子空间的一组基,
  扩展成原空间的一组基,再作用$f$后与像空间进行联系。
\end{note}

~

\begin{exercise}[核像维数公式推广证明]
  证明以下核像命题
  
  (1)定义域为子空间:$\mathcal{A}$是有限维线性空间$V$的线性变换,
  $W$是$V$的子空间,证明:
  \begin{equation*}
    \mathrm{dim} (\mathcal{A} W) + \mathrm{dim}( \mathrm{Ker}\mathcal{A} \cap W) = \mathrm{dim}(W)
  \end{equation*}

  (2)复合映射:$\sigma, \tau$是有限维空间$U \rightarrow V, V \rightarrow W$的线性映射,证明:
  \begin{equation*}
    \mathrm{dim}( \mathrm{Ker} (\sigma)) + \mathrm{dim}( \mathrm{Im}(\sigma) \cap \mathrm{Ker} (\tau)) = \mathrm{dim} ( \mathrm{Ker}(\tau \sigma))
  \end{equation*}
\end{exercise}

\begin{proof}
  (1)设$\alpha_1,\cdots,\alpha_r$是$\mathrm{Ker}\mathcal{A} \cap W$的一组基,
  $\alpha_1,\cdots,\alpha_n$是$W$的一组基,
  则$\mathcal{A}W = L(\mathcal{A} \alpha_{r+1}, \cdots, \mathcal{A} \alpha_n)$。
  考虑$\mathcal{A} W$的维数,设$k_{r+1} \mathcal{A} \alpha_{r+1} + \cdots + k_n \mathcal{A} \alpha_n = 0$,
  则有$k_{r+1}\alpha_{r+1} + \cdots + k_n \alpha_n \in \mathrm{Ker}\mathcal{A} \cap W$,
  因此得到$k_{r+1} = \cdots = k_n = 0$。

  (2)先得找原空间最小的空间,可知$\mathrm{Ker}(\sigma) \subseteq \mathrm{Ker}(\tau \sigma)$,
  因此设$\alpha_1,\cdots,\alpha_r$是$\mathrm{Ker}(\sigma)$的基,
  $\alpha_1,\cdots, \alpha_n$是$\mathrm{Ker}(\tau \sigma)$的基。
  设$V_1  = L(\sigma \alpha_{r+1},\cdots,\sigma \alpha_n)$。
  显然$V_1 \subseteq \mathrm{Im}(\sigma) \cap \mathrm{Ker}(\tau)$。

  另一侧$\forall \alpha \in \mathrm{Im}(\sigma ) \cap \mathrm{Ker}(\tau)$,
  $\exists \beta$使得$\sigma (\beta) = \alpha, \tau (\alpha )= 0$,
  这说明$\tau(\sigma(\beta)) = 0$,因此$\beta \in \mathrm{Ker}(\tau \sigma)$,即$\alpha \in V_1$,
  最终得出$V_1 = \mathrm{Im}(\sigma) \cap \mathrm{Ker}(\tau)$。
\end{proof}

~

\begin{theorem}[核像与单射、满射]
  $\sigma$是$U \rightarrow W$的线性映射,则:
  \begin{itemize}
  \item $\sigma$是满射当且仅当$\mathrm{Im}(\sigma) = W$
  \item $\sigma$是单射当且仅当$\mathrm{Ker}(\sigma) = \{\mathbf{0}\}$
  \item 推论:有限维线性空间上线性\textbf{变换}(非映射)是单射当且仅当其是满射(一般单射更好证)
  \end{itemize}
\end{theorem}

\begin{theorem}[核、像任意性]
  $V$是$P$上$n$维线性空间,$V_1,V_2$是其子空间,$\mathrm{dim}V_1 + \mathrm{dim}V_2 = n$,
  则存在$V$上的线性变换$\mathcal{A}$使得$\mathrm{Ker}\mathcal{A} = V_1, \mathrm{Im}\mathcal{A} = V_2$
\end{theorem}

\begin{proof}
  设$V_1$的一组基为$\alpha_1 ,\cdots, \alpha_r$,
  扩充为$V$的一组基$\alpha_1,\cdots,\alpha_n$,
  设$\beta_{r+1},\cdots,\beta_n$是$V_2$的一组基,
  构造映射$\mathcal{A}(k_1 \alpha_1 + \cdots + k_n \alpha_n) = k_{r+1} \beta_{r+1} + \cdots + k_n\beta_n$。
  可证明$\mathcal{A}$是线性变换,且核、像满足条件。
\end{proof}

~

\begin{exercise}[计算核与像]
  (1)$\mathcal{A}: \mathbb{P}^4 \rightarrow \mathbb{P}^3$,求$\mathrm{Ker}\mathcal{A}, \mathrm{Im}\mathcal{A}$
  \begin{equation*}
    \mathcal{A} \left(
      \begin{array}{c}
        x_1\\
        x_2\\
        x_3\\
        x_4
      \end{array}
    \right) = \left(
      \begin{array}{c}
        x_1 - 3x_2 + x_3 - 2x_4\\
        -x_1 - 11x_2 + 2x_3 - 5x_4\\
        3x_1 + 5x_2 + x_4
      \end{array}
    \right)
  \end{equation*}
\end{exercise}

\begin{solution}
  (1)$\mathrm{Ker}\mathcal{A}$相当于$AX = 0$的解空间,
  $\mathrm{Im}\mathcal{A}$相当于$A$的列空间
\end{solution}


\subsection{幂等变换的核与像}

一般而言核与像虽然有维数关系,但不具备直和关系(可能相交也可能和不为全空间),
但是幂等变换是一个特例。

\begin{definition}[幂等变换]
  若线性变换$\mathcal{A}$满足$\mathcal{A}^2 = \mathcal{A}$,则称其为幂等变换。
\end{definition}

\begin{theorem}[幂等变换核像空间分解]
  $\mathcal{A}$是$V$上的幂等变换,则
  \begin{equation*}
    V = \mathrm{Ker}\mathcal{A} \oplus \mathrm{Im}\mathcal{A}
  \end{equation*}
\end{theorem}

\begin{proof}
  (1)先证明$V$是核与像的和:$\forall \alpha \in V$,
  有$\alpha = \alpha - \mathcal{A} \alpha + \mathcal{A} \alpha$,
  其中$\mathcal{A} \alpha \in \mathrm{Im}\mathcal{A}$。
  而$\mathcal{A}(\alpha - \mathcal{A} \alpha) = \mathcal{A} \alpha - \mathcal{A}^2 \alpha = 0$,
  因此$\alpha - \mathcal{A}\alpha \in \mathrm{Ker}\mathcal{A}$

  (2)证明直和:$\forall \alpha \in \mathrm{Ker}\mathcal{A} \cap \text{Im}\mathcal{A}$,
  则$\mathcal{A} \alpha = 0$且$\exists \beta, \mathcal{A} \beta = \alpha$,
  两侧同时作用$\mathcal{A}$,左侧得到$\mathcal{A}^2 \beta = \mathcal{A} \beta = \alpha$,
  右侧$\mathcal{A} \alpha = 0$,因此$\alpha = 0$
\end{proof}

% \begin{theorem}[幂等变换核像推广]
%   $\mathcal{A}_1,\cdots,\mathcal{A}_s$是$V$上$s$个幂等变换,且对$\forall i\neq j$有$\mathcal{A}_i \mathcal{A}_j = \mathcal{O}$,
%   则:
%   \begin{equation*}
%     V = \mathrm{Im}\mathcal{A}_1 \oplus \cdots \oplus \mathrm{Im}\mathcal{A}_s \oplus \mathop{\cap}\limits_{i = 1}^s \mathrm{Ker}\mathcal{A}_i
%   \end{equation*}
% \end{theorem}


~

\begin{exercise}[经典幂等变换核像题]
  $\mathcal{A}, \mathcal{B}$为$\mathbb{P}$上线性空间中的幂等变换,证明:

  (1)$\mathcal{A},\mathcal{B}$有相同的像当且仅当$\mathcal{A}\mathcal{B} = \mathcal{B}, \mathcal{B}\mathcal{A} = \mathcal{A}$

  (2)$\mathcal{A},\mathcal{B}$有相同的核当且仅当$\mathcal{A}\mathcal{B} = \mathcal{A}, \mathcal{B}\mathcal{A} = \mathcal{B}$
\end{exercise}

\begin{proof}
  (1)左推右:$\forall \alpha \in V$有$\mathcal{B} \alpha = \in \mathrm{Im}(\mathcal{B})$,
  根据条件即$\mathcal{B}\alpha \in \mathrm{Im}\mathcal{A}$,
  因此根据幂等变换投影的性质($\mathcal{A}$不改变$\mathrm{Im}\mathcal{A}$中的向量)得到
  \begin{equation*}
    \mathcal{A}(\mathcal{B} \alpha) = \mathcal{B} \alpha \Rightarrow \mathcal{A}\mathcal{B} = \mathcal{B}
  \end{equation*}
  同理得到$\mathcal{B}\mathcal{A} = \mathcal{A}$

  右推左:$\forall \gamma \in \mathrm{Im}(\mathcal{A})$,有$\mathcal{A} \gamma = \gamma$,
  而$\mathcal{B}\mathcal{A} \gamma = \mathcal{A} \gamma = \gamma$,因此$\mathcal{B} \gamma = \gamma$,
  即$\gamma \in \mathrm{Im}(\mathcal{B})$,另一侧同理

  (2)左推右:由于$V = \mathrm{Im}(\mathcal{A}) \oplus \mathrm{Ker}(\mathcal{A})$,
  因此$\forall \alpha \in V$有
  \begin{equation*}
    \alpha = \alpha_1 + \alpha_2, \alpha_1 \in \mathrm{Im}(\mathcal{A}), \alpha_2 \in \mathrm{Ker}\mathcal{A}
  \end{equation*}
  $\alpha - \alpha_1 \in \mathrm{Ker}\mathcal{A}$,因此$\alpha - \alpha_1 \in \mathrm{Ker}\mathcal{B}$,
  故$\mathcal{B}(\alpha - \alpha_1) = 0$,
  即$\mathcal{B} \alpha = \mathcal{B} \alpha_1$。
  而$\mathcal{A} \alpha = \alpha_1$,得到
  \begin{equation*}
    (\mathcal{B}\mathcal{A})\alpha = \mathcal{B}\alpha_1 = \mathcal{B} \alpha
  \end{equation*}
  因此得出$\mathcal{B}\mathcal{A} = \mathcal{B}$

  右推左
\end{proof}

\subsection{一般线性变换的核像分解}

\begin{theorem}[线性变换核像空间分解]
  假设$\mathcal{A}$为线性空间$V$中的线性变换,
  $\mathrm{dim}V = n$,
  则$V$可进行如下分解:
  \begin{equation*}
    V = \mathrm{Ker}(\mathcal{A}^n) \oplus \mathrm{Im}(\mathcal{A}^n)
  \end{equation*}
\end{theorem}

\begin{note}
  该结论在后续证明空间分解法中会用到
\end{note}


\section{线性变换}

\subsection{线性变换基本概念}

\begin{definition}[线性变换]
  $V$是数域$\mathbb{P}$上的线性空间,$\mathcal{A}$是$V$到自身的线性映射,
  则称$\mathcal A$是$V$内的线性变换。$V$内全体线性变换组成集合$\mathrm{End}(V)$。
\end{definition}


~

\begin{exercise}[线性变换基本练习]
  (1)若$T_1,\cdots,T_m$是$m$个不同的线性变换,证明存在$\alpha \in V$,使得$T_1\alpha, \cdots,T_m\alpha$互不相同
\end{exercise}

\begin{proof}
  (1)令$V_{ij} = \{x \in V: T_ix = T_jx\}$,
  则显然$V_{ij}$是$V$的子空间。
  因为$T_1,\cdots,T_m$两两不同,故$\forall T_i,T_j, \exists \beta$使得$T_i\beta \neq T_j\beta$,
  从而$V_{ij} \neq V$。
  设$V_{ij}$中有$s$个是零空间,则其余均为非平凡子空间,
  而根据线性空间习题(和与交一节)可知存在$\alpha$不属于任一非平凡子空间,
  故$\exists \alpha$不属于$V_{ij}, \forall i,j$,
  因此$T_1\alpha,\cdots,T_m\alpha$两两不同。
\end{proof}

~

\begin{theorem}[线性变换的可逆性]
  线性变换$T$可逆当且仅当
  \begin{itemize}
  \item $T\epsilon_1,\cdots, T\epsilon_n$线性无关
  \item $T$是单射
  \item $T$是满射
  \end{itemize}
\end{theorem}

\begin{proof}
  (1)从矩阵角度看即可

  (2)左推右:若$T\alpha_1 = T\alpha_2$,则两侧作用$T^{-1}$得到$\alpha_1 = \alpha_2$

  右推左:下证$T\epsilon_1,\cdots,T\epsilon_n$线性无关,
  若$k_1T\epsilon_1 + \cdots + k_nT\epsilon_n = 0$,
  则
  \begin{equation*}
    T(k_1\epsilon_1 + \cdots + k_n\epsilon_n) = 0
  \end{equation*}
  由于$T$是单射,得到$k_1\epsilon_1 + \cdots + k_n\epsilon_n = 0$,因此$k_1 = \cdots = k_n = 0$,
  根据命题(1)知可逆

  (3)左推右:设$T$可逆,则$\forall \alpha \in V$,存在$T^{-1}\alpha$,而$\alpha = T(T^{-1}\alpha)$,
  因此是满射

  右推左:由于满射,对任一一组基$\eta_1,\cdots,\eta_n$,总存在$\epsilon_1,\cdots,\epsilon_n$,使得
  \begin{equation*}
    T\epsilon_1 = \eta_1,\cdots,T\epsilon_n = \eta_n
  \end{equation*}
  下面证明$\epsilon_1,\cdots,\epsilon_n$线性无关,
  设$k_1\epsilon_1 + \cdots + k_n\epsilon_{n }= 0$,
  则
  \begin{equation*}
    T(k_1\epsilon_1 + \cdots + k_n\epsilon_n) = k_1\eta_1 + \cdots + k_n\eta_n = 0
  \end{equation*}
  这说明$k_1 = \cdots = k_n = 0$,因此可逆。
\end{proof}


\subsection{线性变换在基下的矩阵、变换后的坐标}


\begin{definition}[线性变换的矩阵]
  $\mathcal A$是$V$内的线性变换,$V = L(\epsilon_1,\epsilon_2,\cdots,\epsilon_n)$,
  若$(\mathcal{A}(\epsilon_1),\cdots,\mathcal{A}(\epsilon_n)) = (\epsilon_1,\cdots,\epsilon_n)A$,
  则$A$称为$\mathcal{A}$在$\epsilon_1,\cdots,\epsilon_n$基下的矩阵。
\end{definition}

\begin{note}
  计算线性变换在基下的矩阵:根据$(\mathcal{A}\epsilon_1,\cdots,\mathcal{A} \epsilon_n) =  (\epsilon_1,\cdots,\epsilon_n)A$推出
  \begin{equation*}
    A = (\epsilon_1,\cdots,\epsilon_n)^{-1} (\mathcal{A} \epsilon_1,\cdots,\mathcal{A}\epsilon_n)
  \end{equation*}
  可以用矩阵求逆方式求解。
\end{note}


~

\begin{exercise}[线性变换在基下的矩阵]
  已知$A = (a_{ij}), i,j = 1,2,3$是$\mathcal{A}$在$\epsilon_1,\epsilon_2,\epsilon_3$下的矩阵,计算

  (1)$\mathcal{A}$在$\epsilon_3,\epsilon_2,\epsilon_1$下的矩阵
  
  (2)$\mathcal{A}$在$\epsilon_1,k\epsilon_2,\epsilon_3$下的矩阵

  (3)$\mathcal{A}$在$\epsilon_1 + \epsilon_2,\epsilon_2,\epsilon_3$下的矩阵
\end{exercise}

\begin{solution}
  根据$T(\epsilon_1,\epsilon_2,\epsilon_3 )= (\epsilon_1,\epsilon_2,\epsilon_3)A$,
  得到
  \begin{equation*}
    \begin{cases}
      T \epsilon_1 = a_{11}\epsilon_1 + a_{21}\epsilon_2 + a_{31}\epsilon_3\\
      T \epsilon_2 = a_{12}\epsilon_1 + a_{22}\epsilon_2 + a_{32}\epsilon_3\\
      T \epsilon_3 = a_{13}\epsilon_1 + a_{23}\epsilon_2 + a_{33}\epsilon_3
    \end{cases}
  \end{equation*}
  
  (1)重新排列得到:
  \begin{equation*}
    \begin{cases}
      T \epsilon_3 = a_{33}\epsilon_3 + a_{23}\epsilon_2 + a_{13}\epsilon_1\\
      T \epsilon_2 = a_{32}\epsilon_3 + a_{22}\epsilon_2 + a_{12}\epsilon_1\\
      T \epsilon_1 = a_{31}\epsilon_3 + a_{21}\epsilon_2 + a_{11}\epsilon_1
    \end{cases}
  \end{equation*}
  得到矩阵为
  \begin{equation*}
    \left[
      \begin{array}{ccc}
        a_{33}&a_{32}&a_{31}\\
        a_{23}&a_{22}&a_{21}\\
        a_{13}&a_{12}&a_{11}
      \end{array}
    \right]
  \end{equation*}

  (2)(3)同理列开即可
\end{solution}

~

\begin{exercise}[幂零变换的矩阵]
  $T$是$n$维线性空间$V$上的线性变换,$T^{n-1} \neq \mathcal{O}, T^n = \mathcal{O}$,证明:
  $T$在某组基下的矩阵为
  \begin{equation*}
    \left[
      \begin{array}{ccccc}
        0&0&\cdots&0&0 \\
         1&0&\cdots&0&0 \\
         \vdots&\vdots&&\vdots&\vdots \\
         0&0&\cdots&1&0
      \end{array}
    \right]
  \end{equation*}
\end{exercise}

\begin{proof}
  首先$\alpha,T\alpha,\cdots,T^{n-1}\alpha$均线性无关,
  因此它们是$V$的一组基,$T$在这组基下的矩阵即目标。
\end{proof}

~

\begin{theorem}[线性变换后的坐标]
  若$\alpha$在$(\epsilon_1,\cdots,\epsilon_n)$基下的坐标为$X$,
  则$\mathcal{A}\alpha$在这组基下的坐标为$AX$。
\end{theorem}

\begin{proof}
  $\mathcal{A} \alpha = \mathcal{A}(x_1\epsilon_1 + \cdots + x_n\epsilon_n)
  = \mathcal{A}(\epsilon_1,\cdots,\epsilon_n)X 
  = (\epsilon_1,\cdots,\epsilon_n)AX$,
\end{proof}



\subsection{线性变换在不同基下的矩阵:相似关系}

\begin{theorem}[线性变换在不同基下的矩阵]
  $\epsilon_i,\eta_i(i = 1,\cdots,n)$是$V$两组基,
  且$(\eta_1,\cdots,\eta_n) = (\epsilon_1,\cdots,\epsilon_n)T$。
  若$\mathcal{A}$在$\epsilon_i,\eta_i$基下的矩阵分别为$A,B$,则有$B = T^{-1}AT$。
\end{theorem}

\begin{proof}
  由线性变换矩阵定义,$\mathcal{A}(\epsilon_1,\cdots,\epsilon_n) = (\epsilon_1,\cdots,\epsilon_n)A$,
  $\mathcal{A}(\eta_1,\cdots,\eta_n) = (\eta_1,\cdots,\eta_n)B$。
  而$\mathcal{A}(\eta_1,\cdots,\eta_n) = \mathcal{A}(\epsilon_1,\cdots,\epsilon_n)T = (\epsilon_1,\cdots,\epsilon_n)AT$,
  且$\mathcal{A}(\eta_1,\cdots,\eta_n) = (\epsilon_1,\cdots,\epsilon_n)TB$
  上面两式取等号即可。
\end{proof}

\begin{note}
  记忆时根据$(\eta_1,\cdots,\eta_n) = (\epsilon_1,\cdots,\epsilon_n)T$,
  将$\eta,\epsilon$基直接换成对应的矩阵$B,A$,
  有$T$的一边添上$T^{-1}$即可,即$B = T^{-1}AT$
\end{note}


~

\begin{exercise}[线性变换在不同基下的矩阵]
  给定下述两组基,线性变换$T$满足$T\epsilon_i = \eta_i$,
  \begin{equation*}
    \begin{cases}
      \epsilon_1 = (1,0,1)^T, \epsilon_2 = (2,1,0)^T, \epsilon_3 = (1,1,1)^T\\
      \eta_1 = (1,2,-1)^T, \eta_2 = (2,2,-1)^T, \eta_3 = (2,-1,-1)^T
    \end{cases}
  \end{equation*}

  (1)写出$\epsilon_1,\epsilon_2,\epsilon_3$到$\eta_1,\eta_2,\eta_3$的过渡矩阵

  (2)写出$T$在$\epsilon_1,\epsilon_2,\epsilon_3$下的矩阵

  (3)写出$T$在$\eta_1,\eta_2,\eta_3$下的矩阵
\end{exercise}

\begin{solution}
  (1)$e_1,e_2,e_3$下的矩阵为
  \begin{equation*}
    (\eta_1,\eta_2,\eta_3) = 
  \end{equation*}
\end{solution}

~

\begin{theorem}[线性变换构造的等价类]
  矩阵$A,B$相似的充要条件为它们是一个线性变换在两组基下的矩阵。
\end{theorem}

\subsection{相似矩阵的性质}

\begin{theorem}[相似矩阵的基本性质]
  $K$上矩阵$A,B$相似,且满足$B = P^{-1}AP$,则
  \begin{itemize}
  \item $r(A) = r(B)$
  \item $A^m \sim B^m$
  \item $f(x)$为多项式,则$f(A) \sim f(B)$
  \item 若$A_1 \sim B_1, A_2 \sim B_2$,不一定有$A_1 + A_2 \sim B_1 + B_2$,要求过渡矩阵相同才行
  \end{itemize}
\end{theorem}

\begin{proof}
  (1)$r(A) = r(B)$:
  因为$B = P^{-1}AP$,而$P$满秩,等价于一系列初等变换,
  故不改变秩。
\end{proof}

\section{特征值、相似对角化}


\subsection{特征值与特征向量}

\begin{definition}[特征值与特征向量]
  $\mathcal{A}$是数域$\mathbb{P}$上线性空间$V$内的一个线性变换,
  若对于$\lambda \in \mathbb{P}$,$\exists \xi \neq 0$,
  使得$\mathcal{A} \xi = \lambda \xi$,则称$\lambda$是$\mathcal{A}$的一个特征值,
  $\xi$是$\mathcal{A}$属于$\lambda$的特征向量。
\end{definition}

\begin{definition}[特征子空间]
  对于$\lambda \in \mathbb{P}$,定义其特征子空间为所有满足$\mathcal{A} \xi = \lambda \xi$的特征向量$\xi$所组成的集合$V_{\lambda}$。
\end{definition}

\begin{lemma}[特征子空间的性质]
  特征子空间$V_{\lambda}$有如下性质:
  \begin{itemize}
  \item $V_{\lambda}$是$V$的子空间。
  \item  对于$\lambda \in \mathbb{P}$,$V_{\lambda} \neq \{\mathbf{0}\}$的充要条件为$|\lambda E - A| = 0$,
    其中$A$是$\mathcal{A}$在任意基下的矩阵。
  \end{itemize}
\end{lemma}

\begin{proof}
  (1)只需要证明$V_{\lambda}$对$V$内的数乘和加法封闭即可。

  (2)对于$\lambda \alpha = \mathcal{A} \alpha$,可以取$V$一组基,若$\alpha$在这组基下的坐标为$X$。
  则等价于求解$\lambda X = AX$,其中$A$是$\mathcal{A}$在这组基下的矩阵。
  等价于求解$(\lambda E - A)X = 0$,而线性方程组有非零解的充要条件为系数矩阵行列式非零。
\end{proof}

\begin{definition}[特征多项式]
  对于$\mathbb{P}$上的矩阵$A$,特征多项式定义为$f(\lambda) = |\lambda E - A|$。
\end{definition}

\begin{theorem}[相似矩阵的特征多项式]
  相似矩阵有相同特征多项式。即对于同一个线性变换,特征多项式是不变的。
\end{theorem}

\begin{proof}
  若$B = T^{-1}AT$,则$|\lambda E - B| = |\lambda E - T^{-1}AT|
  = |T^{-1}||\lambda E - A||T| = |\lambda E - A|$。
\end{proof}

\begin{theorem}[特征值基本性质]
  $\lambda_1,\cdots,\lambda_n$是矩阵$A$的特征值。
  则$\sum \lambda_i = \mathrm{tr}(A)$,$\prod \lambda_i =  |A|$
\end{theorem}

\begin{proof}
  用行列式展开与韦达定理证明。
\end{proof}

~

\begin{exercise}[特征值与特征向量的基本练习]
  (1)$\lambda_1,\lambda_2$是$T$两个不同的特征值,$\alpha_1,\alpha_2$分别是对应的特征向量,
  证明:$\alpha_1 + \alpha_2$不是$T$的特征向量

  (2)$T$是可逆线性变换,若$\lambda$是$T$的特征值,则$\lambda^{-1}$是$T^{-1}$的特征值
\end{exercise}

\begin{proof}
  (1)$T\alpha_1 = \lambda_1\alpha_1, T\alpha_2 = \lambda_2\alpha_2$,
  因此
  \begin{equation*}
    T(\alpha_1 + \alpha_2) = \lambda_1 \alpha_1 + \lambda_2 \alpha_2
  \end{equation*}
  若$\alpha_1 + \alpha_2$是特征向量,$\lambda$是特征值,
  则$T(\alpha_1 + \alpha_2) = \lambda(\alpha_1 + \alpha_2)$,
  得到
  \begin{equation*}
    (\lambda - \lambda_1)\alpha_1 + (\lambda - \lambda_2)\alpha_2 = 0
  \end{equation*}
  因此$\lambda - \lambda_1 = \lambda - \lambda_2 = 0$,这与假设矛盾。

  (2)由于$T\alpha = \lambda \alpha$,因此$\alpha = \lambda T^{-1}(\alpha)$,
  即$T^{-1}(\alpha) = \lambda^{-1}\alpha$,
  因此$\lambda^{-1}$是$T^{-1}$的特征值
\end{proof}

~


\subsection{特征值与特征向量进一步结论}

\begin{theorem}[$\mathcal{A}\mathcal{B}$与$\mathcal{B}\mathcal{A}$的特征值]
  $V,V^{\prime}$是$K$上的线性空间,$\mathcal{A} \in \mathrm{Hom}(V,V^{\prime}), \mathcal{B} \in \mathrm{Hom}(V^{\prime},V)$(矩阵同理),则
  \begin{itemize}
  \item $\mathcal{A}\mathcal{B}$与$\mathcal{B}\mathcal{A}$有相同非零特征值
  \item 若$\xi$是$\mathcal{A}\mathcal{B}$属于$\lambda_0$的特征向量,则$\mathcal{B}\xi$是$\mathcal{B}\mathcal{A}$属于$\lambda_0$的一个特征向量
  \end{itemize}
\end{theorem}

\begin{proof}
  设$\lambda_0$是$\mathcal{A}\mathcal{B}$非零特征值,
  $\exists \xi \in V^{\prime}$使得$(\mathcal{A}\mathcal{B})\xi = \lambda_0 \xi$,
  两侧同时作用$\mathcal{B}$得到:
  \begin{equation*}
    \mathcal{B}(\mathcal{A}\mathcal{B})\xi = \mathcal{B}(\lambda_0 \xi) \Rightarrow (\mathcal{B}\mathcal{A})(\mathcal{B}\xi) = \lambda_0(\mathcal{B} \xi)
  \end{equation*}
\end{proof}

\begin{theorem}[Sylvester公式]
  $A_{m \times n} ,B_{n \times m}$,$f_{AB}(\lambda), f_{BA}(\lambda)$是$AB,BA$的特征多项式,则
  \begin{equation*}
    \lambda^nf_{AB}(\lambda) = \lambda^m f_{BA}(\lambda)
  \end{equation*}
\end{theorem}

\begin{proof}
  证明:设$\lambda \neq 0$,根据
  \begin{equation*}
    \left[
      \begin{array}{cc}
        E_n&0\\
        -A&E_m
      \end{array}
    \right] \left[
      \begin{array}{cc}
        E_n&\frac{1}{\lambda}B\\
        A&E_m
      \end{array}
    \right] = \left[
      \begin{array}{cc}
        E_n&\frac{1}{\lambda}B\\
        O&E_m - \frac{1}{\lambda}AB
      \end{array}
    \right] \Rightarrow |H| = \left|
      \begin{array}{cc}
        E_n&\frac{1}{\lambda}B\\
        O&E_m - \frac{1}{\lambda}AB
      \end{array}
    \right| = \left( \frac{1}{\lambda} \right)^m |\lambda E_m - AB|
  \end{equation*}
  同理根据
  \begin{equation*}
    \left[\begin{array}{cc}
            E_{n} & \frac{1}{\lambda} B \\
            A & E_{m}
          \end{array}\right] \cdot\left[\begin{array}{cc}
                                          E_{n} & 0 \\
                                          -A & E_{m}
                                        \end{array}\right]=\left[\begin{array}{cc}
                                                                   E_{n}-\frac{1}{\lambda} A B & \frac{1}{\lambda} B \\
                                                                   0 & E_{m}
                                                                 \end{array}\right] \Rightarrow |H| = \left( \frac{1}{\lambda} \right)^n |\lambda E_n - BA|
  \end{equation*}
  根据$|H|$的值即可推出结论。
\end{proof}

~

\begin{exercise}[相关结论]
  (1)$A,B$是两个$n$阶方阵,证明:$AB + A$和$BA + A$有相同的特征多项式

  (2)计算下面矩阵$C$的特征值
  \begin{equation*}
    C = \left[
      \begin{array}{cccc}
        a_1b_1&a_1b_2&\cdots&a_1b_n \\
              a_2b_1&a_2b_2&\cdots&a_2b_n \\
              \vdots&\vdots&&\vdots \\
              a_nb_1&a_nb_2&\cdots&a_nb_n
      \end{array}
    \right]
  \end{equation*}

  (3)特例:已知$A = (a_1,\cdots,a_n)$,证明$A^TA$的特征值为$0$和$a_1^2 + \cdots + a_n^2$,
  这里$0$是$n-1$重根
\end{exercise}

\begin{proof}
  (1)由于$AB +A = A(B + E)$,$BA + A = (B + E)A$,
  因此根据前面定理可知。

  (2)令$A = (a_1,\cdots,a_n)^T, B = (b_1,\cdots,b_n)$,则$C = AB$,
  而$BA = \sum\limits_{i = 1}^n a_ib_i$为一阶矩阵,其特征值为$\sum\limits_{i = 1}^n a_ib_i$,
  $AB$和$BA$特征多项式相差$\lambda^{n-1}$,
  因此$C = AB$的特征多项式为
  \begin{equation*}
    \lambda^{n-1}(\lambda - \sum\limits_{i = 1}^n a_ib_i)
  \end{equation*}

  (3)即(2)的特例
\end{proof}

~

\begin{theorem}[可逆的特征值表达]
  $n$阶方阵$A$可逆当且仅当$A$的特征值全不为$0$
\end{theorem}

\begin{proof}
  用$|A| = \lambda_1 \cdots \lambda_n$即可
\end{proof}

\begin{theorem}[逆矩阵的特征值]
  $n$阶可逆矩阵$A$的特征值为$\lambda_1,\cdots,\lambda_n$,
  则$A^{-1}$的全部特征值为$\lambda_1^{-1},\cdots,\lambda_n^{-1}$
\end{theorem}

\begin{proof}
  用$A$相似于Jordan阵$J$,
  $J^{-1}$为上三角且对角为$\lambda_1^{-1},\cdots,\lambda_n^{-1}$,
  因此$A^{-1}$的特征值也为$\lambda_1^{-1},\cdots,\lambda_n^{-1}$
\end{proof}

\subsection{相似的进一步研究}

\begin{theorem}[相似的矩阵分解]
  $A,B$相似当且仅当存在方阵$P,Q$,使得
  \begin{equation*}
    A = PQ, B = QP
  \end{equation*}
  这里$P,Q$至少有一个可逆
\end{theorem}

\begin{proof}
  由于$A,B$相似,故存在可逆$P$使得$P^{-1}AP = B$,令$Q = P^{-1}A$,
  则得到$A = PQ, B = QP$
\end{proof}

\begin{theorem}[一些相似结论]
  $A,B$为$n$阶方阵,则
  \begin{itemize}
  \item $AB$与$BA$:若方阵$A$可逆,则$AB$与$BA$相似
  \item $A^{-1}$与$B^{-1}$:若$A,B$可逆,则$A^{-1}$与$B^{-1}$相似
  \item 多项式:任一多项式$f(x)$,有$f(A),f(B)$相似
  \item 伴随矩阵:若$A,B$相似,则$A^{\ast}$与$B^{\ast}$相似
  \end{itemize}
\end{theorem}

\begin{proof}
  (1)$BA = (A^{-1}A)(BA) = A^{-1}(AB)A$

  (2)$(P^{-1}AP)^{-1} = B^{-1}$,得到$P^{-1}A^{-1}P = B^{-1}$

  (3)显然

  (4)$B^{\ast} = (P^{-1}AP)^{\ast} = P^{\ast}A^{\ast}(P^{-1})^{\ast} = P^{\ast}A^{\ast}(P^{\ast})^{-1}$
\end{proof}

~

\begin{corollary}[伴随矩阵的特征值]
  若矩阵$A$的特征值为$\lambda_1,\cdots,\lambda_m$,则$A^{\ast}$的特征值为$\prod \limits _{i \neq j}\lambda_i$
\end{corollary}

\begin{proof}
  考虑$A$在$\mathbb{C}$上相似于Jordan阵$J$,
  而$J^{\ast}$的特征值易算,相似时特征值不变,因此可算出$A^{\ast}$的特征值。
\end{proof}


\subsection{相似对角化}

\begin{definition}[相似对角化]
  $\mathcal{A}$是$V$内的线性变换,
  如果存在$V$的一组基使得$\mathcal{A}$在该基下的矩阵为对角阵,
  则称$\mathcal{A}$可对角化。
\end{definition}

\begin{definition}[代数重数、几何重数]
  代数重数指特征值$\lambda_i$在特征多项式$f(\lambda)$中的重数。
  几何重数指$\lambda_i$对应特征子空间的维数。
\end{definition}

\begin{lemma}[代数重数与几何重数关系]
  对每个特征值$\lambda_i$,其代数重数大于等于几何重数。
\end{lemma}

\begin{lemma}[不同特征值的特征向量] \label{lemma:不同特征值的特征向量}
  $\mathcal{A}$属于不同特征值的特征向量线性无关。
\end{lemma}

\begin{proof}
  设$\lambda_1,\cdots,\lambda_s$是$A$的$s$个不同的特征值,
  对应特征向量$\xi_1,\cdots,\xi_s$。
  用归纳法,假设$\xi_1,\cdots,\xi_{k-1}$线性无关,
  若
  \begin{equation} \label{equ:不同特征值的特征向量}
    c_1\xi_1 + \cdots + c_k\xi_k = 0
  \end{equation}
  两侧作用$\mathcal{A}$得到$c_1\lambda_1\xi_1 + \cdots + c_k \lambda_k\xi_k = 0$,
  用该式子减去$\lambda_k$倍的(\ref{equ:不同特征值的特征向量})得到
  \begin{equation*}
    c_1(\lambda_1 - \lambda_k)\xi_1 + \cdots + c_{k-1}(\lambda_{k-1} - \lambda_k)\xi_{k-1} = 0
  \end{equation*}
  由于$\lambda_i$两两不相等,因此$c_1 = \cdots = c_{k-1} = 0$,即特征向量彼此线性无关。
\end{proof}

\begin{theorem}[可对角化条件]
  若$V$是$n$维线性空间,则$\mathcal{A}$可对角化的条件
  按照递进关系如下排列:
  \begin{itemize}
  \item 充要条件1:$\mathcal{A}$有$n$个线性无关特征向量。
  \item 充分条件:若$\mathcal{A}$有$n$个不同特征值,则$\mathcal{A}$可对角化
  \item 充要条件2:$V$是所有特征子空间的直和$V = \mathop{\oplus}\limits_{i = 1}^m V_i$(特征子空间维数可能没有$\lambda_i$重数高) 
  \item 充要条件3:所有特征值的代数重数等于几何重数
  \item 充要条件4:$\mathrm{dim} V = \sum\limits_{i = 1}^m \mathrm{dim}(V_i)$
  \item 充要条件5(最小多项式):最小多项式都是一次幂的乘积
  \end{itemize}
\end{theorem}

\begin{proof}
  (1)左推右:由对角化形式,右侧乘开$n$个基向量均是特征向量,显然基向量都线性无关。
  \begin{equation*}
    \left(A \eta_{1}, A \eta_{2}, \cdots, A \eta_{n}\right)=\left(\eta_{1}, \eta_{2}, \cdots, \eta_{n}\right)\left[\begin{array}{lll}
                                                                                                                     \lambda_{1} & & \\
                                                                                                                                 & \lambda_{2} & \\
                                                                                                                                 & \ddots & \lambda_{n}
                                                                                                                   \end{array}\right]
  \end{equation*}

  右推左:$n$个线性无关特征向量当一组基即可。

  (2)根据引理\ref{lemma:不同特征值的特征向量}可知

  后面三个都根据第一个可以推出
\end{proof}

~

\begin{exercise}[对角化计算]
  (1)计算$A = \left(
    \begin{array}{cc}
      3&4\\
      5&2
    \end{array}
  \right)$的特征值,并计算$Q$使得$Q^{-1}AQ$是对角矩阵
\end{exercise}



~

\begin{theorem}[相关矩阵的可对角化性质]
  若矩阵$A$可对角化,则
  \begin{enumerate}
  \item $A^{\ast}$可对角化
  \item $f(x)$为多项式,则$f(A)$可对角化
  \item 若$A$可逆,则$A^{-1}$可对角化
  \end{enumerate}
\end{theorem}

\begin{proof}
  (1)设$P^{-1}AP = \mathrm{diag}\{\lambda_1,\cdots,\lambda_n\}$,
  两侧取伴随得到$P^{\ast}A^{\ast}(P^{\ast})^{-1} = P^{-1}A^{\ast}P = \mathrm{diag}\{\prod \limits_{i \neq 1} \lambda_i, \prod \limits_{i \neq 2}\lambda_i, \cdots, \prod \limits_{i \neq n}\lambda_i\}$。
  其中伴随转逆用了$A^{-1} = \frac{1}{|A|}A^{\ast}$

  (2,3)显然
\end{proof}





\subsection{相似上三角型}

\begin{theorem}[相似上三角型]
  任意$n$阶矩阵$A$都相似于一个上三角矩阵,
  且上三角对角为$A$的特征值。
\end{theorem}

\begin{proof}
  $n = 1$时显然成立。
  假设$n - 1$时成立,
  则取$x_1$为$A$对应$\lambda_1$的特征向量,构造矩阵$P_1 = [x_1,\cdots,x_n]$($x_2 \sim x_n$无要求),
  可以得到:
  \begin{equation*}
    A P_1 = (x_1,\cdots,x_n) \left[
      \begin{array}{cccc}
        \lambda_1&b_{12}&\cdots&b_{1n}\\
        0&b_{22}&\cdots& b_{2n}\\
        \vdots&\vdots&&\vdots\\
        0&b_{n1}&\cdots&b_{nn}
      \end{array}
    \right] \Rightarrow P_1^{-1}AP_1 =
    \left[
      \begin{array}{cccc}
        \lambda_1&b_{12}&\cdots&b_{1n}\\
        0&b_{22}&\cdots& b_{2n}\\
        \vdots&\vdots&&\vdots\\
        0&b_{n1}&\cdots&b_{nn}
      \end{array}
    \right] = 
    \left[
      \begin{array}{cc}
        \lambda_1&*\\
        0&A_1
      \end{array}
    \right]
  \end{equation*}
  根据归纳假设可知对$A_1$有可逆矩阵$Q$使得$Q^{-1}A_1Q$为上三角且对角为特征值,
  取$P_2 = \left[
    \begin{array}{cc}
      1&\mathbf{0}^T\\
      \mathbf{0}&Q
    \end{array}
  \right]$,取$S = P_1P_2$,
  $S^{-1}AS$即所求上三角
\end{proof}



\section{最小多项式}

\subsection{最小多项式与基本性质}

\begin{definition}[零化多项式与最小多项式]
  $A$是数域$K$上的$n$阶方阵,$g(x)$是$K$上的多项式,
  \begin{itemize}
  \item 零化多项式: 若$g(A) = 0$,则称$g(x)$是$A$的一个零化多项式。
  \item 最小多项式:将$A$的次数最低首一零化多项式称为$A$的最小多项式
  \end{itemize}
\end{definition}

\begin{theorem}[最小多项式的性质]
  最小多项式有以下性质:
  \begin{itemize}
  \item 唯一性:最小多项式是唯一的
  \item 整除性:最小多项式整除任意零化多项式
  \item 分块对角:
    $A = \text{diag}\{A_1,\cdots,A_s\}$,则$m_A(\lambda) = [m_1(\lambda),\cdots,m_s(\lambda)]$(最小公倍数)
  \item 相似:相似矩阵有相同最小多项式
  \end{itemize}
\end{theorem}

\begin{proof}
  (1)假设有两个不同最小多项式$m_1(\lambda), m_2(\lambda)$,则令$h(\lambda) = m_1(\lambda) - m_2(\lambda)$,
  有$h( A) = m_1( A) - m_2( A) = \mathbf{0}$,显然$h(\lambda)$是最小多项式,矛盾。

  (2)对任意零化多项式$f(x)$,其可以表示为$f(x) = q(x)m(x) + r(x)$,代入$x = A$得到$r(A) = 0$,
  其只能为$r(x) = 0$,否则与最小多项式矛盾,因此整除。

  (3)由于$m(A) = \text{diag}\{m(A_1),\cdots,m(A_s)\}$,最大公因式即可满足所有分块均为$0$。

  (4)若$B = P^{-1}AP$,则$f(B) = f(P^{-1}AP) = P^{-1}f(A) P$,
  因此可知最小多项式相同。
\end{proof}

\subsection{Hamilton-Cayley定理}

\begin{theorem}[Hamilton-Cayley定理]
  $A$是$K$上的$n$阶矩阵,$f(\lambda) = |\lambda E - A|$即$A$的特征多项式,
  则$f(x)$是$A$的零化多项式。
\end{theorem}

\begin{proof}
  设$B(\lambda)$是$\lambda E - A$的伴随矩阵,
  则$B(\lambda)(\lambda E - A) = f(\lambda )E$,
  由于$B(\lambda)$元素含$1 \sim \lambda^{n-1}$,因此可以写为$B(\lambda) = \lambda^{n-1} B_0 + \cdots + B_{n-1}$,
  从而得到:
  \begin{equation*}
    B(\lambda)(\lambda E - A) = \lambda^n B_0 + \lambda^{n-1}(B_1 - B_0A) + \cdots + \lambda(B_{n-1} - B_{n-2}A) - B_{n-1}A
  \end{equation*}
  根据$B(\lambda)(\lambda E - A) = f(\lambda)E = \lambda^n E + a_1\lambda^{n-1}E + \cdots + a_{n-1}\lambda E + a_nE$进行比较得到:
  \begin{equation*}
    \begin{cases}
      B_0 = E\\
      B_1 - B_0A = a_1E\\
      \quad\quad \vdots\\
      B_{n-1}- B_{n-2}A = a_{n-1}E\\
      -B_{n-1}A = a_nE
    \end{cases}
    \Rightarrow
    \begin{cases}
      B_0A^n = A^n\\
      B_1A^{n-1} - B_0A^n = a_1A^{n-1}\\
      \quad \quad \vdots\\
      B_{n-1}A - B_{n-2}A^2 = a_{n-2}A\\
      -B_{n-1}A = a_nE
    \end{cases}
  \end{equation*}
  两端分别相加,左侧为$\mathbf{0}$,因此右侧为$\mathbf{0}$
\end{proof}

\begin{proof}
  使用Jordan阵证明:分块看,显然每个$\lambda_i$对应的块的阶数小于$\lambda_i$的代数重数,每个分块都是零,
  因此是零化多项式。
\end{proof}

\begin{corollary}[Hamilton-Cayley定理推论]
  Hamilton-Cayley定理有以下推论:
  \begin{itemize}
  \item  最小多项式与特征多项式同根(重数可以不同)。
  \end{itemize}
\end{corollary}

\subsection{最小多项式与相似对角化}

\begin{theorem}[最小多项式与相似对角化]
  $\mathbb{P}$上的$n$阶矩阵$A$可对角化当且仅当$A$的最小多项式在$\mathbb{P}$上可分解为互素一次因子乘积。
\end{theorem}

~

\begin{exercise}[最小多项式判断可对角化]
  (1)幂等矩阵一定可对角化

  (2)对合矩阵$A^2 = E$一定可对角化

  (3)幂零矩阵$A^l = O$不一定可对角化
\end{exercise}

\begin{proof}
  (1)$A(A-E) = O$,一定为一次因子

  (2)$(A+E)(A- E) = O$,一定可对角化

  (3)例如Jordan块
\end{proof}





