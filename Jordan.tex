


\chapter{Jordan标准型}


\section{Jordan标准形:空间分解法}

\subsection{空间第二分解定理:循环不变子空间分解}

\begin{definition}[幂零线性变换]
  $\mathcal{A}$是$V$内的线性变换,若$\exists m$使得$\mathcal{A}^m = 0$,
  则$\mathcal{A}$是幂零线性变换。
\end{definition}

\begin{lemma}[幂零基]
  $\mathcal{A}$是幂零变换,$\forall \alpha \in V$若$\mathcal{A}^{k-1}\alpha \neq 0, \mathcal{A}^k \alpha = 0$,
  则$\alpha, \mathcal{A}\alpha , \cdots, \mathcal{A}^{k-1}\alpha$线性无关。
\end{lemma}

\begin{proof}
  设$a_0 \alpha + a_1 \mathcal{A}\alpha + \cdots + a_{k-1} \mathcal{A}^{k-1}\alpha = 0$,
  若从左至右第一个非零的是$a_i$,即$a_i \mathcal{A}^i \alpha + \cdots + a_k \mathcal{A}^{k-1} \alpha = 0$,
  两侧同时作用$\mathcal{A}^{k-1-i}$,上式变为$a_i \mathcal{A}^{k-1}\alpha = 0$,
  而$\mathcal{A}^{k-1}\alpha \neq 0$,故$a_i = 0$,与假设矛盾。
\end{proof}

\begin{definition}[循环不变子空间]
  对$\forall \alpha \in V$,$\exists k$使得$\mathcal{A}^k = \mathbf{0}, \mathcal{A}^{k-1} \neq \mathbf{0}$,
  则$I(\alpha) := L(\alpha, \mathcal{A}\alpha,\cdots, \mathcal{A}^{k-1}\alpha)$称为$\alpha$生成的$\mathcal{A}$的循环不变子空间。
\end{definition}

\begin{theorem}[循环不变子空间分解]
  对于$V$内的幂零变换$\mathcal{A}$,$\exists v_1,\cdots,v_n \in V$,$m_1,\cdots,m_n \in \mathbb{N}$使得$V = I(v_1) \oplus I(v_2) \oplus \cdots \oplus I(v_n)$。
  其中$I(v_i) = L( v_i, \mathcal{A}v_i \cdots, \mathcal{A}^{m_i} v_i)$,且$\mathcal{A}^{m_i + 1} v_i = 0$。
\end{theorem}

\begin{proof}
  采用数归与核像分解。(1)若$dim V = 1$,则显然可分解。
  (2)若$dim V > 1$,则用归纳法。
  由于$\mathcal{A}$不是单射(因为将不同的向量均映射到$0$),因此$\mathcal{A}$不是满射(因为线性映射是满射当且仅当是双射),因此$\mathcal{A}$的像维数低于$dim V$。
  对$Im \mathcal{A}$用归纳假设,得到$Im \mathcal{A}$的一组基$\mathcal{A}^{m_1} v_1,\cdots, \mathcal{A}v_1,v_1, \cdots, \mathcal{A}^{m_n}v_n,\cdots,v_n$,
  由于上述向量组是$Im \mathcal{A}$的一组基,对$\forall j$存在$V$中向量$u_j$使得$v_j = \mathcal{A}u_j$。

  下面说明下面的向量组线性无关:
  \begin{equation} \label{equ:循环不变子空间分解1}
    \mathcal{A}^{m_1 + 1}u_1, \cdots, \mathcal{A}u_1,u_1,\cdots, \mathcal{A}^{m_n+1}u_n ,\cdots,u_n
  \end{equation}

  原因:可设上述向量的一个线性组合为$0$,
  两侧同时作用$\mathcal{A}$,则仅有$\mathcal{A}^{m_1+1}u_1, \cdots, \mathcal{A}^{m_n + 1}u_n$的系数可以非零(因为它们被作用后为0)。
  去除其他项后,相当于设上面这组向量的某个线性组合等于$0$,由于它们也是$Im \mathcal{A}$基的一部分,彼此线性无关,所以它们的系数也必须为零。

  于是将(\ref{equ:循环不变子空间分解1})扩成$V$的一组基$\mathcal{A}^{m_1 + 1}u_1,\cdots,u_1,\cdots, \mathcal{A}^{m_n+1}u_n ,\cdots, u_n, w_1, \cdots, w_p$,
  由于$\mathcal{A} w_j \in Im \mathcal{A}$,因此存在(\ref{equ:循环不变子空间分解1})张成空间中元素$x_j$,使得$\mathcal{A} w_j = \mathcal{A} x_j$。
  令$u_{n+j} = w_j - x_j$,则有$\mathcal{A} u_{n+j} = 0$,
  从而(\ref{equ:循环不变子空间分解1})配上$u_{n+1},\cdots,u_{n+p}$是所需$V$的一组基。
\end{proof}

\subsection{核像升链与根子空间}

设$\mathcal{A}$是$V$上的线性变换,
定义$M_0 = \{0\}, M_i = \text{Ker} \mathcal{A}^i$,
$N_0 = V, N_i = \text{Im}( \mathcal{A}^i)$,
下面来研究它们的性质。

\begin{lemma}[核像空间升链] \label{lemma:M和N的包含关系}
  上述空间有:$\{0\} = M_0 \subseteq M_1 \subseteq M_2 \subseteq \cdots$,
  $V = N_0 \supseteq N_1 \supseteq N_2 \supseteq \cdots$
\end{lemma}

\begin{proof}
  (1)对$\forall \alpha \in M_i$,由定义有$\mathcal{A}^i\alpha = 0$,
  显然$\mathcal{A}^{i+1}\alpha =0$,因此$M_i \subseteq M_{i+1}$。

  (2)对于$\alpha \in N_i$,由定义$\exists \beta \in V$使得$\alpha = \mathcal{A}^i \beta$,
  从而$\alpha = \mathcal{A}^{i-1}( \mathcal{A} \beta) \in N_{i-1}$。
  由$\alpha$任意性可知$N_i \subseteq N_{i-1}$。
\end{proof}

\begin{lemma}[核空间增长]
  核空间$M_i$的增长满足以下性质:
  \begin{itemize}
  \item 若$M_k = M_{k+1}$,则$M_k = M_{k+1} = M_{k+2} = \cdots$
  \item 若$dim V = n$,则$M_n = M_{n+1} = M_{n+2} = \cdots$。
  \end{itemize}
\end{lemma}

\begin{proof}
  (1)显然有$M_{k+1} \subseteq M_{k+2}$,只需要证明$M_{k+2} \subseteq M_{k+1}$。
  对于$\forall \alpha \in M_{k+2}$,有$\mathcal{A}^{k+2}\alpha = 0$,即$\mathcal{A}^{k+1}( \mathcal{A}\alpha) = 0$,
  因此$\mathcal{A} \alpha \in M_{k+1} = M_k$,即$\mathcal{A}^{k+1}\alpha = 0 \rightarrow \alpha \in M_{k+1}$,
  由任意性可知$M_{k+2} \subseteq M_{k+1}$,故$M_{k+1} = M_{k+2}$,后续类似。

  (2)只需要证明$M_n = M_{n+1}$,假设不然,
  则$\{0\} = M_0 \subset M_1 \subset \cdots \subset M_n \subset M_{n+1}$,$dim M_{n+1} > n$,与$M_{n+1} \subseteq V$矛盾。
\end{proof}

\begin{theorem}[幂次变换核像分解] \label{thm:根子空间核像分解}
  $V = M_n \oplus N_n$,其中$n = dim V$。
\end{theorem}

\begin{proof}
  显然由核像性质有$dim M_n + dim N_n = n$,只需证$M_n \cap N_n = \{0\}$。
  $\forall v \in M_n \cap N_n$,有$\mathcal{A}^n v = 0$,且$\exists u \in V$使得$v = \mathcal{A}^n u$,
  从而$\mathcal{A}^{2n} u = 0$,而$M_{2n} = M_{2n-1} = \cdots = M_n$,
  故$\mathcal{A}^n u = 0$,从而$v = 0$。
\end{proof}

\begin{definition}[广义特征向量与根子空间(广义特征子空间)]
  \begin{itemize}
  \item 广义特征向量:线性变换$\mathcal{A}$关于特征值$\lambda$的广义特征向量定义为满足$\exists j \in \mathbb{N}^+, ( \mathcal{A} - \lambda E)^j v = 0$的所有向量。
  \item  根子空间:线性变换$\mathcal{A}$关于$\lambda$的所有广义特征向量加上零向量组成的空间称为$\mathcal{A}$关于特征值$\lambda$的根子空间,
    记作$G(\lambda, \mathcal{A})$。
  \end{itemize}
\end{definition}

\begin{theorem}[根子空间的具体表达] \label{lemma:根子空间表达式}
  根子空间$G(\lambda, \mathcal{A})$满足$G(\lambda, \mathcal{A}) = \text{Ker}( \mathcal{A} - \lambda E)^{dim V}$。
\end{theorem}

\begin{proof}
  右推左:若$v \in Ker( \mathcal{A} - \lambda E)^{dim V}$,则由$G(\lambda, \mathcal{A})$定义可知$v \in G(\lambda, \mathcal{A})$。
  左推右:$v \in G(\lambda, \mathcal{A})$即$\exists j \in \mathbb{N} \text{ s.t. }(\mathcal{A} - \lambda E)^j v = 0$,
  而前面已证明这样的$v$必然满足$v \in Ker( \mathcal{A} - \lambda E)^{dim V}$。
\end{proof}

\begin{note}
  为了知识体系的完整,这里补充一下:$\lambda_i$根子空间的次数最低压缩为$\lambda_i$对应Jordan块的最高阶数,即最小多项式中的$\lambda_i$的阶数。
  其他书也常用特征多项式的代数重数表示(题目中会遇到),
  上面引理的dimV明显放缩过大。
\end{note}

\subsection{空间第一分解定理:根子空间分解}

\begin{theorem}[根子空间的不变性]
  根子空间$G(\lambda, \mathcal{A})$是$\mathcal{A}$不变的。
\end{theorem}

\begin{proof}
  由定理\ref{lemma:根子空间表达式}可知$G(\lambda, \mathcal{A}) = Ker( \mathcal{A} - \lambda E)^n$,
  $\forall \alpha \in G(\lambda, \mathcal{A})$有$( \mathcal{A} - \lambda E)^n \alpha = 0$,
  显然$( \mathcal{A} - \lambda E)^n \mathcal{A} \alpha = \mathcal{A}( ( \mathcal{A} - \lambda E)^n \alpha) = 0$,故$\mathcal{A} \alpha \in G(\lambda, \mathcal{A})$。
\end{proof}

\begin{theorem}[空间第一分解定理:根子空间分解]
  若$V$是复线性空间,$\mathcal{A}$是线性变换,$\lambda_1,\cdots,\lambda_m$是其特征值,
  则$V = G(\lambda_1, \mathcal{A}) \oplus \cdots \oplus G(\lambda_m, \mathcal{A})$。
\end{theorem}

\begin{proof}
  用归纳法,(1)$n = 1$时显然成立(2)假设对所有小于$n$维空间均成立,由定理\ref{thm:根子空间核像分解}可知$V = G(\lambda_1, \mathcal{A}) \oplus Im ( \mathcal{A} - \lambda_1 E)^n$,
  若$U = Im ( \mathcal{A} - \lambda_1 E)^n$为空,则显然成立。
  其他情况首先显然$\mathcal{A}$在$U$上是不变的($\mathcal{A}$与$( \mathcal{A} - \lambda_1 E)$可交换),
  去除$\lambda_1$所有对应特征向量后,
  $\mathcal{A}|_U$在$U$有$\lambda_2,\cdots,\lambda_m$特征值,
  对$U$用归纳法,$U = G(\lambda_2, \mathcal{A}|_U) \oplus \cdots \oplus G(\lambda_m, \mathcal{A}|_U)$,
  由于$\forall j \geq 2, G(\lambda_j, \mathcal{A}|_U) = G(\lambda_j, \mathcal{A})$,
  从而得证。
\end{proof}

\subsection{Jordan标准形:空间分解法}

% \begin{theorem}[上三角标准型]
%   若$V$是复线性空间,$\mathcal{A}$是$V$上的线性变换,
%   若$\lambda_1,\cdots,\lambda_m$是重数为$d_1,\cdots,d_m$的特征值,
%   则存在一组基使得$\mathcal{A}$在该基下的矩阵为
%   $\left[
%   \begin{array}{ccc}
%     \lambda_j& & *\\
%              & \ddots & \\
%     0& &\lambda_j
%   \end{array}\right] _{d_j \times d_j}
%   $组成的分块对角矩阵。
% \end{theorem}


\begin{theorem}[幂零变换的Jordan标准型]
  幂零变换$\mathcal{A}$在循环不变子空间分解基下的矩阵如下。
  称为幂零变换的Jordan标准形。
\end{theorem}

\begin{equation*}
  J=\left[\begin{array}{cccc}
            J_{1} & & & \\
                  & J_{2} & & 0 \\
            0 & & \ddots & \\
                  & & & J_{s}
          \end{array}\right], \quad J_{i}=\left[\begin{array}{cccc}
                                                  0 & 1 & & 0 \\
                                                    & 0 & \ddots & \\
                                                    & & \ddots & 1 \\
                                                  0 & & & 0
                                                \end{array}\right]_{n_{i} \times n_{i}}
\end{equation*}

\begin{proof}
  代入计算显然可得。
\end{proof}

\begin{theorem}[核限制变换是幂零变换]
  对于$\forall$线性变换$\mathcal{A}$,
  令$\mathcal{B} = \mathcal{A} - \lambda E$,
  则$\mathcal{B}$在$M_k = Ker \mathcal{B}^k$中的限制变换$\mathcal{B}|_{M_k}$是幂零变换。
\end{theorem}

\begin{proof}
  由于$M_k = \text{Ker} \mathcal{B}^k$,因此$\forall \alpha \in M_k$有$\mathcal{B}^k \alpha = 0$,
  故$( \mathcal{B} |_{M_k})^k = 0$。
\end{proof}

\begin{theorem}[Jordan标准型:空间分解法]
  对于$\mathbb{C}$上线性变换$\mathcal{A}$而言,
  其有在先根子空间分解,根子空间再循环不变子空间分解的基下有以下形式。
  称为线性变换的Jordan标准形。
\end{theorem}


\begin{equation*}
  J=\left[\begin{array}{cccc}
            J_{1} & & & 0 \\
                  & J_{2} & & \\
                  & & \ddots & \\
            0 & & & J_{s}
          \end{array}\right], \quad J_{i}=\left[\begin{array}{cccc}
                                                  \lambda_{i} & 1 & & 0 \\
                                                              & \lambda_{i} & \ddots & \\
                                                              & & \ddots & 1 \\
                                                  0 & & & \lambda_{i}
                                                \end{array}\right]_{n_{i} \times n_{i}}
\end{equation*}

\begin{proof}
  先由空间分解第一定理将$V$分解为根子空间直和,
  由于$\mathcal{A} - \lambda E$在$G(\lambda, \mathcal{A})$中的限制变换是幂零变换,
  因此再对$G(\lambda, \mathcal{A})$进行循环不变子空间分解,再由幂零变换的Jordan标准形可证。
\end{proof}



\section{空间分解法:零化多项式角度}

\subsection{根子空间分解:矩阵角度}

\begin{theorem}[解空间分解]
  $A$是$n$阶方阵,
  $f(x),f_1(x),f_2(x) \in P[x]$,
  且$f(x) = f_1(x)f_2(x), (f_1(x),f_2(x)) = 1$,
  设$f(A)X = 0, f_1(A)X = 0, f_2(A)X = 0$的解空间分别为$V,V_1,V_2$,
  则
  \begin{equation*}
    V = V_1 \oplus V_2
  \end{equation*}
\end{theorem}

\begin{proof}
  由于互素,得到$\exists u(x),v(x)$,使得$E = u(A)f_1(A) + v(A) f_2(A)$,
  从而$\forall \alpha \in V$有
  \begin{equation*}
    \alpha = u(A)f_1(A)\alpha + v(A) f_2(A)\alpha := \alpha_1 + \alpha_2
  \end{equation*}
  且满足$f(A)\alpha = f_1(A)f_2(A)\alpha = 0$。
  故考察$f_2(A)\alpha_1$:
  \begin{equation*}
    f_2(A)\alpha_1 = f_2(A) u(A)f_1(A)\alpha = u(A)f_1(A)f_2(A) = 0 \Rightarrow \alpha_1 \in V_2
  \end{equation*}
  同理得到$f_1(A)\alpha_2 = 0, \alpha_2 \in V_1$,
  从而得到$V = V_1 + V_2$。

  下证直和,
  $\forall \alpha \in V_1 \cap V_2$,
  则$f_1(A)\alpha = f_2(A) \alpha = 0$,
  因此
  \begin{equation*}
    \alpha = u(A)f_1(A) \alpha + v(A)f_2(A)\alpha = 0
  \end{equation*}
  因此结论成立。
\end{proof}

\begin{theorem}[多空间分解]
  $A$是数域$P$上$n$阶方阵,
  $f(x),f_1(x),\cdots, f_s(x) \in P[x]$,
  $f(x) = f_1(x)f_2(x)\cdots f_s(x)$,
  且$f_1(x),\cdots,f_s(x)$ \textbf{两两互素},
  设$f(A)X = 0, f_1(A)X = 0,\cdots,f_s(A)X = 0$的解空间为$V,V_1,\cdots,V_s$,
  则
  \begin{equation*}
    V = V_1 \oplus V_2 \oplus \cdots \oplus V_s
  \end{equation*}
\end{theorem}

\begin{proof}
  用数学归纳法即可。
\end{proof}

~

\begin{theorem}[空间分解研究矩阵秩]
  考虑$f(A) = f_1(A)\cdots f_s(A)$,
  则
  \begin{equation*}
    n - \mathrm{r}(f(A)) = n - \mathrm{r}(f_1(A)) + \cdots + n - \mathrm{r}(f_s(A))
  \end{equation*}
\end{theorem}

\begin{proof}
  设$f(A)X = 0, f_1(A)X = 0,\cdots,f_s(A) X = 0$的解空间为$V,V_1,\cdots,V_s$,
  根据解空间分解得到$V = V_1 \oplus \cdots \oplus V_s$,
  因此根据解空间的维数可得到结论。
\end{proof}

~

\begin{exercise}[空间分解研究矩阵的秩]
  证明以下命题

  (1)$A^2 = A$当且仅当$r(A) + r(A - E) = n$

  (2)$A^2 = E$当且仅当$r(A+E) + r(A - E) = n$

  (3)$A^3 = E$当且仅当$r(A - E) + r(A^2 + A + E) = n$
\end{exercise}


\begin{proof}
  (1)$f(A) = A^2 - A = A(A - E)$,
  因此$n - r(A^2 - A) = n - r(A) + n - r(A - E)$,
  左侧等于$n$,右侧等于$2n - r(A) - r(A - E)$,
  因此$r(A) + r(A - E) = n$
\end{proof}

~

\subsection{根子空间分解:线性变换角度}

\begin{lemma}[零变换的核空间]
  $V$是$\mathbb{P}$上的线性空间,$\mathcal{O}$是零变换,则其核空间为整个空间:
  \begin{equation*}
    \mathrm{Ker}(\mathcal{O}) = V
  \end{equation*}
\end{lemma}

\begin{theorem}[核空间分解]
  $\mathcal{A}$为线性变换,
  $f(x) = f_1(x)f_2(x)$为多项式,
  且$(f_1(x),f_2(x)) = 1$,
  则有如下核空间分解:
  \begin{equation*}
    \mathrm{Ker}(f(\mathcal{A})) = \mathrm{Ker}(f_1(\mathcal{A})) \oplus \mathrm{Ker}(f_2(\mathcal{A}))
  \end{equation*}
\end{theorem}

\begin{proof}
  根据$(f_1(x),f_2(x)) = 1$得到$\exists u(x),v(x)$使得$u(x)f_1(x) + v(x)f_2(x) = 1$,
  因此
  \begin{equation*}
    \mathcal{E} = u(\mathcal{A})f_1(\mathcal{A}) + v(\mathcal{A}) f_2(\mathcal{A})
  \end{equation*}
  对$\forall \alpha \in \mathrm{Ker}(f(\mathcal{A}))$
  \begin{equation*}
    \alpha = \alpha_1 + \alpha _2 := u(\mathcal{A})f_1(\mathcal{A})\alpha + v(\mathcal{A})f_2(\mathcal{A})\alpha
  \end{equation*}
  显然$f_2(\mathcal{A})\alpha_1 = u(\mathcal{A})f(\mathcal{A})\alpha = 0$,
  得到$\alpha_1 \in \mathrm{Ker}(f_1(\mathcal{A}))$,同理有$\alpha_2 \in \mathrm{Ker}(f_2(\mathcal{A}))$,综上得到
  \begin{equation*}
    \mathrm{Ker}(f(\mathcal{A})) = \mathrm{Ker}(f_1(\mathcal{A})) + \mathrm{Ker}(f_2(\mathcal{A}))
  \end{equation*}

  下证直和:对$\forall \alpha \in \mathrm{Ker}(f_1(\mathcal{A})) \cap \mathrm{Ker}(f_2(\mathcal{A}))$,有
  \begin{equation*}
    \alpha = u(\mathcal{A})f_1(\mathcal{A})\alpha + v(\mathcal{A})f_2(\mathcal{A})\alpha = 0
  \end{equation*}
  因此为直和
\end{proof}

\begin{theorem}[全空间分解:零化多项式角度]
  $V$是$\mathbb{P}$上的线性空间,$\mathcal{A}$是$V$上的线性变换,
  $f(x) = f_1(x) f_2(x) \cdots f_s(x)$为零化多项式,
  且$f_1(x),\cdots,f_s(x)$两两互素,则
  \begin{equation*}
    V = \mathrm{Ker}(f_1(\mathcal{A})) \oplus \cdots \oplus \mathrm{Ker}(f_s(\mathcal{A}))
  \end{equation*}
\end{theorem}

\begin{proof}
  由于$f(A) = \mathcal{O}$,根据零变换的核空间以及核空间分解定理可知。
\end{proof}


\begin{lemma}[Hamilton-Cayley]
  若$f(\lambda)$是线性变换$\mathcal{A}$的特征多项式,则$f(\lambda)$是$\mathcal{A}$的零化多项式
\end{lemma}

\begin{proof}
  该定理在最小多项式一章已经证明过了
\end{proof}

\begin{theorem}[根子空间分解:零化多项式角度]
  若$\mathbb{C}$上线性变换$\mathcal{A}$的特征多项式为$f(\lambda) = (\lambda - \lambda_1)^{k_1} \cdots (\lambda - \lambda_s)^{k_s}$,
  最小多项式为$m(\lambda )= (\lambda - \lambda_1)^{l_1} \cdots (\lambda - \lambda_s)^{l_s}$,则
  有以下根子空间分解
  \begin{equation*}
    V = \mathrm{Ker}(\mathcal{A} - \lambda_1 \mathcal{E})^{k_1}\oplus \cdots \oplus \mathrm{Ker}(\mathcal{A} - \lambda_s \mathcal{E})^{k_s} = \mathrm{Ker}(\mathcal{A} - \lambda_1 \mathcal{E})^{l_1} \oplus \cdots \oplus \mathrm{Ker}(\mathcal{A} - \lambda_s \mathcal{E})^{l_s}
  \end{equation*}
\end{theorem}

\begin{proof}
  根据Hamilton-Cayley定理以及零化多项式全空间分解可知。
\end{proof}

~

\begin{exercise}[根子空间分解:零化多项式角度]
  (1)ZJU2021.8:$V$是线性空间,$\varphi$是$V$上线性变换,特征多项式$f(\lambda) = (\lambda - 2)^6(\lambda + 2)^4$,
  将$V$分解为两个非平凡不变子空间的直和,并证明结论。
\end{exercise}

~

\begin{theorem}[根子空间分解与Jordan标准型]
  根子空间分解将Jordan阵按照特征值$\lambda_i$分为若干份,每份由$\lambda_i$为对角的若干个Jordan块组成,
  对角$\lambda_i$在Jordan阵中的出现次数等于其代数重数。
\end{theorem}


\subsection{Jordan阵与循环不变子空间}

\begin{lemma}[核限制变换是幂零变换]
  $\lambda_i$为$\mathcal{A}$的特征值,则$(\mathcal{A} - \lambda_i \mathcal{E}) \big|_{\mathrm{Ker}(\mathcal{A} - \lambda_i \mathcal{E})^k}$在$\mathrm{Ker}(\mathcal{A} - \lambda_i \mathcal{E})^k$中为幂零变换
\end{lemma}

\begin{proof}
  由于$\forall \alpha \in \mathrm{Ker}(\mathcal{A} - \lambda_i \mathcal{E})^k $满足
  $(\mathcal{A} - \lambda_i\mathcal{E})^k \alpha = 0$,
  因此限制变换满足$(\mathcal{A} - \lambda_i\mathcal{E})^k\big|_{\mathrm{Ker}(\mathcal{A} - \lambda_i\mathcal{E})^k} = \mathcal{O}$
\end{proof}

\begin{theorem}[循环不变子空间分解]
  对于$V$内的幂零变换$\mathcal{A}$,$\exists v_1,\cdots,v_n \in V$,$m_1,\cdots,m_n \in \mathbb{N}$使得$V = I(v_1) \oplus I(v_2) \oplus \cdots \oplus I(v_n)$。
  其中$I(v_i) = L( v_i, \mathcal{A}v_i \cdots, \mathcal{A}^{m_i} v_i)$,且$\mathcal{A}^{m_i + 1} v_i = 0$。
\end{theorem}

\begin{proof}
  在空间第二分解定理中已经证明过了。
\end{proof}

\begin{theorem}[Jordan阵与循环不变子空间]
  线性变换$\mathcal{A}$在根子空间$\mathrm{Ker}(\mathcal{A} - \lambda_i\mathcal{E})^k$下的限制变换$\mathcal{A} \big|_{\mathrm{Ker}(\mathcal{A} - \lambda_i \mathcal{E})^k}$在一组循环不变基$((\mathcal{A} - \lambda_i\mathcal{E})^s\alpha, (\mathcal{A} - \lambda_i \mathcal{E})^{s-1}\alpha,\cdots,\alpha)$下的矩阵为一个Jordan块:
  \begin{equation*}
    J_s(\lambda_i) = \left[
      \begin{array}{ccccc}
        \lambda_i&1&&& \\
                 &\lambda_i&1&&\\ 
                 &&\ddots&\ddots& \\ 
                 &&&\lambda_i&1\\
                 &&&&\lambda_i
      \end{array}
    \right]
  \end{equation*}
\end{theorem}

\begin{proof}
  直接验证即可。
\end{proof}


\begin{theorem}[Jordan阵与几何重数]
  特征值$\lambda_i$的代数重数是特征多项式$f(\lambda)$的$\lambda = \lambda_i$根的重数,
  几何重数($\lambda_i E - A = 0$解空间维数)是$\lambda_i$对应以$\lambda_i$为对角的Jordan块个数。
\end{theorem}

\begin{proof}
  根据$\lambda_i E - A = 0$的解空间维数等于$\lambda_i E - J = 0$解空间维数,
  而每有一个以$\lambda_i$为对角的Jordan块,$\lambda_i E - J$上就有一行全为$0$,
  因此秩减$1$,从而块数等于几何重数。
\end{proof}

\begin{note}
  即以$\lambda_i$为对角的Jordan块个数为$n - r(\lambda E - A)$,
  反过来写也要分得清。
\end{note}


\section{Jordan标准形:$\lambda$矩阵法}

\subsection{$\lambda$矩阵与相抵}

\begin{definition}[$\lambda$矩阵]
  $A(\lambda)$是元素为$\lambda$多项式的矩阵,则将$A(\lambda)$称为$\lambda$矩阵。
\end{definition}

\begin{definition}[$\lambda$矩阵初等变换]
  $\lambda$矩阵有三种初等变换:
  \begin{itemize}
  \item 互换两行/列
  \item 用常数c乘一行/列(不能乘$\lambda$多项式!)
  \item 将一行/列乘以一个$\lambda$多项式加到另一行/列上
  \end{itemize}
\end{definition}

\begin{definition}[相抵]
  若可以用初等变换将一个$\lambda$矩阵转换为另一个$\lambda$矩阵,
  则称两个$\lambda$矩阵相抵。
\end{definition}

\begin{theorem}[相抵标准形:法式]
  对于数值矩阵$A$,$\lambda E - A$总是相抵于$\text{diag}\{1,\cdots,1,d_1(\lambda), \cdots, d_m(\lambda)\}$,
  其中$d_i(\lambda)$是首一多项式,且$d_i(\lambda)|d_{i+1}(\lambda)$。
\end{theorem}

\begin{lemma}[对角整除化]
  $A(\lambda)$为非零$\lambda$矩阵,
  则$A(\lambda)$一定相抵于$B(\lambda)$,
  满足$b_{11}(\lambda) \neq 0$且$b_{11}(\lambda)$可整除$B(\lambda)$中的所有元素
\end{lemma}

\begin{proof}
  先将$A(\lambda)$中次数最低的元素放置于对角$a_{11}(\lambda)$
  
  (1)消去第一行、第一列:不妨设$a_{i1}(\lambda)$不被$a_{11}(\lambda)$整除,
  则做带余除法$a_{i1}(\lambda) = q(\lambda)a_{11}(\lambda) + r(\lambda)$,
  再将$r(\lambda)$放置于$a_{11}$,直至全部整除,可消去第一列和第一行。

  (2)考虑第一列、第一行外的其他元素,若存在$a_{ij}(\lambda)$不被$a_{11}$整除,
  则把第$j$列加到第一列上,做带余除法,把余项放到$a_{11}$,以此类推。
\end{proof}

\begin{theorem}[法式的计算:初等变换法]
  每次把最低次移动到角上,尝试消除,消除过程按照以下步骤
  \begin{itemize}
  \item 通过初等变换将当前对角元素变为所有非零元素中次数最低的$\lambda$多项式
  \item 将矩阵对角整除化:具体见引理的过程
  \item 消去一列,再消去一行,重复对角整除化和消去
  \end{itemize}
\end{theorem}

\begin{theorem}[相抵与相似]
$A$与$B$相似当且仅当$\lambda E - A$与$\lambda E - B$相抵。
\end{theorem}

\begin{proof}
  (1)左推右:若$B = P^{-1}AP$,则$\lambda E - B = P^{-1}(\lambda E - A)P$,
  显然右侧两个$\lambda$矩阵相抵。

  (2)右推左:若$\lambda E - B = P(\lambda)(\lambda E - A)Q(\lambda)$,
  设$Q(\lambda) = Q_k\lambda^k + \cdots + Q_1\lambda + Q_0$,
  $Q(\lambda)^{-1} = R_m\lambda^m + \cdots + R_1\lambda + R_0$,
  下记$W = Q(B)$。由$Q(\lambda) Q^{-1}(\lambda) = E$得
  $R_mQ(\lambda)\lambda^m + \cdots + R_1Q(\lambda)\lambda + R_0Q(\lambda) = E$,
  代入$B$得到$R_mWB^m + \cdots + R_1WB + R_0W = E$。

  由于$P(\lambda)^{-1}(\lambda E - B) = (\lambda E - A)Q(\lambda) = Q(\lambda)\lambda - A Q(\lambda)$,代入$\lambda = B$得到$Q(B)B = AQ(B)$(前面一式子左=$0$),
  即$WB = AW$,从而$WB^l = A^lW$(多作用一个$A$相当于多右乘一个$B$)。

  再由前面的式子得到$(R_mA^m + \cdots + R_0)W = E$,则表明$W$可逆,
  从而$B = W^{-1}AW$。
\end{proof}

\subsection{行列式因子、不变因子}

\begin{definition}[行列式$k$阶子式]
  $|A|$的$k$阶子式是其任选$k$行、$k$列组成的矩阵的行列式。
\end{definition}

\begin{definition}[行列式因子]
  $A(\lambda)$是$\lambda$矩阵,若其所有$k$阶子式(行列式)均为$0$,
  则定义其$k$阶行列式因子$D_k(\lambda) = 0$,
  否则定义为所有非零$k$阶子式的首一最大公因式。
\end{definition}

\begin{note}
  一般行列式因子比初等变换方便,不过一定注意这里的子式是所有子式,
  而非主子式,甚至可以不连着,$0$的情况也得分开考虑。
\end{note}

\begin{lemma}[行列式因子的相除性]
  设$D_1(\lambda),\cdots,D_r(\lambda)$是$A(\lambda)$非零行列式因子,
  则$D_i(\lambda) | D_{i+1}(\lambda)$。
\end{lemma}

\begin{proof}
  设$A_{i+1}$是$A(\lambda)$任意$i+1$阶子式,
  这个行列式按某一行展开,则其每一展开项都是一个多项式与一个$i$阶子式乘积,
  而$D_i(\lambda)$是所有$i$阶子式公因子,因此$D_i(\lambda) | A_{i+1}(\forall i)$,
  而$D_{i+1}(\lambda)$是所有$A_{i+1}$的最大公因式,因此证毕。
\end{proof}

\begin{definition}[不变因子]
  定义$d_1(\lambda ) = D_1(\lambda)$,$d_i(\lambda) = \frac{D_i(\lambda)}{D_{i-1}(\lambda)}(\forall i \neq 1)$为$A(\lambda)$的不变因子。
\end{definition}

\begin{theorem}[行列式因子、不变因子的性质]
  \begin{itemize}
  \item 相抵矩阵的因子:相抵的$\lambda$矩阵有相同的行列式因子、不变因子(初等变换不改变行列式、不变因子)
  \item 法式与不变因子:
    $\lambda E - A$法式的所有对角元素即$\lambda E - A$的不变因子。
  \end{itemize}
\end{theorem}

\begin{proof}
  (1)只需证明行列式因子在三种初等变换不变即可。
  而不变因子由行列式因子唯一确定。

  (2)由于不变因子在初等变化中不变,求出法式的不变因子(即对角),因此$\lambda E - A$的不变因子即其法式对角元素。
\end{proof}

\begin{theorem}[不变因子与最小多项式]
  $A$的最小多项式即$\lambda E - A$的最后一个不变因子
\end{theorem}

\begin{proof}
  设$A$的互异特征值为$\lambda_1,\cdots,\lambda_s$,
  因此可以列出所有的初等因子(具体见下一节),
  最小多项式即初等因子的最小公倍式,
  最后一个不变因子也是,
  因此结论成立
\end{proof}

\subsection{初等因子与Jordan标准型}

\begin{definition}[初等因子]
将所有不变因子分解为不可约因子之积,这些不可约因子全体称为初等因子组。
\end{definition}

\begin{theorem}[对角化计算初等因子] \label{lemma:对角化计算初等因子}
  $\lambda E - A$经过初等变化变为对角阵(不一定要法式)
  \begin{equation*}
    \left(\begin{array}{llll}
            f_{1}(\lambda) & & & \\
                           & f_{2}(\lambda) & & \\
                           & & \ddots & \\
                           & & & f_{n}(\lambda)
          \end{array}\right)
  \end{equation*}
  其中$f_i(\lambda)$为非零首一多项式,则$A$的初等因子组是所有$f_i(\lambda)$做不可约分解得到的因式组合。
\end{theorem}

\begin{lemma}[Jordan块、Jordan阵的初等因子]
  $r$阶$\lambda_0$对角的Jordan块的初等因子组为$(\lambda - \lambda_0)^r$。
  对角为$J_1,J_2,\cdots,J_k$的Jordan阵的初等因子组为$(\lambda - \lambda_1)^{r_1}, \cdots, (\lambda - \lambda_k)^{r_k}$。
\end{lemma}

\begin{proof}
  (1)Jordan块的初等因子组直接用行列式因子计算即可(只有$r$阶非$1$,因此不变因子好算)

  (2)Jordan阵的初等因子进行分块初等变换,将各个Jordan块分块变成法式,
  再用引理\ref{lemma:对角化计算初等因子}即可。
\end{proof}

\begin{theorem}[使用初等因子计算Jordan标准型]
  若$K$上的矩阵$A$有初等因子组$(\lambda - \lambda_1)^{r_1}, \cdots, (\lambda - \lambda_k)^{r_k}$,
  则$A$相似于Jordan阵$J = \text{diag}\{J_1,\cdots,J_k\}$,
  其中$J_i$是以$\lambda_i$为对角的$r_i$阶Jordan块。
\end{theorem}

\begin{theorem}[Jordan标准型存在充要条件]
  $K$上矩阵$A$有Jordan标准型当且仅当其初等因子都是一次项的幂次乘积。
\end{theorem}



\section{$\lambda$矩阵的进一步研究}

\subsection{Jordan标准型的直接计算}

一、初等变换方法:只需要将$\lambda E - A$转换为对角矩阵,对角所有分解即初等因子组,直接构造Jordan标准型

~

\begin{exercise}[无法直接整除的情况]
  (1)计算下面矩阵$A$的法式:
  \begin{equation*}
    A = \left[
      \begin{array}{ccc}
        -1&1&3 \\
          3&0&-4 \\
          -2&1&4
      \end{array}
    \right]
  \end{equation*}
\end{exercise}

\begin{solution}
  (1)进行到下面步骤的时候无法整除,即发现不整除列上的多项式,要进行对角整除化
  \begin{equation*}
    \left[
      \begin{array}{ccc}
        1&0&0 \\
         0&\lambda-1&-\lambda+1 \\
         0&\lambda^2 + \lambda - 3&-3\lambda + 4
      \end{array}
    \right] \rightarrow \left[
      \begin{array}{ccc}
        1&0&0 \\
         0&\lambda - 1&- \lambda + 1 \\
         0&(\lambda^2 + \lambda - 3) - (\lambda-1)(\lambda+2)&(-3\lambda + 4) - (\lambda+2)(-\lambda + 1)
      \end{array}
    \right] = \left[
      \begin{array}{ccc}
        1&0&0 \\
         0&\lambda - 1&-\lambda + 1 \\
         0&-1&\lambda^2 - 2\lambda +2
      \end{array}
    \right]
  \end{equation*}
  因此最终可以化简为:
  \begin{equation*}
    \left[
      \begin{array}{ccc}
        1&& \\
         &1& \\
         &&(\lambda-1)^3
      \end{array}
    \right]
  \end{equation*}
\end{solution}

~

\begin{exercise}[初等变换方法计算Jordan标准型]
  计算$A = \left[
    \begin{array}{ccc}
      1&-1&2\\
      3&-3&6\\
      2&-2&4
    \end{array}
  \right]$的Jordan标准型
\end{exercise}

\begin{solution}
  通过初等变换得到$\lambda E - A$的法式为$\left[
    \begin{array}{ccc}
      1&0&0\\
      0&\lambda&0\\
      0&0&-\lambda^2 + 2\lambda
    \end{array}
  \right]$,
  因此初等因子为$\lambda,\lambda,(\lambda - 2)$,
  对应Jordan阵为:
  \begin{equation*}
    J = \left[
      \begin{array}{ccc}
        0&& \\
         &0& \\
         &&2
      \end{array}
    \right]
  \end{equation*}
\end{solution}

~

二、行列式因子方法:求一般矩阵的Jordan阵用行列式因子可能比较麻烦,但是对于抽象的$n$阶矩阵却有很大的作用,
原因在于只要有两个子式互素,则行列式因子就为$1$。

~

\begin{exercise}[经典而重要的题目]
  $a,b \in \mathbb{C}$,求下面矩阵$A$的Jordan标准型
  \begin{equation*}
    A = \left[
      \begin{array}{ccccc}
        a&b&\cdots&b&b \\
         &a&\cdots&b&b \\
         &&\ddots&\vdots&\vdots \\
         &&&a&b \\
         &&&&a
      \end{array}
    \right]
  \end{equation*}
\end{exercise}

\begin{solution}
  若$b = 0$,则Jordan标准型即$aE$。
  若$b \neq 0$,则考虑$\lambda E - A$:
  \begin{equation*}
    \lambda E - A = 
    \left[
      \begin{array}{ccccc}
        \lambda - a&-b&\cdots&-b&-b \\
         &\lambda - a&\cdots&-b&-b \\
         &&\ddots&\vdots&\vdots \\
         &&&\lambda - a&-b \\
         &&&&\lambda - a
      \end{array}
    \right]
  \end{equation*}
  考虑$n-1$阶子式,右下子式行列式为$(\lambda - a)^{n-1}$,
  右上子式在$\lambda = a$时显然不是$0$(根据余数定理),
  因此互素,即初等因子为$(\lambda - a)^n$,因此Jordan阵为一个$n$阶的以$a$为对角的Jordan块
\end{solution}

~

\begin{exercise}[行列式因子应用相关练习]
  求不变因子:
  (1)$A = \left[
    \begin{array}{cccc}
      &&1&\lambda+2 \\
      &1&\lambda+2& \\
      1&\lambda+2&& \\
      \lambda+2&&&
    \end{array}
  \right]$
  (2)$A = \left[
    \begin{array}{cccc}
      \lambda&-1&& \\
             &\lambda&-1& \\
             &&\lambda&-1 \\
             5&4&3&\lambda+2
    \end{array}
  \right]$
\end{exercise}

\begin{solution}
  (1)左上行列式因子为$-1$,
  因此不变因子为$1,1,1,(\lambda+2)^4$

  (2)右上行列式因子为$-1$,
  因此不变因子$1,1,1,|A|$
\end{solution}

~

\begin{theorem}[重要$\lambda$矩阵]
  下面$\lambda$矩阵$A$的不变因子为$1,1,\cdots,\lambda^n + a_1\lambda^{n-1}+a_2\lambda^{n-2} + \cdots + a_{n-1}\lambda + a_n$
  \begin{equation*}
    A = \left[
      \begin{array}{cccccc}
        \lambda&&&&&a_n \\
               -1&\lambda&&&&a_{n-1} \\
               &-1&\lambda&&&a_{n-2} \\
               &&\ddots&\ddots&&\vdots \\
               &&&-1&\lambda&a_2 \\
               &&&&-1&\lambda+a_1
      \end{array}
    \right]
  \end{equation*}
\end{theorem}

\begin{proof}
  考虑左下方行列式因子,其值为$1$,因此不变因子为$1,1,\cdots,1,|A|$,
  而计算出$|A| = \lambda^n + a_1 \lambda^{n-1} + \cdots + a_n $,因此结论成立
\end{proof}

\begin{corollary}
  下列矩阵$A$在$\mathbb{C}$可对角化当且仅当$\lambda^n + a_1\lambda^{n-1}+a_2\lambda^{n-2} + \cdots + a_{n-1}\lambda + a_n$在$\mathbb{C}$上无重根
  \begin{equation*}
    A = \left[
      \begin{array}{cccccc}
        0&&&&&a_n \\
        -1&0&&&&a_{n-1} \\
               &-1&0&&&a_{n-2} \\
               &&\ddots&\ddots&&\vdots \\
               &&&-1&0&a_2 \\
               &&&&-1&a_1
      \end{array}
    \right]
  \end{equation*}
\end{corollary}

~

\begin{exercise}[一些用行列式因子计算方便的特殊矩阵]
  计算Jordan标准型
  (1)$A = \left[
    \begin{array}{cccc}
      1&2&3&4 \\
       &1&2&3 \\
       &&1&2 \\
       &&&1
    \end{array}
  \right]$
  (2)$A = \left[
    \begin{array}{cccc}
      a&&& \\
       a&a&& \\
       \vdots&\vdots&\ddots& \\
       a&a&\cdots&a
    \end{array}
  \right]$
  (3)$A = \left[
    \begin{array}{cccc}
      a&&& \\
       b_1&a&& \\
       &\ddots&\ddots& \\
       &&b_{n-1}&a
    \end{array}
  \right]$
  (4)$A = \left[
    \begin{array}{ccccc}
      a&a_{12}&a_{13}&\cdots&a_{1n} \\
       &a&a_{23}&\cdots&a_{2n} \\
       &&\ddots&&\vdots \\
       &&&a&a_{n-1,n} \\
       &&&&a
    \end{array}
  \right]$,
  $a_{12}a_{23}\cdots a_{n-1,n} \neq 0$
\end{exercise}

\begin{solution}
  (1)$\lambda E - A$左上行列式为$(\lambda - 1)^3$,
  右上在$\lambda = 1$是显然非零,因此不变因子$1,1,1,|\lambda E - A|$

  (2)左上行列式$(\lambda - a)^{n-1}$,
  左下在$\lambda = a$显然非零,因此不变因子$1,1,\cdots,1,|\lambda E - A|$

  (3)(4)显然
\end{solution}

\begin{note}
  如果对角相等,则用行列式因子计算可能会比较方便
\end{note}

\subsection{$k$阶Jordan块的个数}

\begin{theorem}[$k$阶Jordan块个数]
  $n$阶复矩阵$A$的Jordan标准型为$J$,
  则$J$中对角元$\lambda$的$k$阶Jordan块$J_k(\lambda)$的个数如下。这里约定$r((A - \lambda E_n)^0) = n$
  \begin{equation*}
    r((A - \lambda E_n)^{k-1}) - 2r((A - \lambda E_n)^k) + r((A - \lambda E_n)^{k+1})
  \end{equation*}
\end{theorem}

\begin{proof}
  考虑零对角Jordan块的幂次:
  \begin{equation*}
    r(J_r(0)^k) =
    \begin{cases}
      r - k, & 0 \leq k < r\\
      0, & k \geq r
    \end{cases} \quad
    r(J_r(0)^k) - r(J_r(0)^{k+1}) =
    \begin{cases}
      1, & 0 \leq k < r\\
      0, & k \geq r
    \end{cases}
  \end{equation*}
  而非零对角Jordan块的幂次满足
  $r(J_r(a)^k) - r(J_r(a)^{k+1}) = 0$。
  
  设$A$的Jordan标准型为$J = \text{diag}\{J_{r_1}(\lambda_1),\cdots,J_{r_s}(\lambda_s)\}$,
  $(A - \lambda E)^k, (J - \lambda E)^k$相似。
  根据$J$是分块对角的性质可知$r((A - \lambda E)^k) = r((J - \lambda E)^k) = \sum\limits_{i = 1}^s r(J_{r_i}(\lambda_i - \lambda)^k)$,
  因此得到:
  \begin{equation*}
     r((A - \lambda E)^k) - r((A - \lambda E)^{k+1}) = \sum\limits_{i = 1}^s [r(J_{r_i}(\lambda_i - \lambda)^k) - r(J_{r_i}(\lambda_i - \lambda)^{k+1})]
  \end{equation*}
  根据前面的分析,$r((A - \lambda E)^k) - r((A - \lambda E)^{k+1})$等于$\lambda_i = \lambda$的且阶数大于$k$那些块个数,
  同理$r((A - \lambda E)^{k-1}) - r((A - \lambda E)^k)$等于对角为$\lambda$的且阶数大于$k - 1$的块个数,
  相减即可得到阶数为$k$的块个数。
\end{proof}

~

\begin{exercise}[上述定理的应用]
  计算$A$的Jordan标准型
  \begin{equation*}
    A = \left[
      \begin{array}{ccccc}
        a&0&1&& \\
         &a&0&\ddots& \\
         &&a&\ddots&1 \\
         &&&\ddots&0 \\
         &&&&a
      \end{array}
    \right]
  \end{equation*}
\end{exercise}

~

\begin{exercise}
  $A,B$是两个$n$阶矩阵,证明$A,B$相似的充要条件是$\forall \lambda \in \mathbb{C}$,$k = 1,2,\cdots,n$,
  均有$r((A - \lambda E)^k) = r((B - \lambda E)^k)$
\end{exercise}

\begin{proof}
  直接用结论即可
\end{proof}

\subsection{分块矩阵的Jordan阵}

\begin{theorem}[分块矩阵的Jordan标准型]
  $P$上$n$阶矩阵$A,B$无公共复特征值,
  且$A,B$的Jordan标准型分别为$J_1,J_2$,则对任意$C \in P^{m \times n}$,
  则:
  \begin{equation*}
    \left[
      \begin{array}{cc}
        A&C\\
        O&B
      \end{array}
    \right] \sim \text{diag}\{J_1,J_2\}
  \end{equation*}
\end{theorem}

\begin{proof}
  当$A,B$无公共特征值时,根据$AX - XB$的矩阵方程结论可知,
  对任意矩阵$C$,$AX - XB = C$都有唯一解$X_0$使得$AX_0 - X_0B = C$,因此:
  \begin{equation*}
    \left[
      \begin{array}{cc}
        E_m&X_0\\
        O&E_n
      \end{array}
    \right] \left[
      \begin{array}{cc}
        A&C\\
        O&B
      \end{array}
    \right] \left[
      \begin{array}{cc}
        E_m&-X_0\\
        O&E_n
      \end{array}
    \right] = \left[
      \begin{array}{cc}
        A&C - (AX_0 - X_0B )\\
        O&B
      \end{array}
    \right] = \left[
      \begin{array}{cc}
        A&O\\
        O&B
      \end{array}
    \right]
  \end{equation*}
  因此得到:
  \begin{equation*}
    \left[
      \begin{array}{cc}
        A&C\\
        O&B
      \end{array}
    \right] \sim \left[
      \begin{array}{cc}
        A&O\\
        O&B
      \end{array}
    \right] \sim \left[
      \begin{array}{cc}
        J_1&O\\
        O&J_2
      \end{array}
    \right]
  \end{equation*}
\end{proof}

\begin{proof}
  第二种证明:直接看Jordan标准型,
  $\lambda_i$为对角的$k$阶Jordan块结论直接往上套即可。
\end{proof}

~

\begin{corollary}
  对于分块上三角矩阵$A$,$A_{11},A_{22},\cdots,A_{ss}$两两无公共特征值,
  $A_{ii}$的Jordan标准型为$J_i$,则$A$的Jordan标准型为$\text{diag}\{J_1,\cdots,J_s\}$
  \begin{equation*}
    A = \left[
      \begin{array}{cccc}
        A_{11}&A_{12}&\cdots&A_{1s} \\
              &A_{22}&\cdots&A_{2s} \\
              &&\ddots&\vdots \\
              &&&A_{ss}
      \end{array}
    \right]
  \end{equation*}
\end{corollary}

~

\begin{exercise}[分块矩阵应用]
  求$A$的Jordan标准型:
  \begin{equation*}
    A = \left[
      \begin{array}{cccc}
        1&5&-1&2\\
        0&2&4&3 \\
         0&0&-3&0 \\
         0&0&4&-3
      \end{array}
    \right]
  \end{equation*}
\end{exercise}

\begin{solution}
  $\text{diag}\{1,2,J_2(-3)\}$
\end{solution}

\subsection{过渡矩阵的计算}

\begin{theorem}[过渡矩阵计算总结]
  若已知$A$的Jordan标准型为$J$,要计算$P$使得$P^{-1}AP = J$,则可以按照以下步骤:
  \begin{enumerate}
  \item 对于非一阶,以$\lambda$为对角的Jordan块,设对应的$P$位置为$[X_1,\cdots,X_k]$,
    则满足
    \begin{equation*}
      AX_1 = \lambda X_1, AX_2 = \lambda X_2 + X_1, \cdots, AX_k = \lambda X_k + X_{k-1}
    \end{equation*}
    设右端项为$b = [b_1,\cdots,b_n]^T$(未知量),
    统一对增广矩阵做初等变换$[A - \lambda E,b] \rightarrow [A^{\prime}, b^{\prime}]$,
    选取顺序为$X_1,\cdots,X_k$。
    $X_1$要保证不能选$[0,\cdots,0]^T$,且选的$X_k$要保证$X_{k+1}$有解。
  \item 对于一阶:即$AX = \lambda X$特征向量,先求完非一阶再求,不要选$[0,\cdots,0]^T$,
    保证与其他列向量线性无关。
  \end{enumerate}
\end{theorem}



\begin{exercise}
  已知矩阵$A$及其Jordan标准型$J$,求过渡矩阵$P$使得$P^{-1}AP = J$
  \begin{equation*}
    A = \left[
      \begin{array}{ccc}
        2&3&2 \\
         1&8&2 \\
         -2&-14&-3
      \end{array}
    \right] \quad
    J = \left[
      \begin{array}{ccc}
        1&0&0 \\
         0&3&1 \\
         0&0&3
      \end{array}
    \right]
  \end{equation*}
\end{exercise}

\begin{solution}
  设$P = (X_1,X_2,X_3)$,
  根据$P^{-1}AP = J$得到$(AX_1,AX_2,AX_3) = (X_1,X_2,X_3)J$,
  因此$AX_1 = X_1, AX_2 = 3X_2, AX_3 = X_2 + 3X_3$,
  即$(A - E)X_1 = 0, (A - 3E)X_2 = 0, (A - 3E)X_3 = X_2$。
  先算后面非一阶,
  做初等变换得到:
  \begin{equation*}
    [A - 3E | b] \rightarrow
    \left[
      \begin{array}{cccc}
        -1&3&2&b_1 \\
          0&8&4&b_1 + b_2 \\
          0&8&4&\frac{1}{2}b_1 + \frac{5}{2}b_2 + b_3
      \end{array}
    \right]
  \end{equation*}
  有解必须满足$\frac{1}{2}b_1 + \frac{5}{2}b_2 + b_3 = 0$,
  因此$AX_2 = 0$等价于:
  \begin{equation*}
    \left[
      \begin{array}{ccc}
        -1&3&2 \\
          0&8&4 \\
          \frac{1}{2}&\frac{5}{2}&1
      \end{array}
    \right]X_2 = 0  
  \end{equation*}
  而最后一个条件显然可以被前两行消掉,因此可以无视,
  随便取$X_2 = [1,-1,2]^T$,因此$X_3$满足
  \begin{equation*}
    \left[
      \begin{array}{ccc}
        -1&3&2 \\
          0&8&4 \\
          0&0&0
      \end{array}
    \right]X_3 = \left[
      \begin{array}{c}
        1 \\
        0\\
        0
      \end{array}
    \right]
  \end{equation*}
  $X_3$取法无所谓,因此取$X_3 = [-1,0,0]^T$。
  而$X_1$取线性无关即可,如$X_1 = [-2,0,1]^T$
\end{solution}


~

\begin{exercise}
  已知矩阵$A$及其Jordan标准型$J$,求过渡矩阵$P$使得$P^{-1}AP = J$
  \begin{equation*}
    A = \left[
      \begin{array}{ccc}
        0&1&0 \\
         -4&4&0 \\
         -2&1&2
      \end{array}
    \right],
    J = \left[
      \begin{array}{ccc}
        2&& \\
         &2&1 \\
         &&2
      \end{array}
    \right]
  \end{equation*}
\end{exercise}

\begin{solution}
  设$P = [X_1,X_2,X_3]$,
  则$(AX_1,AX_2,AX_3) = (2X_1,2X_2,X_2 + 2X_3)$,
  得到方程组$(A - 2E)X_1 = 0, (A - 2E)X_2 = 0, (A - 2E)X_3 = X_2$。
  做初等变换后:
  \begin{equation*}
    [A|b] \rightarrow
    \left[
      \begin{array}{cccc}
        -2&1&0&b_1 \\
          0&0&0&b_2 - 2b_1 \\
          0&0&0&b_3 - b_1
      \end{array}
    \right]
  \end{equation*}
  因此要使$AX_3 = X_2$有解,
  $X_2$必须满足$b_3 = b_1$,
  先选取$X_2 = [1,2,1]^T$,
  解出$X_3 = [-\frac{1}{2},0,0]^T$,
  $X_1$要线性无关,取$[0,0,1]^T$即可。
\end{solution}





