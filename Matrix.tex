


\chapter{矩阵}

\section{矩阵基础}

\subsection{矩阵概念与运算}

略

\subsection{矩阵的逆}


\begin{definition}[矩阵的逆]
  若$AB = BA = E$,则称$B$为$A$的逆,记$A^{-1}$
\end{definition}

\begin{theorem}[矩阵逆的基本性质]
  (1)$(kA)^{-1} = k^{-1}A^{-1}$
  (2)$(AB)^{-1} = B^{-1}A^{-1}$
  (3)$(A^T)^{-1} = (A^{-1})^T$
  (4)$|A^{-1}| = |A|^{-1}$
\end{theorem}

\begin{proof}
  (2)$(AB)^{-1}AB = E$,
  因此$(AB)^{-1} = B^{-1}A^{-1}$

  (3)$(A^{-1})^TA^T = (AA^{-1})^T = E$

  (4)$|A^{-1}A| = |A^{-1}| \cdot |A| = 1$,
  因此$|A^{-1}| = |A|^{-1}$
\end{proof}

\begin{theorem}[可逆充要条件]
  矩阵$A$可以的充要条件如下:
  \begin{itemize}
  \item $|A| \neq 0$
  \item $A$可表示为初等矩阵的积
  \end{itemize}
\end{theorem}


\begin{theorem}[利用初等行变换求逆]
  若$A$可逆,则利用初等行变换可以得到$(A,E) \rightarrow (E,A^{-1})$。
  特别地,$(A,B) \rightarrow (E, A^{-1}B)$
\end{theorem}

\begin{note}
  同样可以通过初等列变换得到:$\left(
    \begin{array}{c}
      A\\
      B
    \end{array}
  \right) \rightarrow \left(
    \begin{array}{c}
      E\\
      BA^{-1}
    \end{array}
  \right)$
\end{note}

~

\begin{exercise}[利用初等变换求逆]
  (1)计算下面矩阵的逆:
  \begin{equation*}
    \left[
      \begin{array}{ccc}
        1&1&-1 \\
         2&1&0 \\
         1&-1&0
      \end{array}
    \right]
  \end{equation*}
\end{exercise}

\begin{solution}
  结果为$\frac{1}{3} \left[
    \begin{array}{ccc}
      0&1&1 \\
       0&1&-2 \\
       -3&2&-1
    \end{array}
  \right]$
\end{solution}

~

\begin{exercise}[逆矩阵理论练习]
  (1)$A$为方阵,若$A^k = O$,证明$E - A$可逆,且$(E - A)^{-1} = E + A + A^2 + \cdots + A^{k-1}$

  (2)$A,B$为$n$阶方阵,若$A + B = AB$,证明:$A - E$可逆,并求逆矩阵

  (3)若$A + B$可逆,证明$A^{-1} + B^{-1}$也可逆,并求其逆
\end{exercise}

\begin{solution}
  (1)直接做乘法验证:
  \begin{equation*}
    (E - A)(E + A + A^2 + \cdots + A^{k-1}) = E + A + A^2 + \cdots + A^{k-1} - A - A^2 - \cdots - A^k = E - A^k = E
  \end{equation*}

  (2)由于$A + B = AB$,则$(A-E)(B - E) = AB - (A+B) + E = E$,
  因此$(A-E)^{-1} = B - E$

  (3)根据$B(A^{-1}+B^{-1})A = (A + B)$得到
  \begin{equation*}
    (A^{-1} + B^{-1})B = A^{-1}(A + B) \Rightarrow (A^{-1} + B^{-1})B(A + B)^{-1} = A^{-1}
  \end{equation*}
  因此$(A^{-1} + B^{-1})^{-1} = B(A + B)^{-1}A$
\end{solution}


~

\begin{definition}[(代数)余子式]
  $A = (a_{ij})$为$n \times n$矩阵,
  划去$i$行$j$列的行列式称为余子式$M_{ij}$,
  代数余子式为$(-1)^{i+j}M_{ij}$
\end{definition}

\begin{definition}[伴随矩阵]
  $A = (a_{ij})_n$,$A_{ij}$是$a_{ij}$的代数余子式,
  则其伴随为(注意行列颠倒!!):
  \begin{equation*}
    A^{*} = \left[
      \begin{array}{cccc}
        A_{11}&A_{21}& \cdots& A_{n1} \\
        A_{12}&A_{22}&\cdots&A_{n2} \\
        \vdots&\vdots&\ddots&\vdots \\
        A_{1n}&A_{2n}&\cdots&A_{nn}
      \end{array}
    \right]
  \end{equation*}
\end{definition}

\begin{theorem}[利用伴随求逆]
  $A$及其伴随矩阵$A^{*}$满足$AA^{*} = |A|E$,
  若$A$可逆,则$A^{-1} = \frac{1}{|A|}A^{*}$
\end{theorem}

\begin{proof}
  考虑$B = AA^{\ast}$,
  则$B_{ij} = \sum\limits_{k = 1}^n a_{ik}A_{kj}$,
  根据行列式的展开式可知
  \begin{equation*}
    B_{ij} =
    \begin{cases}
      |A|, &i = j\\
      0, & i \neq j
    \end{cases}
  \end{equation*}
  因此$AA^{\ast} = |A|E$
\end{proof}

\subsection{伴随矩阵的性质}

\begin{theorem}[伴随运算的性质]
  $A$为$n$阶方阵,则
  \begin{enumerate}
  \item $(A^{\ast})^T = (A^T)^{\ast}$
  \item $(A^{\ast})^{-1} = (A^{-1})^{\ast}$
  \item $(AB)^{\ast} = B^{\ast}A^{\ast}$
  \end{enumerate}
\end{theorem}

\begin{proof}
  (1)设$A = (a_{ij})$,则
  \begin{equation*}
    (A^{\ast})^T = \left[
      \begin{array}{cccc}
        A_{11}&A_{12}&\cdots&A_{1n} \\
              A_{21}&A_{22}&\cdots&A_{2n} \\
              \vdots&\vdots&&\vdots \\
              A_{n1}&A_{n2}&\cdots&A_{nn}
      \end{array}
    \right]
  \end{equation*}
  同理设$(A^T)^{\ast} = (B_{ij})$,则$B_{ij} = A^T_{ji}$,
  因此行列式大小相同。

  (2)$A^{\ast} = |A|A^{-1}$,
  则$(A^{\ast})^{-1} = \frac{1}{|A|}A$
  因此
  \begin{equation*}
    (A^{-1})^{\ast} = |A^{-1}|A = \frac{1}{|A|}A
  \end{equation*}
  因此结论成立

  (3)$(AB)^{\ast} = |AB|(AB)^{-1} = |B|(B)^{-1}|A|A^{-1} = B^{\ast}A^{\ast}$
\end{proof}

\begin{theorem}[伴随矩阵的行列式]
  $|A^{\ast}| = |A|^{n-1}$
\end{theorem}

\begin{proof}
  $|A^{\ast}| = \left| |A|A^{-1} \right| = |A|^n \cdot |A^{-1}| = |A|^{n-1}$
\end{proof}

\begin{theorem}[伴随矩阵的秩]
  $A^{\ast}$为$A$的伴随矩阵,则
  \begin{equation*}
    r(A^{\ast}) =
    \begin{cases}
      n, & r(A) = n\\
      1, & r(A) = n-1\\
      0, & r(A) < n-1
    \end{cases}
  \end{equation*}
\end{theorem}

\begin{proof}
  $r(A) = n$时,$|A^{\ast}| = |A|^{n-1} \neq 0$,因此满秩。
  $r(A) = n-1$时,由于$AA^{\ast} = |A|E = O$,
  根据秩不等式:
  \begin{equation*}
    r(AA^{\ast}) \geq r(A) + r(A^{\ast}) - n = r(A^{\ast}) - 1 \Rightarrow r(A^{\ast}) \leq 1
  \end{equation*}
  而由于$r(A) = n-1$,其至少有一个$n-1$阶代数余子式不为$0$,因此$r(A^{\ast}) \geq 1$,
  综上$r(A^{\ast}) = 1$。
  其余显然
\end{proof}

\begin{theorem}[伴随矩阵的特征值]
  矩阵$A$的特征值为$\lambda_1,\cdots,\lambda_n$,
  则$A^{\ast}$的特征值$\lambda^{\ast}_j= \prod \limits_{i \neq j}\lambda_i$
\end{theorem}

\begin{proof}
  考虑$A = P^{-1}JP$,
  由于$J$是上三角阵,
  根据初等变换法,$J^{-1}$是上三角阵,
  $J^{\ast} = |J|J^{-1}$也是上三角阵,
  而$J$对角元素的代数余子式为$\prod \limits_{i \neq j}\lambda_i$,
  因此$J^{\ast}$的对角元为$\prod \limits_{i \neq j}\lambda_i$,
  代数余子式在初等变换下不变(位置可能变化,但值整体相同),因此结论成立。
\end{proof}


\subsection{广义逆矩阵}

\begin{definition}[广义逆]
  $A_{m \times n}$是数域$\mathbb{P}$上的矩阵,若存在$G_{n \times m}$使得
  \begin{equation*}
    AGA = A
  \end{equation*}
  则称$G$是$A$的广义逆矩阵。
\end{definition}

\begin{note}
  广义逆矩阵一般不唯一。
\end{note}

\begin{theorem}[广义逆的表达式]
  设$A_{m \times n}$的秩为$r$,设$A = P \left(
    \begin{array}{cc}
      E_r&O\\
      O&O
    \end{array}
  \right)Q$,
  其中$P_{m \times m}, Q _{n \times n}$为可逆矩阵,则$A$全部的广义逆矩阵为
  \begin{equation*}
    G = Q^{-1} \left(
      \begin{array}{cc}
        E_r&C\\
        D&F
      \end{array}
    \right)P^{-1}
  \end{equation*}
  这里$C,D,F$为任意$r \times (m-r), (n-r)\times r, (m-r)\times (n-r)$阶矩阵。
\end{theorem}

\begin{proof}
  (1)验证$G$是广义逆:直接乘起来就行,得到$AGA = A$

  (2)验证广义逆都是$G$:设$G = Q^{-1} \left(
    \begin{array}{cc}
      B&C\\
      D&F
    \end{array}
  \right)P^{-1}$,
  乘法得到
  \begin{equation*}
    AGA = P \left(
      \begin{array}{cc}
        B&O\\
        O&O
      \end{array}
    \right)Q
  \end{equation*}
  因此得到$B = E_r$
\end{proof}

\begin{theorem}[非方阵线性方程组解的表达式]
  $A_{m \times n}$,$G$是$A$的广义逆,则$AX = b$的通解可表示为
  \begin{equation*}
    X = Gb + (E_n - GA)Y
  \end{equation*}
  这里$Y$是任一$n$维向量
\end{theorem}

\begin{proof}
  (1)证明是解:
  设$Z$是$AZ = b$的解,则此时有
  $b = AZ = AGAZ = AGb$。
  将$X$表达式代入$AX$验证:
  \begin{equation*}
    AX = AGb + (A - AGA)Y = b + O = b
  \end{equation*}

  (2)证明解都有该形式:设$X_0$是$AX = b$的解,则
  \begin{equation*}
    X_0 = Gb + X_0 - Gb = Gb + X_0 - GAX_0 = Gb + (E_n - GA)X_0 
  \end{equation*}
\end{proof}

\subsection{Moore-Penrose广义逆}

\begin{definition}[共轭转置]
  $A$是$\mathbb{C}$上$n \times m$矩阵,
  则定义$A^H = (\overline{A})^T$
\end{definition}

\begin{definition}[Moore-Penrose广义逆]
  $A$是$\mathbb{C}$上$m \times n$阶矩阵,若存在$\mathbb{C}$上$n \times m$阶矩阵$G$满足:
  \begin{enumerate}
  \item $AGA = A$
  \item $GAG = G$
  \item $(AG)^H = AG$
  \item $(GA)^H = GA$
  \end{enumerate}
  则称$G$为$A$的M-P广义逆
\end{definition}

\begin{theorem}[M-P广义逆的存在唯一性]
  $A_{m \infty n}$的M-P广义逆总是存在且唯一的,
  且若$\beta_1,\cdots,\beta_r$是$A$列向量的极大线性无关组,
  记$B = [\beta_1,\cdots,\beta_r]$,
  分解$A = BC$,则M-P广义逆的具体表达式为:
  \begin{equation*}
    A^+ = C^H(CC^H)^{-1}(B^HB)^{-1}B^H
  \end{equation*}
\end{theorem}

\begin{proof}
  存在性:设$A$的秩为$r$,取$\beta_1,\cdots,\beta_r$为$A$列向量的极大线性无关组,
  此时
  \begin{equation*}
    A = (\alpha_1,\cdots,\alpha_n) = (\beta_1,\cdots,\beta_r) \left(
      \begin{array}{ccc}
        c_{11}&\cdots&c_{1n} \\
              \vdots&&\vdots \\
              c_{r1}&\cdots&c_{rn}
      \end{array}
    \right) = BC
  \end{equation*}
  此时$r(BC) = r \leq r(C) \leq \min\{r,n\} \leq r$,
  得到$r(C) = r$行满秩,$B$显然列满秩,
  因此$B^HB, CC^H$是$r$阶可逆矩阵(根据$r(A^HA), r(AA^H)$的性质)。令
  \begin{equation*}
    G = C^H(CC^H)^{-1}(B^HB)^{-1}B^H
  \end{equation*}
  可验证$G$满足上述所有条件

  唯一性:设$X$是另一个广义逆,则进行下面步骤,本质是用$AXA = A$消去$X$
  \begin{align*}
\boldsymbol{X} &=\boldsymbol{X} \boldsymbol{A} \boldsymbol{X}=\boldsymbol{X}(\boldsymbol{A} \boldsymbol{X})^{\mathrm{H}}=\boldsymbol{X} \boldsymbol{X}^{\mathrm{H}} \boldsymbol{A}^{\mathrm{H}}=\boldsymbol{X} \boldsymbol{X}^{\mathrm{H}}(\boldsymbol{A} \boldsymbol{G} \boldsymbol{A})^{\mathrm{H}} \\
&=\boldsymbol{X} \boldsymbol{X}^{\mathrm{H}} \boldsymbol{A}^{\mathrm{H}}(\boldsymbol{A} \boldsymbol{G})^{\mathrm{H}}=\boldsymbol{X}(\boldsymbol{A} \boldsymbol{X})^{\mathrm{H}}(\boldsymbol{A} \boldsymbol{G})^{\mathrm{H}}=\boldsymbol{X} \boldsymbol{A} \boldsymbol{X} \boldsymbol{A} \boldsymbol{G} \\
&=\boldsymbol{X} \boldsymbol{A} \boldsymbol{G}=\boldsymbol{X}(\boldsymbol{A} \boldsymbol{G} \boldsymbol{A}) \boldsymbol{G}=(\boldsymbol{X} \boldsymbol{A})^{\mathrm{H}}(\boldsymbol{G} \boldsymbol{A})^{\mathrm{H}} \boldsymbol{G} \\
&=(\boldsymbol{G} \boldsymbol{A} \boldsymbol{X} \boldsymbol{A})^{\mathrm{H}} \boldsymbol{G}=(\boldsymbol{G} \boldsymbol{A})^{\mathrm{H}} \boldsymbol{G}=(\boldsymbol{G} \boldsymbol{A}) \boldsymbol{G} \\
&=\boldsymbol{G} .
\end{align*}
\end{proof}

~

\begin{exercise}[计算M-P逆]
  (1)$A = \left(
    \begin{array}{ccc}
      1&0&-1 \\
       1&2&0 \\
       0&2&1
    \end{array}
  \right)$的M-P逆
\end{exercise}

\begin{solution}
  (1)先进行分解,得到
  \begin{equation*}
    B = \left(
      \begin{array}{cc}
        1&0\\
        1&1\\
        0&1
      \end{array}
    \right), C = \left(
      \begin{array}{ccc}
        1&0&-1 \\
         0&2&1
      \end{array}
    \right)
  \end{equation*}
  最终结果得到
  \begin{equation*}
    A^+ = \frac{1}{9} \left(
      \begin{array}{ccc}
        3&2&-1 \\
         0&2&2 \\
         -3&-1&2
      \end{array}
    \right)
  \end{equation*}
\end{solution}

\begin{theorem}[最小二乘解的表达式]
  $AX = b$最小二乘解为$X = A^+b + (E_n - A^+A)Y$,
  其中$A^+b$是唯一的极小最小二乘解。
\end{theorem}

\section{分块矩阵}

\subsection{常见矩阵拆分技巧}

\begin{theorem}[点态分解]
  $A$是$n \times n$矩阵,则
  \begin{equation*}
    A = \left[
      \begin{array}{cccc}
        a_1b_1&a_1b_2&\cdots&a_1b_n \\
              a_2b_1&a_2b_2&\cdots&a_2b_n \\
              \vdots&\vdots&&\vdots \\
              a_nb_1&a_nb_2&\cdots&a_nb_n
      \end{array}
    \right] \Rightarrow A = \left(
      \begin{array}{c}
        a_1\\
        a_2\\
        \vdots \\
        a_n
      \end{array}
    \right) \left(
      \begin{array}{cccc}
        b_1&b_2&\cdots&b_n
      \end{array}
    \right)
  \end{equation*}
\end{theorem}



\begin{theorem}[$A^TA$的和式表达]
  $A$为$n \times m$矩阵,将$A$做下面的行划分,则$A^TA = \sum\limits_{i = 1}^n \alpha_i^T \alpha_i$
  \begin{equation*}
    A = \left(
      \begin{array}{c}
        \alpha_1\\
        \alpha_2\\
        \vdots\\
        \alpha_s
      \end{array}
    \right)
  \end{equation*}
\end{theorem}


\subsection{分块初等变换}

\begin{definition}[分块矩阵初等变换]
  分块矩阵初等变换分为行变换和列变换:
  \begin{itemize}
  \item 行变换:(1)左乘矩阵$P$,加到另一行(2)互换两行位置(3)将可逆矩阵左乘至一行
  \item 列变换:(1)右乘矩阵$P$,加到另一列(2)互换两列位置(3)将可逆矩阵右乘至一列
  \end{itemize}
\end{definition}

\begin{definition}[分块初等矩阵]
  对应三种分块初等变换有三种分块初等矩阵,常用的即加至另一行/列的初等分块矩阵:
  \begin{equation*}
    A = \left[
      \begin{array}{cc}
        E&0\\
        P&E
      \end{array}
    \right]
  \end{equation*}
\end{definition}

\begin{theorem}[初等变换的分块初等矩阵表达]
  分块初等矩阵左乘一个矩阵即做行变换,右乘即做列变换:
  \begin{equation*}
    \left[
      \begin{array}{cc}
        E&0\\
        P&E
      \end{array}
    \right] \left[
      \begin{array}{cc}
        A_1&A_2\\
        A_3&A_4
      \end{array}
    \right] = \left[
      \begin{array}{cc}
        A_1&A_2\\
        PA_1 + A_3& PA_2 + A_4
      \end{array}
    \right]
  \end{equation*}
\end{theorem}


\subsection{分块打洞技巧总结}

\begin{theorem}[分块打洞]
  对于分块矩阵$\left[
    \begin{array}{cc}
      A&B\\
      C&D
    \end{array}
  \right]$可通过分块初等变换转换为:
  \begin{equation*}
    \left[
      \begin{array}{cc}
        A&B\\
        O&D - CA^{-1}B
      \end{array} 
    \right] \quad \left[
      \begin{array}{cc}
        A - BD^{-1}C&O\\
        O&D
      \end{array}
    \right]
  \end{equation*}
  记忆方式:除去被减的位置,减去的内容顺序都是顺时针转的。
\end{theorem}

\begin{corollary}[两种特殊形式]
  对称形式和归纳形式:
  \begin{itemize}
  \item 对称形式:$\left[
      \begin{array}{cc}
        A&B\\
        B^T&D
      \end{array}
    \right]
    \rightarrow
    \left[
      \begin{array}{cc}
        A&O\\
        O&D - B^TA^{-1}B
      \end{array}
    \right]
    $
  \item 归纳形式:$\left[
      \begin{array}{cc}
        A&\alpha\\
        \beta^T&a_{nn}
      \end{array}
    \right]
    \rightarrow 
    \left[
      \begin{array}{cc}
        A&\alpha\\
        0&a_{nn} - \beta^T A^{-1}\alpha
      \end{array}
    \right]
    $
  \end{itemize}
\end{corollary}


~

\begin{exercise}[分块矩阵与数学归纳法]
  (1)$n$阶方阵$A$的顺序主子式非零,证明:
  存在$n$阶下三角矩阵$B$使得$BA$是上三角阵

  (2)$A$是$n$阶实方阵,$A$所有顺序主子式大于$0$,
  且非主对角元素为负,证明:$A^{-1}$的元素均为正数
\end{exercise}

\begin{proof}
  (1)$n=1$显然成立。设$n-1$时成立,考虑如下分块,
  由于$A$顺序主子式非零,因此$A_1$顺序主子式非零,存在$n-1$阶下三角矩阵$B_1$满足$B_1A_1$为上三角阵,
  因此做对应分块初等变换
  \begin{equation*}
    A = \left[
      \begin{array}{cc}
        A_1&\alpha \\
        \beta^T&a_{nn}
      \end{array}
    \right] \Rightarrow \left[
      \begin{array}{cc}
        B_1& \\
           &1
      \end{array}
    \right] \cdot \left[
      \begin{array}{cc}
        E_{n-1}&O\\
        -\beta^T A_1^{-1}&1
      \end{array}
    \right] \cdot \left[
      \begin{array}{cc}
        A_1&\alpha\\
        \beta^T&a_{nn}
      \end{array}
    \right] = \left[
      \begin{array}{cc}
        B_1A_1&B_1\alpha\\
        O&a_{nn} - \beta^T A^{-1}_1\alpha
      \end{array}
    \right]
  \end{equation*}
  右侧即一个上三角矩阵,把两个初等矩阵的乘积视作$B$即可。

  (2)$n = 1$显然。设$n-1$时成立,对$A$做下述分块,
  则$A_1^{-1}$元素都是正的
  \begin{equation*}
    A = \left[
      \begin{array}{cc}
        A_1&\alpha \\
        \beta^T&a_{nn}
      \end{array}
    \right]  \Rightarrow \left[
      \begin{array}{cc}
        E_{n-1}&O\\
        -\beta^T A_1^{-1}& 1
      \end{array}
    \right] \left[
      \begin{array}{cc}
        A_1&\alpha\\
        \beta^T&a_{nn}
      \end{array}
    \right] \left[
      \begin{array}{cc}
        E_{n-1}&-A_1^{-1}\alpha\\
        O&1
      \end{array}
    \right] = \left[
      \begin{array}{cc}
        A_1&O\\
        O&a_{nn} - \beta^T A^{-1}_1\alpha
      \end{array}
    \right]
  \end{equation*}
  根据$|A| > 0, |A_1|> 0$得到$a_{nn} - \beta^T A_1^{-1}\alpha > 0$。
  记上述等式为$BAC = D$,则$A^{-1} = CD^{-1}B$,
  根据矩阵乘法乘开即可获得结论。
\end{proof}


\subsection{分块矩阵行列式}

\begin{theorem}[分块对角矩阵的行列式]
  根据Laplace多行展开定理可以得到:
  \begin{equation*}
    \left|
      \begin{array}{cc}
        A&0\\
        B&C
      \end{array}
    \right| = |A||B|, \quad
    \left|
      \begin{array}{cccc}
        A_{11}&0&\cdots&0 \\
              A_{21}&A_{22}&\cdots&0 \\
              \vdots&\vdots&\ddots&\vdots \\
              A_{n1}&A_{n2}&\cdots&A_{nn}
      \end{array}
    \right| = |A_{11}| |A_{22}| \cdots |A_{nn}|
  \end{equation*}
\end{theorem}

\begin{theorem}[一般分块矩阵的行列式]
  设$A,D$分别为$r,s$阶矩阵,$A,D$可逆,则
  \begin{equation*}
    \left|
      \begin{array}{cc}
        A&B \\
         C&D
      \end{array}
    \right| = |D| |A - BD^{-1}C| = |A| |D - CA^{-1}B|
  \end{equation*}
\end{theorem}

\begin{proof}
  (1)如果$D$可逆,则把$B$消掉(左乘$BD^{-1}$),取行列式即可
  \begin{equation*}
    \left[
      \begin{array}{cc}
        A&B\\
        C&D
      \end{array}
    \right] \Rightarrow \left[
      \begin{array}{cc}
        A - BD^{-1}C&0\\
        C&D
      \end{array}
    \right]
  \end{equation*}
\end{proof}

\begin{theorem}[行列式计算技巧1:配凑分块矩阵]
  矩阵$A$可分解为$A = J - u^Tu$,
  要计算$|A|$,则可以构造分块矩阵:
  \begin{equation*}
    \left|
      \begin{array}{cc}
        J&u\\
        u^T&1
      \end{array}
    \right| = |1|\cdot |J - u^Tu| = |J|\cdot |1 - uJ^{-1}u^T| \Rightarrow |A| = |J| \cdot |1 - uJ^{-1}u^T|
  \end{equation*}
  注意这里$u$是列向量
\end{theorem}






\subsection{分块矩阵求逆}

一般分块矩阵求逆都先转换为分块对角矩阵,具体如下。

\begin{theorem}[上三角分块矩阵求逆]
  当下面的$A_1,A_2$都是方阵,则$A$可逆当且仅当$A_1,A_2$都可逆,
  且逆矩阵如下:
  \begin{equation*}
    A = \left[
      \begin{array}{cc}
        A_1&A_3\\
        O&A_2
      \end{array}
    \right] \Rightarrow A^{-1} = \left[
      \begin{array}{cc}
        A_1^{-1}&-A_1^{-1}A_3A_2^{-1} \\
        0&A_2^{-1}
      \end{array}
    \right]
  \end{equation*}
\end{theorem}

\begin{proof}
  根据分块矩阵初等变换可知
  \begin{equation*}
    \left[
      \begin{array}{cc}
        E&-A_2^{-1}A_3\\
        O&E
      \end{array}
    \right]
    \left[
      \begin{array}{cc}
        A_1&A_3\\
        O&A_2
      \end{array}
    \right] = \left[
      \begin{array}{cc}
        A_1&O\\
        O&A_2
      \end{array}
    \right] \Rightarrow 
    \left[
      \begin{array}{cc}
        E&-A_2^{-1}A_3\\
        O&E
      \end{array}
    \right]^{-1}
    \left[
      \begin{array}{cc}
        A_1&A_3\\
        O&A_2
      \end{array}
    \right]^{-1} = \left[
      \begin{array}{cc}
        A_1^{-1}&O\\
        O&A_2^{-1}
      \end{array}
    \right]
  \end{equation*}
  因此可以得到结论。
\end{proof}

\begin{theorem}[一般分块矩阵可逆条件]
  分块矩阵$H = \left(
    \begin{array}{cc}
      A&B\\
      C&D
    \end{array}
  \right)$可逆当且仅当$A - BD^{-1}C$和$D - CA^{-1}B$都可逆
\end{theorem}

~

\begin{exercise}[分块矩阵求逆]
  $A,B$时$n,m$阶可逆方阵,
  求$\left[
    \begin{array}{cc}
      O&A\\
      B&O
    \end{array}
  \right]$的逆矩阵
\end{exercise}

\begin{solution}
  先做交换行的变换:$\left[
    \begin{array}{cc}
      O&A\\
      B&O
    \end{array}
  \right]\left[
    \begin{array}{cc}
      O&E\\
      E&O
    \end{array}
  \right] = \left[
    \begin{array}{cc}
      A&\\
      &B
    \end{array}
  \right]$,
  两侧求逆即可得到:
  \begin{equation*}
    \left[
      \begin{array}{cc}
        O&A\\
        B&O
      \end{array}
    \right]^{-1} =  \left[
      \begin{array}{cc}
        A^{-1}&O\\
        O&B^{-1}
      \end{array}
    \right]
    \left[
      \begin{array}{cc}
        O&E\\
        E&O
      \end{array}
    \right]
    = \left[
      \begin{array}{cc}
        O&B^{-1}\\
        A^{-1}&O
      \end{array}
    \right]
  \end{equation*}
\end{solution}

\section{矩阵的秩}

\subsection{秩的概念}

\begin{definition}[矩阵的秩]
  矩阵的秩等于:
  \begin{itemize}
  \item 矩阵行列向量的秩
  \item 最大非零子式的阶数
  \end{itemize}
\end{definition}



\subsection{初等变换标准型与矩阵分解}

\begin{theorem}[初等变换标准型]
  设$A$是秩为$r$的$s \times n$矩阵,
  则存在$s$阶可逆矩阵$P$和$n$阶可逆矩阵$Q$使得
  \begin{equation*}
    A = P \left[
      \begin{array}{cc}
        E_r&O\\
        O&O
      \end{array}
    \right]Q
  \end{equation*}
\end{theorem}

\begin{note}
  如果一个题目中没给出条件,或者只给出了矩阵的秩,则往往需要用到初等变换标准型
\end{note}

\begin{theorem}[行列满秩的标准型]
  $A$是$m \times n$矩阵,$A$列满秩当且仅当存在$m$阶可逆矩阵$P$使得$A = P \left[
    \begin{array}{c}
      E_n\\
      O
    \end{array}
  \right]$。
  $A$行满秩当且仅当存在$n$阶可逆矩阵$Q$,使得$A = \left[
    \begin{array}{cc}
      E_m&O
    \end{array}
  \right]Q$
\end{theorem}

\begin{proof}
  由于列满秩,可以用行初等变换消成阶梯型,再从下到上消回去即可(若列不满秩,则阶梯型可能消不干净,因此需要列变换)
\end{proof}

\subsection{利用初等变换标准型进行矩阵分解}

\begin{theorem}[矩阵分解]
  $s \times n$矩阵$\left[
    \begin{array}{cc}
      E_r & O\\
      O & O
    \end{array}
  \right]$可分解为
  \begin{equation*}
    \left[
      \begin{array}{cc}
        E_r&O\\
        O&O
      \end{array}
    \right] = \left[
      \begin{array}{c}
        E_r\\
        O
      \end{array}
    \right] \left[
      \begin{array}{cc}
        E_r&O
      \end{array}
    \right]
  \end{equation*}
  若为方阵时,可以分解为:
  \begin{equation*}
    \left[
      \begin{array}{cc}
        E_r&O\\
        O&O
      \end{array}
    \right]
    = 
    \left[
      \begin{array}{cc}

        E_r&O\\
        O&O
      \end{array}
    \right]
    \left[
      \begin{array}{cc}
        E_r&O\\
        O&O
      \end{array}
    \right]
  \end{equation*}
\end{theorem}


\begin{exercise}[行列满秩分解]
  (1)证明:任意非零矩阵都可以分解为列满秩和行满秩矩阵的乘积

  (2)设$B_1,B_2$为数域$P$上$s \times n$列满秩矩阵,
  证明:存在$P$上$s$阶可逆矩阵$C$,使得$B_2 = CB_1$(即$B_1,B_2$只需要行变换就可以相互转换)

  (3)$C_1,C_2$为$P$上$s \times n$行满秩矩阵,
  证明:$C_1,C_2$可以仅通过列变换相互转换。
\end{exercise}

\begin{proof}
  (1)使用等价标准型:
  \begin{equation*}
    A = P \left[\begin{array}{cc}
                  E_r&O\\
                  O&O
                \end{array}
              \right]Q = P \left[
                \begin{array}{c}
                  E_r\\
                  O
                \end{array}
              \right] \left[
                \begin{array}{cc}
                  E_r&O
                \end{array}
              \right]Q
            \end{equation*}

            (2)根据列满秩初等标准型,$\exists P$使得$PB_1 =  \left[
              \begin{array}{c}
                E_n\\
                O
              \end{array}
            \right]$,
            同理存在$Q$使得$Q B_2 = \left[
              \begin{array}{c}
                E_n\\
                O
              \end{array}
            \right]$,结论成立。

            (3)和(2)一样
\end{proof}

~

\begin{exercise}[幂等矩阵分解]
  (1)证明:任意方阵可以分解为可逆矩阵与幂等矩阵乘积
\end{exercise}

\begin{proof}
  (1)利用等价标准型得到:
  \begin{equation*}
    A = P \left(
      \begin{array}{cc}
        E_r&O\\
        O&O
      \end{array}
    \right)Q = PQ Q^{-1} \left(
      \begin{array}{cc}
        E_r&O\\
        O&O
      \end{array}
    \right) Q
  \end{equation*}
  $PQ$为可逆矩阵,后面为幂等
\end{proof}


\subsection{秩不等式}

\begin{lemma}[乘可逆矩阵不改变秩]
  $P,Q$为可逆矩阵,则$r(A) = r(PAQ)$
\end{lemma}

\begin{proof}
  相当于做了一系列初等变换,因此不改变秩
\end{proof}

~

\begin{exercise}[非方阵情况]
  $A,B$分别为$m \times n, n \times s$矩阵,证明:

  (1)若$r(A) = n$,则$r(AB) = r(B)$

  (2)若$r(B) = n$,则$r(AB) = r(A)$
\end{exercise}

\begin{proof}
  (1)存在可逆矩阵$P$使得$PA = \left[
    \begin{array}{c}
      A_1\\
      O
    \end{array}
  \right]$,这里$A_1$为可逆矩阵,因此
  \begin{equation*}
    r(AB) = r(\left[
      \begin{array}{c}
        A_1\\
        O
      \end{array}
    \right]B) = r \left(
      \begin{array}{c}
        A_1B\\
        O
      \end{array}
    \right) = r(A_1B) = r(B)
  \end{equation*}

  (2)同理(转置即可)
\end{proof}

~

\begin{lemma}[分块矩阵的秩]
  \begin{itemize}
  \item $r \left(
      \begin{array}{cc}
        A&O\\
        O&B
      \end{array}
    \right) = r(A) + r(B)$
  \item $r \left(
      \begin{array}{cc}
        A&O\\
        C&B
      \end{array}
    \right) \geq r(A) + r(B)$
  \end{itemize}
\end{lemma}

\begin{proof}
  (1)根据初等变换等价标准型有$P_1AQ_1 = \left(
    \begin{array}{cc}
      E_{r_1}&O\\
      O&O
    \end{array}
  \right), P_2BQ_2 = \left(
    \begin{array}{cc}
      E_{r_2}&O\\
      O&O
    \end{array}
  \right)$,因此
  \begin{equation*}
    \left(
      \begin{array}{cc}
        P_1&O\\
        O&P_2
      \end{array}
    \right) \left(
      \begin{array}{cc}
        A&O\\
        O&B
      \end{array}
    \right)\left(
      \begin{array}{cc}
        Q_1&O\\
        O&Q_2
      \end{array}
    \right) = \left(
      \begin{array}{cccc}
        E_{r_1}&O&O&O\\
        O&O&O&O\\
        O&O&E_{r_2}&O\\
        O&O&O&O\\
      \end{array}
    \right)
  \end{equation*}
  
  (2)根据初等变换等价标准型有$P_1AQ_1 = \left(
    \begin{array}{cc}
      E_{r_1}&O\\
      O&O
    \end{array}
  \right), P_2BQ_2 = \left(
    \begin{array}{cc}
      E_{r_2}&O\\
      O&O
    \end{array}
  \right)$,因此
  \begin{equation*}
    \left(
      \begin{array}{cc}
        P_1&O\\
        O&P_2
      \end{array}
    \right) \left(
      \begin{array}{cc}
        A&O\\
        C&B
      \end{array}
    \right)\left(
      \begin{array}{cc}
        Q_1&O\\
        O&Q_2
      \end{array}
    \right) =
    \left(
      \begin{array}{cccc}
        E_{r_1}&O&O&O\\
        O&O&O&O\\
        C_{11}&C_{12}&E_{r_2}&O\\
        C_{21}&C_{22}&O&O\\
      \end{array}
    \right)
    \rightarrow 
    \left(
      \begin{array}{cccc}
        E_{r_1}&O&O&O\\
        O&O&O&O\\
        O&O&E_{r_2}&O\\
        O&C_{22}&O&O\\
      \end{array}
    \right)
  \end{equation*}
\end{proof}

\begin{theorem}[矩阵秩基本不等式]
  \begin{itemize}
  \item $r(AB) \leq \min\{r(A),r(B)\}$
  \item $r(A + B) \leq r(A) + r(B)$
  \item Sylvester不等式:$A,B$是$n$阶方阵,则$r(A) + r(B) \leq r(AB) + n $
  \item Frobenius不等式:$A,B,C$是$n$阶方阵,则$r(ABC) \geq r(AB) + r(BC) - r(B)$
  \item $A,B$为$s \times n, n\times m$阶矩阵,$AB = O$,则$r(A) + r(B) \neq n$
  \item $A$是$m \times n$的实矩阵,则$r(A^TA) = r(A)$
  \end{itemize}
\end{theorem}

\begin{proof}
  (1)将$B$写为行向量形式:$B = \left(
    \begin{array}{c}
      \beta_1\\
      \beta_2\\
      \vdots\\
      \beta_n
    \end{array}
  \right)$,则
  \begin{equation*}
    AB = \left(
      \begin{array}{ccc}
        a_{11}&\cdots&a_{1n}\\
        \vdots&&\vdots\\
        a_{n1}&\cdots&a_{nn}
      \end{array}
    \right) \left(
      \begin{array}{c}
        \beta_1\\
        \vdots\\
        \beta_n
      \end{array}
    \right) = \left(
      \begin{array}{c}
        a_{11}\beta_1 + \cdots + a_{1n}\beta_n\\
        \vdots\\
        a_{n1}\beta_1 + \cdots + a_{nn}\beta_n
      \end{array}
    \right)
  \end{equation*}
  因此$AB$每个行向量都可被$B$行向量线性表出,从而$r(AB) \leq r(B)$,同理可知$r(AB) \leq r(A)$

  (2)用列向量角度,$A+B$的列向量显然可以被$A,B$的列向量线性表出,从而结论成立。
  
  
  (3)是(4)的推论
  
  (4)Frobenius:即构造$r(ABC) + r(B) \geq r(AB) + r(BC)$
  \begin{equation*}
    \left[
      \begin{array}{cc}
        ABC& \\
           &B
      \end{array}
    \right] \rightarrow \left[
      \begin{array}{cc}
        ABC&O \\
           BC&B
      \end{array}
    \right] \rightarrow \left[
      \begin{array}{cc}
       O &-AB \\
         BC&B
      \end{array}
    \right]
  \end{equation*}
  根据分块矩阵秩不等式可知
\end{proof}

~

\begin{exercise}[秩不等式的推广]
  (1)$A,B$为$n$阶方阵,则$r(A - ABA) = r(A) + r(E - BA) - n$

  (2)$A,B$是$n$阶方阵,$AB = BA = O, r(A^2) = r(A)$,证明:$r(A + B) = r(A) + r(B)$
\end{exercise}

\begin{proof}
  (1)等号需要构造单位矩阵,即等式中的$n$用$E_n$代替:
  \begin{equation*}
    \left[
      \begin{array}{cc}
        A&\\
        &E_n - BA
      \end{array}
    \right] \rightarrow \left[
      \begin{array}{cc}
        A&O\\
        BA&E_n - BA
      \end{array}
    \right] \rightarrow \left[
      \begin{array}{cc}
        A&A\\
        BA&E_n
      \end{array}
    \right] \rightarrow \left[
      \begin{array}{cc}
        A - ABA&O \\
               O&E_n
      \end{array}
    \right]
  \end{equation*}
  从右往左推非常难,因此一般从无$E_n$的一边推另一边(后面的练习也会体现这一点)

  (2)
  注意从复杂的一侧$A+B$开始,否则$A,B$不知道怎么搞,
  \begin{equation*}
    \left(
      \begin{array}{cc}
        A+B&O\\
        O&O
      \end{array}
    \right) \rightarrow \left(
      \begin{array}{cc}
        A+B&O\\
        A^2 + AB&O
      \end{array}
    \right) = \left(
      \begin{array}{cc}
        A+B&O\\
        A^2&O
      \end{array}
    \right)  \rightarrow \left(
      \begin{array}{cc}
        A+B&A^2\\
        A^2&A^3
      \end{array}
    \right)
  \end{equation*}
  由于$r(A) = r(A^2)$,因此矩阵方程$A^2X = A$有解,故$\exists P$使得$A^2P = A$,
  故第二列可右乘$P$消去第一列:
  \begin{equation*}
    \left(
      \begin{array}{cc}
        A+B&A^2\\
        A^2&A^3
      \end{array}
    \right) \rightarrow \left(
      \begin{array}{cc}
        B&A^2\\
        O&A^3
      \end{array}
    \right)
  \end{equation*}
  因此根据分块矩阵的秩得到$r(A + B) \geq r(A^3) + r(B) = r(A) + r(B)$,
  而不等式另一侧显然.
\end{proof}


~

\begin{exercise}[用秩不等式研究矩阵性质]
  (1)证明零矩阵:$A_{m \times n}, B_{n \times s}$,$r(B) = n$,证明:若$AB = O$,则$A = O$

  (2)判定可逆:$n$为奇数,$A,B$为$n$阶矩阵,$A^2 = O$,证明:$AB - BA$不可逆

  % (3)$A$是$n \times s$的实列满秩矩阵,$s < n$,证明:存在$n \times(n-s)$的实列满秩矩阵$B$使得$(A,B)$为可逆矩阵,
  % 且$B^T A = O$
\end{exercise}

\begin{proof}
  (1)由于$r(A) + r(B) \leq r(AB) + n$,
  而$r(B) = n$,得到$r(A) = 0$,即$A = O$

  (2)$n = 2k+1$,根据$A^2 = O$得到$2r(A) \leq 2k+1 + 0$,
  从而$r(A) \leq k$,
  此时
  \begin{equation*}
    r(AB - BA) \leq r(AB) + r(BA) \leq r(A) + r(A) \leq 2k
  \end{equation*}
  因此不可逆

  % (3)$B^TA = O$等价于$A^TB = O$,
  % $A$的秩为$s$,因此$A^TX = 0$有基础解系$\eta_1,\cdots,\eta_{n-s}$,
  % 取$B = (\eta_1,\cdots,\eta_{n-s})$,则显然$AB = O$。下面证明$(A,B)$可逆,
  % 设$(A,B)X = (A,B) \left(
  %   \begin{array}{c}
  %     X_s\\
  %     X_{n-s}
  %   \end{array}
  % \right) = AX_s + BX_{n-s} = 0$,
  % 同时作用$B^T$得到$B^TA X_s + B^TBX_{n-s} = B^TBX_{n-s} = 0$。
  % 而$r(B^TB) = r(B) = n-s$,故$X_{n-s} = 0$,因此方程组$(A,B)X = 0$只有零解,
  % 因此可逆。
\end{proof}

~

\subsection{空间分解法研究矩阵秩}

\begin{theorem}[多项式矩阵的秩关系]
  $f(x) = f_1(x)f_2(x)$是多项式,
  $(f_1(x),f_2(x)) = 1$,
  则$f(A) = O$当且仅当$r(f_1(A)) + r(f_2(A)) = n$。常见的推论如下:
  \begin{itemize}
  \item $A^2 = A$当且仅当$r(A - E) + r(A) = n$
  \item $A^2 = E$当且仅当$r(A - E) + r(A + E) = n$
  \end{itemize}
\end{theorem}

\begin{proof}
  根据$(f_1(x),f_2(x)) = 1$,有$u(x)f_1(x) + v(x) f_2(x) = 1$,
  因此$u(A)f_1(A) + v(A)f_2(A) = E_n$,
  根据分块矩阵初等变换:
  \begin{equation*}
    \left[
      \begin{array}{cc}
        f_1(A)&\\
        &f_2(A)
      \end{array}
    \right] \rightarrow \left[
      \begin{array}{cc}
        f_1(A)&f_1(A)u(A) + v(A)f_2(A)\\
        O&f_2(A)
      \end{array}
    \right] = \left[
      \begin{array}{cc}
        f_1(A)&E_n\\
        O&f_2(A)
      \end{array}
    \right]
  \end{equation*}
  再做初等变换:
  \begin{equation*}
    \left[
      \begin{array}{cc}
        f_1(A)&E_n\\
        O&f_2(A)
      \end{array}
    \right]
    \rightarrow \left[
      \begin{array}{cc}
        O&E_n\\
        -f_2(A)f_1(A)& f_2(A)
      \end{array}
    \right] \rightarrow \left[
      \begin{array}{cc}
        O&E_n\\
        -f(A)&O
      \end{array}
    \right]
  \end{equation*}
  因此得到$r(f_1(A)) + r(f_2(A)) = n + r(f(A))$
\end{proof}

\begin{note}
  分块矩阵、多项式互素、空间直和本质上是一体的。
\end{note}

~

\begin{exercise}[从线性空间角度解决难题]
  $A,B$是$P$上$n$阶矩阵,$AB = BA$,证明:$r(A) + r(B) \geq r(AB) + r(A+B)$
\end{exercise}

\begin{proof}
  设$AX = 0, BX = 0$的解空间分别为$V_1,V_2$,
  设$ABX = BAX = 0$的解空间为$W_1$,
  $(A+B)X = 0$的解空间为$W_2$。
  显然$V_1 \subseteq W_1, V_2 \subseteq W_1$,
  从而$V_1 + V_2 \subseteq W_1$,
  显然$V_1 \cap V_2 \subseteq W_2$,根据维数公式:
  \begin{equation*}
    dim V_1 + dim V_2 = dim(V_1 + V_2) + dim(V_1 \cap V_2) \leq dim W_1 + dim W_2
  \end{equation*}
  因此推出:
  \begin{equation*}
    (n - r(A)) + (n - r(B)) \leq (n - r(AB)) + (n - r(A+B))
  \end{equation*}
  化简得到结论。
\end{proof}

\begin{note}
  这种$r(A),r(B),r(A+B),r(AB)$相关问题绝大多数从线性空间角度都能迎刃而解
\end{note}

\section{矩阵的迹}

\begin{theorem}[矩阵迹的性质]
  $A,B$为$n$阶方阵,则
  \begin{itemize}
  \item $\mathrm{tr}(A) = \mathrm{tr}(kA)$
  \item $\mathrm{tr}(A + B) = \mathrm{tr}(A) + \mathrm{tr}(B)$
  \item $\mathrm{tr}(AB) = \mathrm{tr}(BA)$
  \item 若$A$为实方阵,则$A = O$当且仅当$\mathrm{tr}(A^TA) = 0$
  \end{itemize}
\end{theorem}

\begin{proof}
  显然
\end{proof}


\section{几种技巧矩阵}

\subsection{初等矩阵}

\begin{definition}[初等矩阵]
  单位矩阵通过一次初等变换得到的矩阵称为初等矩阵。
  记作$P(i,j),P(i(c)),P(i,j(k))$表示$i,j$行互换,$i$行乘$c$,$P_{ij} = k$的矩阵
\end{definition}

\begin{theorem}[初等矩阵乘法]
  $AP$相当于做一次列初等变换,
  $PA$相当于做一次行初等变换,
\end{theorem}





\subsection{基本矩阵:仅一个元素为1的矩阵}

\begin{definition}[基本矩阵]
  基本矩阵$E_{ij}$表示$n\times n$矩阵,但是仅有$(i,j)$元素为$1$,其余元素为$0$的矩阵。
\end{definition}

\begin{theorem}[基本矩阵乘法]
  基本矩阵乘法有以下性质:
  \begin{itemize}
  \item 
    $AE_{ij}$将$A$的第$i$列移动到第$j$列,
  \item 
    $E_{ij}A$将$A$的第$j$行移动到第$i$行。
  \item 基础矩阵分解:$E_{ij} = e_ie^T_j$
  \item 基本矩阵相乘:
    $E_{ik}E_{kj} = E_{ij}$,
    $k \neq l$时$E_{ik}E_{lj} = 0$
  \end{itemize}
\end{theorem}

~

\begin{exercise}[用基础矩阵研究幂次]
  (1)设$A=\left(\begin{array}{ccccc}
              0 & 1 & 0 & \cdots & 0 \\
              0 & 0 & 1 & \cdots & 0 \\
              \vdots & \vdots & \vdots & & \vdots \\
              0 & 0 & 0 & \cdots & 1 \\
              1 & 0 & 0 & \cdots & 0
            \end{array}\right)$,
          证明:$A^k = \left(
            \begin{array}{cc}
              O&E_{n-k}\\
              E_k&O
            \end{array}
  \right)$。

  (2)$A$是$n$阶上三角阵且主对角上元素均为$0$,证明$A^n = O$
\end{exercise}

\begin{proof}
  (1)写为$A = (e_n,e_1,\cdots,e_{n-1})$,
  则$A^2 = (Ae_n,Ae_1,\cdots,Ae_{n-1})$。
  显然$Ae_i$为$A$的第$i$列,因此
  \begin{equation*}
    A^2 = (e_{n-1},e_n,e_1\cdots,e_{n-2})
  \end{equation*}
  以此类推容易得到结论

  (2)$A = \sum\limits_{i < j} a_{ij} E_{ij}$,
  而$j \neq k$时$E_{ij}E_{kl} = O$,
  因此$A^n$展开式中非零项只能是$E_{ij_1}E_{j_1j_2}\cdots E_{j_{n-1}j_n}$这样的项,
  但满足$1 \leq i < j_1 < \cdots < j_n \leq n$的项不存在,因此$A^n = O$
\end{proof}

~

\begin{exercise}[用二次型研究矩阵性质]
  (1)证明:$n$阶对称矩阵$A$为零矩阵当且仅当$\forall \alpha \in \mathbb{R}^n$有$\alpha^TA\alpha = 0$   

  (2)证明:$n$阶方阵$A$是反对称矩阵当且仅当$\forall \alpha \in \mathbb{R}^n$有$\alpha^T A \alpha = 0$
\end{exercise}

\begin{proof}
  (1)左推右显然。先设$\alpha = e_i$,则$e^T_i Ae_i = a_{ii} = 0$,
  再取$\alpha = e_i + e_j$,则
  \begin{equation*}
    0 = (e_i + e_j)^T A(e_i + e_j) = a_{ii}+a_{jj} + a_{ij} + a_{ji}
  \end{equation*}
  而$a_{ij} = a_{ji}$,因此得到$a_{ij }= 0, \forall i,j$

  (2)左推右用$\alpha^TA\alpha =  (\alpha^TA\alpha)^T = -\alpha^TA\alpha $即可。
  右推左:根据条件得到$\alpha^T A^T \alpha = 0$,因此$\alpha^T (A + A^T)\alpha = 0$,
  由于$A + A^T$为对称矩阵,根据上题结论得到$A + A^T = O$,因此$A^T = -A$
\end{proof}



\subsection{置换矩阵}

\begin{definition}[置换矩阵]
  每行只有一个元素为$1$,每列也只有一个元素为$1$,则称该矩阵为置换矩阵。
  一般写为$P = [\epsilon_{i_1},\cdots,\epsilon_{i_n}] = [\epsilon_{j_1}^T,\cdots,\epsilon_{j_n}^T]^T$,其中$\epsilon_i$表示仅第$i$元素为$1$的列向量,$i_k,j_k$分别表示列、行的重排顺序。
\end{definition}

\begin{theorem}[置换矩阵是正交矩阵]
  置换矩阵是正交矩阵。
\end{theorem}

\begin{proof}
  证明$P^TP = I$
\end{proof}

\begin{theorem}[置换矩阵的作用]
  左乘置换矩阵相当于把行原本的$j_1,j_2,\cdots,j_n$行换到$1,2,\cdots,n$行,
  右乘相当于把列原本的$i_1,\cdots,i_n$列换到$1,2,\cdots,n$列。
\end{theorem}

\begin{theorem}[置换矩阵正交相似]
  $i_1\cdots i_n$是$1 \sim n$的排列,$A = (a_{ij})$,取$P = [\epsilon_{i_1},\cdots,\epsilon_{i_n}]$,
  则:
  \begin{equation*}
    P^TAP =P^{-1}AP= \left[
      \begin{array}{cccc}
        a_{i_1i_1}&a_{i_1i_2}&\cdots&a_{i_1i_n}\\
        a_{i_2i_1}&a_{i_2i_2}&\cdots&a_{i_2i_n}\\
        \vdots&\vdots&&\vdots\\
        a_{i_ni_1}&a_{i_ni_2}&\cdots&a_{i_ni_n}
      \end{array}
    \right]
  \end{equation*}
  相当于将行列均按新顺序排列了一遍(先把行放到对应位置,再把列放到对应位置),且是一种正交变换。
\end{theorem}

\begin{exercise}[置换矩阵正交相似例子]
  $i_1 \cdots i_n$是$1,2,\cdots,n$的排列,
  $A = \text{diag}\{\lambda_1,\cdots,\lambda_n\}, B = \text{diag}\{\lambda_{i_1},\cdots,\lambda_{i_n}\}$,
  证明$A,B$是正交相似的
\end{exercise}

\begin{proof}
  取$P = [\epsilon_{i_1},\cdots,\epsilon_{i_n}]$即可。
\end{proof}

\begin{definition}[特殊的置换矩阵]
  特殊置换矩阵$K = \left[
    \begin{array}{cccc}
      &&&1 \\
      &&1& \\
      &\reflectbox{$\ddots$}&& \\
      1&&&
    \end{array}
  \right]$,
  $K^TAK$的功能是绕中心转180度(行列全部倒排)
\end{definition}





\section{特殊矩阵性质总结}

\subsection{对角矩阵}

\begin{theorem}[对角矩阵乘法]
  矩阵$A$右乘一个对角阵,则每列乘上对角相应元素。
  $A$左乘对角阵,则每行乘上对角相应元素。
\end{theorem}

\begin{theorem}[对角矩阵求逆]
  $\text{diag}\{d_1,\cdots,d_n\}^{-1} = \text{diag}\{d_1^{-1},\cdots,d_n^{-1}\}$
\end{theorem}

\subsection{上、下三角矩阵}

\begin{theorem}[上下三角乘法、逆、伴随]
  $A,B$是上三角矩阵,则
  \begin{itemize}
  \item $AB$仍然是上三角矩阵,且$AB$对角元为$a_{11}b_{11},\cdots,a_{nn}b_{nn}$
  \item $A^{-1}$也是上三角阵,且$A^{-1}$对角元为$a_{11}^{-1},\cdots,a_{nn}^{-1}$
  \item $A^{*}$也是上三角阵,且$A^{*}$的对角元为$a_{22}a_{33}\cdots a_{nn},a_{11}a_{33}\cdots a_{nn},\cdots,a_{11}a_{22}\cdots a_{n-1,n-1}$(也是特征值)
  \end{itemize}
\end{theorem}

\begin{proof}
  (1)令$C = AB$,若$c_{ij} = \sum\limits_{k = 1}^n a_{ik}b_{kj} = \sum\limits_{k = 1}^{i-1}a_{ik}b_{kj} + \sum\limits_{k = i}^n a_{ik}b_{kj}$,
  当$i > j$时,显然$c_{ij} = 0$。
  而$c_{ii} = \sum\limits_{k = 1}^{i-1}a_{ik}b_{ki} + a_{ii}b_{ii} + \sum\limits_{k = i+1}^na_{ik}b_{ki} = a_{ii}b_{ii}$

  (2)根据行变换可以将$(A,I) \rightarrow (I,A^{-1})$,当$A$为上三角时,
  先用最后一行消去$A$最后一列其余行的元素,以此类推可以发现$A^{-1}$为上三角,
  根据矩阵乘法显然有$A^{-1}$对角为$A$对角的逆。

  (3)根据$A^{\ast} = |A| A^{-1}$即可
\end{proof}

\begin{theorem}[n个上三角矩阵的乘积]
  $A_1,\cdots,A_n$时$n$个对角为$0$的$n$阶上三角矩阵,
  证明$A_1A_2\cdots A_n = O$
\end{theorem}

\begin{proof}
  用数学归纳法,设$n-1$时成立,取$A_n = \left[
    \begin{array}{cc}
      B_{n-1}&*\\
      0&0
    \end{array}
  \right]$,此时
  \begin{equation*}
    A_1\cdots A_n = \left[
      \begin{array}{cc}
        B_1\cdots B_{n-1}&*\\
        0&0
      \end{array}
    \right] \cdot \left[
      \begin{array}{cc}
        B_n&*\\
        0&0
      \end{array}
    \right] = \left[
      \begin{array}{cc}
        0&*\\
        0&0
      \end{array}
    \right] \left[
      \begin{array}{cc}
        B_n&*\\
        0&0
      \end{array}
    \right] = O
  \end{equation*}
\end{proof}


\subsection{实对称矩阵}

\begin{definition}[实对称矩阵]
  每个元素都是实数,且满足$A^T = A$的矩阵$A$称为实对称矩阵
\end{definition}

\begin{theorem}[实对称矩阵的矩阵性质]
  实对称矩阵$A$具有以下性质:
  \begin{itemize}
  \item $A^{-1}$也是实对称矩阵
  \end{itemize}
\end{theorem}

\begin{proof}
  (1)由于$(AA^{-1})^T = E_n$,得到$(A^{-1})^TA^T = E_n$,而$A$实对称,
  因此$(A^{-1})^TA = E_n$,这说明$(A^{-1})^T = A^{-1}$
\end{proof}

\begin{theorem}[实对称矩阵的特征值性质]
  实对称矩阵有如下性质:
  \begin{itemize}
  \item 每个特征值都是实数
  \item 不同特征值的特征向量正交
  \item 一定正交相似(相似且合同)于对角矩阵
  \end{itemize}
\end{theorem}

\begin{proof}
  (1)共轭转置法:设$\lambda \in \mathbb{C}$是特征值,$\xi \in \mathbb{C}^n$是特征向量,
  一方面根据$A\xi = \lambda \xi$两侧同时左乘$\overline{\xi}^T$得到
  \begin{equation*}
    \overline{\xi}^T A\xi = \lambda \overline{\xi}^T \xi 
  \end{equation*}
  另一方面直接对$A\xi = \lambda \xi$取共轭转置,再同时右乘$\xi$得到
  \begin{equation*}
    \overline{\xi}^T A = \overline{\xi}^T \overline{\lambda} \Rightarrow \overline{\xi}^T A \xi = \overline{\lambda} \overline{\xi}^T \xi
  \end{equation*}
  因此比较得到$\lambda = \overline{\lambda}$
  
  (2)不同特征值的特征向量正交:设$\lambda_1,\lambda_2$是两个互异特征值,$\xi_1,\xi_2$分别属于$\lambda_1,\lambda_2$的实特征向量,
  则
  \begin{equation*}
    \lambda_1(\xi_1,\xi_2) = (\lambda_1\xi_1)^T\xi_2 = (A\xi_1)^T\xi_2 = \xi^T_1(A\xi_2) = \lambda_2(\xi_1,\xi_2)
  \end{equation*}
  从而得到$(\lambda_2 - \lambda_1)(\xi_1,\xi_2) = 0$,而$\lambda_1 \neq \lambda_2$,
  从而$(\xi_1,\xi_2) = 0$

  (3)做数学归纳法,$n = 1$显然,假设$n - 1$成立,
  任取特征值$\lambda$,对应单位特征向量$\alpha_1$,
  将$\alpha_1$扩充为$\mathbb{R}^n$的一组标准正交基$\alpha_1,\cdots,\alpha_n$,令$T = (\alpha_1,\cdots,\alpha_n)$,
  则
  \begin{equation*}
    AT = T \left(
      \begin{array}{cc}
        \lambda&\alpha^T\\
        0&A_{n-1}
      \end{array}
    \right) \Rightarrow T^T AT = \left(
      \begin{array}{cc}
        \lambda&\alpha^T \\
               0&A_{n-1}
      \end{array}
    \right)
  \end{equation*}
  由于$T^TAT$对称,因此$\alpha^T = 0$,再根据归纳假设,
  存在正交矩阵$P_1$使得$P_1^TA_{n-1}P_1$为对角阵,
  因此令$P = \mathrm{diag}(1, P_1)$,
  此时
  \begin{equation*}
    P^TT^TATP = \left(
      \begin{array}{cc}
        \lambda&0\\
        0&P^TA_{n-1}P
      \end{array}
    \right)
  \end{equation*}
  显然$TP = Q$,则$Q$也是正交阵。
\end{proof}


\subsection{反对称矩阵}


\begin{theorem}[反对称矩阵的行列式]
  奇数阶反对称矩阵的行列式必为$0$。
\end{theorem}

\begin{proof}
  根据$A^T = -A$得到$|A| = (-1)^n|A|$,当$n$为奇数时$|A| = -|A|$,因此$|A| = 0$
\end{proof}

\begin{theorem}[反对称矩阵的矩阵性质]
  $A$为反对称矩阵,则
  \begin{itemize}
  \item $A^{-1}$为反对称矩阵
  \end{itemize}
\end{theorem}

\begin{proof}
  (1)类似实对称矩阵的
\end{proof}

\begin{theorem}[反对称矩阵的二次型]
  $A$为反对称矩阵当且仅当任意$X \in \mathbb{R}^n$,
  有$X^TAX = 0$
\end{theorem}

\begin{proof}
  在基础矩阵技巧一节中给出过证明
\end{proof}

\begin{theorem}[反对称矩阵的特征值、特征向量]
  反对称矩阵$A$的特征值是$0$或者纯虚数。
  且纯虚数对应特征向量的实部和虚部的实向量等长、正交。
\end{theorem}

\begin{proof}
  (1)设$A$的特征值$\lambda = a + bi$,对应特征向量$x = u + vi$,
  $A(u + vi) = (a + bi)(u + vi)$,因此$Au + iAv = (au - bv) + (bu + av)i$,
  实虚部分离得到:
  \begin{equation*}
    Au = au - bv,Av = bu + av \Rightarrow u^TAu = au^Tu - bu^Tv , v^TAv = bv^Tu + a v^Tv
  \end{equation*}
  因此得到$u^TAu + v^TAv = a(|u|^2 + |v|^2)$,由于$u^TAu \in \mathbb{C}$是一个数,因此
  \begin{equation*}
    u^TAu = (u^TAu)^T = u^TA^Tu = - u^TAu \Rightarrow u^TAu = 0
  \end{equation*}
  同理有$v^TAv = 0$,因此$a(|u|^2 + |v|^2) = 0$,
  由于$u + vi \neq 0$,得到$a = 0$,即$\lambda = a + bi = bi$

  (2)设$\alpha + \beta i$是特征值$k i$的特征向量,
  即$A(\alpha + i \beta) = k i(\alpha + i\beta)$,
  对比实部虚部,得到
  \begin{equation*}
    A \alpha = - k \beta, A \beta = k \alpha
  \end{equation*}
  由于$\alpha^TA\alpha = 0$,得到$\alpha^T A \alpha = \alpha^T (-k \beta) = -k \alpha^T\beta = 0$,
  这说明$\alpha,\beta$正交,
  再由于
  \begin{equation*}
    \alpha^T A \beta + \beta^T A \alpha =
    \begin{cases}
      \alpha^T A \beta - (\alpha^T A \beta)^T = 0\\
      k (\alpha^T \alpha - \beta^T \beta)
    \end{cases}
  \end{equation*}
  得到$\alpha^T \alpha - \beta^T \beta = 0$,即模长相等
\end{proof}

\begin{theorem}[反对称矩阵的标准型]
  $A$为$n$阶反对称矩阵,则
  \begin{itemize}
  \item $A$正交相似于如下准对角阵
    \begin{equation*}
      \mathrm{diag}\left\{ \left(
          \begin{array}{cc}
            0&b_1\\
            -b_1&0
          \end{array}
        \right), \cdots,
      \left(
        \begin{array}{cc}
          0&b_s\\
          -b_s&0
        \end{array}
      \right),0,\cdots,0\right\}
    \end{equation*}
  \item $A$合同于标准型
    \begin{equation*}
      \mathrm{diag}\left\{ \left(
          \begin{array}{cc}
            0&1\\
            -1&0
          \end{array}
        \right),\cdots, \left(
          \begin{array}{cc}
            0&1\\
            -1&0
          \end{array}
        \right),0,\cdots,0 \right\}
    \end{equation*}
  \end{itemize}
\end{theorem}


\begin{theorem}[将矩阵拆为对称+反对称]
  $A = \frac{1}{2}(A + A^T) + \frac{1}{2}(A - A^T)$
\end{theorem}



\subsection{Hermite矩阵}

\begin{definition}[Hermite矩阵]
  $\overline{A}^T = A$的矩阵称为Hermite矩阵
\end{definition}

\begin{theorem}[Hermite矩阵的性质]
  Hermite矩阵的特征值都是实数
\end{theorem}



\subsection{正交矩阵}

\begin{definition}[正交矩阵]
  若矩阵$T$满足$TT^T = I$,则称$T$为正交矩阵。
\end{definition}

\begin{theorem}[正交矩阵的性质]
  $A,B$为$n$阶正交矩阵,则
  \begin{itemize}
  \item 求逆:$A^{-1} = A^T$
  \item $A^{-1},A^T, AB$都是正交矩阵
  \item $S = \left[
      \begin{array}{cc}
        A&0\\
        0&B
      \end{array}
    \right]$也是正交矩阵
  \item $A$正交当且仅当$A$的列向量是$\mathbb{R}^n$的一组标准正交基。
  \end{itemize}
\end{theorem}

\begin{proof}
  (1)显然

  (2)$(AB)^TAB = I$

  (3)直接用$S^TS$验证即可

  (4)$A = [\alpha_1,\cdots,\alpha_n]$,计算$A^TA = I$可得出结论。
\end{proof}

~

\begin{exercise}[正交矩阵练习]
  $b$是单位列向量,证明存在$n$阶对称正交矩阵$A$,使得$b$为$A$的第一列
\end{exercise}

\begin{proof}
  
\end{proof}


~

\begin{theorem}[分块上三角正交矩阵]
  若$A = \left[
    \begin{array}{cc}
      A_1&A_2\\
      0&A_4
    \end{array}
  \right]$是正交矩阵,则$A_2 = 0$,$A_1,A_4$是正交矩阵
\end{theorem}

\begin{proof}
  根据分块矩阵转置以及乘法可以得到结论
  \begin{equation*}
    A^TA = \left[
      \begin{array}{cc}
        A_1^T&0\\
        A_2^T&A_4^T
      \end{array}
    \right] \left[
      \begin{array}{cc}
        A_1&A_2\\
        0&A_4
      \end{array}
    \right]
  \end{equation*}
\end{proof}

\begin{theorem}[正交矩阵的行列式与特征值]
  $A$是$n$阶正交矩阵,
  则$A$的特征值模长为$1$,复特征值成对出现,
  $|A| = \pm 1$,进一步地
  \begin{itemize}
  \item 若$|A| = -1$:$-1$一定是$A$的特征值
  \item 若$A$是奇数阶的,且$|A| = 1$,则$1$一定是$A$的特征值
  \item 复特征值的特征向量实部和虚部模长相等且正交
  \end{itemize}
\end{theorem}

\begin{proof}
  (1)主命题:$A\alpha = \lambda \alpha$,两侧取共轭得到$A\overline{\alpha} = \overline{\lambda} \overline{\alpha}$,
  两侧取转置得到$\overline{\alpha}^T A^T = \overline{\lambda} \overline{\alpha}^T $,
  两侧右乘$\alpha$得到$\overline{\alpha}^T A^{-1}\alpha = \overline{\lambda} \overline{\alpha}^T \alpha$,
  因此$\frac{1}{\lambda}\overline{\alpha}^T \alpha = \overline{\lambda} \overline{\alpha}^T \alpha$,
  因此$\lambda \overline{\lambda} = 1$,即模长为$1$。
  根据$|A^TA| = 1$可知行列式为$\pm 1$

  (2)$|A| = -1$情况:根据$|A|$是特征值乘积可知至少有$1$个$-1$

  (3)$|A| = 1$情况:同理
\end{proof}

\begin{note}
  在计算$A$的特征值时,如果$A$的特殊性需要$A^T$来体现,则经常两侧取共轭转置(例如实对称、反对称、正交矩阵)。
\end{note}

\begin{theorem}[$n$阶正交矩阵的标准型]
  $A$是$n$阶正交矩阵,则$A$总是正交相似于
  \begin{equation*}
    \mathrm{diag} \left\{ \lambda_1,\cdots,\lambda_r, \left(
        \begin{array}{cc}
          \cos \theta_1&- \sin \theta_1 \\
                       \sin \theta_1&\cos \theta_1
        \end{array}
        ,\cdots,
        \left(
          \begin{array}{cc}
            \cos \theta_m&- \sin \theta_m \\
                         \sin \theta_m& \cos \theta_m
          \end{array}
        \right)
      \right) \right\}
  \end{equation*}
\end{theorem}


\subsection{Unitary矩阵}

\begin{theorem}[Unitary矩阵的性质]
  Unitary矩阵的特征值模长均为$1$
\end{theorem}




\subsection{幂零矩阵}

\begin{definition}[幂零矩阵]
  若存在整数$l$,使得$A^l = O$,则称$A$为幂零矩阵。
\end{definition}

\begin{theorem}[幂零矩阵充要条件]
  $A$是幂零矩阵当且仅当
  \begin{enumerate}
  \item 特征值:$A$的特征值只有$0$
  \item 特征多项式:$A$的特征多项式为$x^n$
  \item 最小多项式:$A$的最小多项式为$x^k, k \leq n$
  \item 迹(重点):对$\forall k > 0$,$\mathrm{tr}(A^k) = 0$
  \end{enumerate}
\end{theorem}




\subsection{幂等矩阵}

\begin{definition}[幂等矩阵]
  若$n$阶矩阵$A$满足$A^2 = A$,则称$A$为幂等矩阵
\end{definition}

\begin{theorem}[幂等矩阵充要条件]
  $n$阶方阵$A$是幂等矩阵当且仅当$r(A) + r(A - E) = n$
\end{theorem}

\begin{proof}
  由于$A^2 = A$,即$A(A - E) = O$,
  根据空间分解研究矩阵秩定理(具体见线性空间-直和与空间分解)
  \begin{equation*}
    n = n - r(A) + n - r(A-E)
  \end{equation*}
  因此$r(A) + r(A - E) = n$
\end{proof}

\begin{theorem}[幂等矩阵的性质]
  $A$为$n$阶幂等矩阵,则
  \begin{itemize}
  \item $A$的特征值只有$1$和$0$
  \item $A$一定可以对角化,且标准型为$\mathrm{diag}\{E_r,O\}$
  \item $r(A) = \mathrm{tr}(A)$
  \end{itemize}
\end{theorem}

\begin{proof}
  (1)设$\lambda$是$A$的特征值,故$A\alpha = \lambda \alpha$,
  $A^2 \alpha = \lambda^2 \alpha$,
  由于$A = A^2$,得到$\lambda^2 = \lambda$,因此$\lambda = 0, 1$

  (2)根据最小多项式理论,$f(A) = A(A - E) = O$,
  因此最小多项式只有一次项乘积,故可对角化。

  (3)根据秩、迹在相似下的不变性可知
\end{proof}

~

\begin{theorem}[加减乘法的幂等性]
  $A,B$为两个$n$阶幂等矩阵,则
  \begin{enumerate}
  \item $A+B$为幂等矩阵当且仅当$AB = BA = O$
  \item $A-B$为幂等矩阵当且仅当$AB = BA = B$
  \item 若$AB = BA$,则$AB$为幂等矩阵。反之不一定
  \item 如$E - A - B$可逆,则$r(A) = r(B)$
  \end{enumerate}
\end{theorem}

\begin{proof}
  (1)右推左:
  由于$AB = BA = O$,因此$(A+B)^2 = A^2 + AB + BA + B^2 = A+B$。
  左推右:$(A+B)^2 = A+AB+BA+B=A+B$,因此$AB+BA = O$,
  得到
  \begin{equation*}
    AB = AB^2 = (-BA)B = -BAB = B(BA) = BA
  \end{equation*}
\end{proof}

~

\begin{exercise}[幂等矩阵分解]
  (1)证明:任意方阵都可以分解为可逆矩阵与幂等矩阵的乘积
\end{exercise}

\begin{proof}
  (1)利用等价标准型得到:
  \begin{equation*}
    A = P \left(
      \begin{array}{cc}
        E_r&O\\
        O&O
      \end{array}
    \right)Q = PQ Q^{-1} \left(
      \begin{array}{cc}
        E_r&O\\
        O&O
      \end{array}
    \right) Q
  \end{equation*}
  $PQ$为可逆矩阵,后面为幂等
\end{proof}

~

\begin{theorem}[幂等矩阵列的性质]
  $A_1,\cdots,A_m$为$n$阶方阵,且$A_1+A_2 + \cdots + A_m = E_n$,则下面三个命题等价:
  \begin{itemize}
  \item $A_1,\cdots,A_m$为幂等矩阵
  \item $\mathrm{r}(A_1) + \cdots + \mathrm{r}(A_m) = n$
  \item $\forall i\neq j$有$A_iA_j = 0$
  \end{itemize}
\end{theorem}

\begin{proof}
  $(1) \Rightarrow (2)$:由于$r(A_i) = \mathrm{tr}(A_i)$,因此
  \begin{equation*}
    r(A_1) + \cdots + r(A_m) = \mathrm{tr}(A_1)+\cdots+ \mathrm{tr}(A_m) = \mathrm{tr}(A_1 + \cdots + A_m) = \mathrm{tr}(E_n) = n
  \end{equation*}

  $(2)\Leftarrow (3)$:比较经典,
  先通过分块矩阵的初等变换凑出$E$:
  \begin{equation*}
    \left(\begin{array}{ccccc}
            A_1 & & & & \\
                & A_2 & & & \\
                & & A_3 & & \\
                & & & \ddots & \\
                & & & & A_m
          \end{array}\right)\left(\begin{array}{ccccc}
                                    A_1 & A_2 & A_3 & \cdots & A_m \\
                                    A_2 & & & \\
                                        & & A_3 & & \\
                                        & & & \ddots & \\
                                        & & & & A_m
                                  \end{array}\right) \rightarrow\left(\begin{array}{ccccc}
                                                                        E & A_2 & A_3 & \cdots & A_m \\
                                                                        A_2 & A_2 & & & \\
                                                                        A_3 & & A_3 & & \\
                                                                        \vdots & & & \ddots & \\
                                                                        A_m & & & A_m
                                                                      \end{array}\right)
  \end{equation*}
  再通过$E$消去其他部分:
  \begin{equation*}
    \rightarrow\left(\begin{array}{ccccc}
                       E & O & O & \cdots & O \\
                       A_2 & A_2-A_2^2 & -A_2 A_3 & \cdots & -A_2 A_m \\
                       A_3 & -A_3 A_2 & A_3-A_3^2 & \cdots & -A_3 A_m \\
                       \vdots & \vdots & \vdots & \ddots & \vdots \\
                       A_m & -A_m A_2 & -A_m A_3 & \cdots & A_m-A_m^2
                     \end{array}\right) \rightarrow\left(\begin{array}{ccccc}
                                                           E & O & O & \cdots & O \\
                                                           O & A_2-A_2^2 & -A_2 A_3 & \cdots & -A_2 A_m \\
                                                           O & -A_3 A_2 & A_3-A_3^2 & \cdots & -A_3 A_m \\
                                                           \vdots & \vdots & \vdots & \ddots & \vdots \\
                                                           O & -A_m A_2 & -A_m A_3 & \cdots & A_m-A_m^2
                                                         \end{array}\right)
  \end{equation*}
  根据$r(A_1) + r(A_2) + \cdots + r(A_m) = n$,得到
  \begin{equation*}
    \left(\begin{array}{cccc}
            A_2-A_2^2 & -A_2 A_3 & \cdots & -A_2 A_m \\
            -A_3 A_2 & A_3-A_3^2 & \cdots & -A_3 A_m \\
            \vdots & \vdots & \ddots & \vdots \\
            -A_m A_2 & -A_m A_3 & \cdots & A_m-A_m^2
          \end{array}\right)=O
  \end{equation*}
  这说明$\forall i\neq j$有$A_i^2 = A_i$,且$A_iA_j = O(i,j = 2,3,\cdots,m)$,
  再对$A_1 + \cdots + A_m = E_n$两边同乘$A_i$得到
  \begin{equation*}
    A_1A_i + A_i = A_i
  \end{equation*}
  因此$A_1A_i = O$,结论成立

  $(3) \Rightarrow (1)$:直接对$A_1+\cdots +A_m = E_n$两边右乘$A_i$得到
  \begin{equation*}
    A_1A_i + A_2A_i + \cdots + A_mA_i = A_i^2 = A_i
  \end{equation*}
  因此都是幂等矩阵
\end{proof}

\begin{note}
  此定理在ZJU2020中考过
\end{note}

\subsection{对角占优矩阵}












% \section{矩阵中的其他常见问题}

% \subsection{矩阵的可交换问题}

% \noindent 一、循环矩阵$I = \left[
%   \begin{array}{cccccc}
%     0&1&0&\cdots&0&0 \\
%      0&0&1&\cdots&0&0 \\
%      0&0&0&\cdots&0&0 \\
%      \vdots&\vdots&\vdots&&\vdots&\vdots \\
%      0&0&0&\cdots&0&1 \\
%      1&0&0&\cdots&0&0
%   \end{array}
% \right]$

% \begin{theorem}[$I$的性质]
%   循环矩阵有下述性质:
%   \begin{itemize}
%   \item 右乘:$AI = [\alpha_n,\alpha_1,\cdots,\alpha_{n-1}]$,其中$\alpha_i$为$A$列向量
%   \item 左乘:$IA = [\beta_2,\cdots,\beta_n,\beta_1]^T$,其中$\beta_i$是$A$行向量
%   \item 幂次:$I^k(k \leq n)$每次将$1$的斜向向右上推一格,$I^k(k \geq n) = E$
%   \item $I$的多项式:$a_1I + a_2I^2 + \cdots + a_nI^n$各层斜对角相等的矩阵
%   \end{itemize}
% \end{theorem}

% \begin{proof}
%   (1)因为$I$是置换矩阵,考虑列向量情况下$i_1 = n, i_2 = 2,\cdots,i_n = n-1$,因此相当于把$A$的列重排为$\alpha_n,\alpha_1,\cdots,\alpha_{n-1}$

%   (2)同理考虑行变换下$j_1 = 2,j_2 = 3,\cdots,j_{n-1} = n, j_n = 1$,因此相当于把$A$的行重新排为$\beta_2,\cdots,\beta_n,\beta_1$
% \end{proof}

% \begin{theorem}[循环矩阵可交换]
%   $A$与循环矩阵$I$可交换的矩阵当且仅当$A$可写作$I$的多项式$a_1E + a_2I + \cdots + a_nI^{n-1}$(别忘了$E$)。
% \end{theorem}

% \noindent 二、零对角Jordan阵$J = \left[
%   \begin{array}{cccccc}
%     0&1&0&\cdots&0&0 \\
%     0&0&1&\cdots&0&0 \\
%     0&0&0&\cdots&0&0 \\
%     \vdots&\vdots&\vdots&&\vdots&\vdots \\
%     0&0&0&\cdots&0&1 \\
%     0&0&0&\cdots&0&0
%   \end{array}
% \right]
% $

% \begin{theorem}[零对角Jordan阵的性质]
%   零对角Jordan阵有以下性质:
%   \begin{itemize}
%   \item $AJ = [0,\alpha_1,\cdots,\alpha_{n-1}]$
%   \item $JA = [\beta_2,\cdots,\beta_n,0]^T$
%   \item 幂次:$J^k(k \neq n-1)$每次将$1$斜线往上推一层,$k \geq n-1$时$J^k = 0$
%   \end{itemize}
% \end{theorem}

% \begin{theorem}[$J$的可交换性质]
%   $A$与$J$可交换当且仅当$A$是$J$的多项式$a_1E + a_2J + \cdots + a_nJ^{n-1}$(别忘了$E$)
% \end{theorem}

% \noindent 三、综合练习题,有以下技巧:

% \begin{itemize}
% \item 要证明$AB = BA$,可做$(A - aE)B = B(A - aE)$,消去对角
% \item 能用$I,J$的就用,用不了的一般设未知数硬算
% \end{itemize}

% \begin{exercise}[$J$矩阵的应用]
%   求$P,R$可交换的矩阵
%   \begin{equation*}
%     P = \left[
%       \begin{array}{cc}
%        1 & 1\\
%         0&1
%       \end{array}
%     \right],
%     R = \left[
%       \begin{array}{ccc}
%         3&0&0\\
%         -1&3&0 \\
%          0&-1&3
%       \end{array}
%     \right]
%   \end{equation*}
% \end{exercise}

% \begin{solution}
%   (1)根据技巧等价于与$\left[
%     \begin{array}{cc}
%       0&1\\
%       0&0
%     \end{array}
%   \right]$可交换,
%   根据对角$0$Jordan阵可交换条件可知,形式为$\left[
%     \begin{array}{cc}
%       k_1&k_2 \\
%        0&k_1
%     \end{array}
%   \right]$

%   (2)等价于与$\left[
%     \begin{array}{ccc}
%       0&0&0\\ 
%        -1&0&0 \\
%        0&-1&0
%     \end{array}
%   \right]$可交换,
%   即$\left[
%     \begin{array}{ccc}
%       a_1&0&0\\
%       a_2&a_1&0\\
%       a_3&a_2&a_1
%     \end{array}
%   \right]$
% \end{solution}

% ~



% \subsection{矩阵方程$AX - XB = O$}











