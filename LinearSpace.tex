

\chapter{线性空间与线性变换}

\section{线性空间}

\subsection{线性空间的基本概念}

\begin{definition}[线性空间]
  $\mathbb{P}$是一个数域,$V$是非空集合,$V$上定义了二元运算$+$和数乘,若这两种运算满足:
  \begin{itemize}
  \item 二元运算:(1)交换(2)结合(3)零元(4)负元
  \item 数乘:(1)一元(2)结合$(c_1c_2)\alpha = c_1(c_2 \alpha)$(3)向量分配$(c_1+c_2)\alpha = c_1\alpha + c_2\alpha$(4)数分配$c(\alpha + \beta) = c\alpha + c \beta$
  \end{itemize}
  则称$V$是$\mathbb{P}$上的线性空间
\end{definition}




\subsection{线性相关、线性无关与向量组的秩}

\begin{definition}[线性相关和线性无关]
  $\alpha_1,\cdots,\alpha_n$为向量组,
  若存在不全为$0$的$k_1,\cdots,k_n$使得
  \begin{equation*}
    k_1\alpha_1 + k_2\alpha_2 + \cdots + k_n\alpha_n = 0
  \end{equation*}
  则称它们线性相关。若该等式成立当且仅当$k_1 = k_2 = \cdots = k_n = 0$,
  则称线性无关。
\end{definition}

~

\begin{exercise}[证明线性无关]
  (1)幂零基:$\mathcal{A}$为$V$上的线性变换,
  $\xi \in V$,
  若$\mathcal{A}^{k-1}\xi \neq 0, \mathcal{A}^k \xi = 0$,
  证明$\xi,\mathcal{A}\xi,\cdots,\mathcal{A}^{k-1}\xi$线性无关
\end{exercise}

\begin{proof}
  设$k_0 \xi + k_1\mathcal{A} \xi + \cdots + k_{n-1}\mathcal{A}^{k-1} \xi = 0$,
  两侧同时作用$\mathcal{A}^{k-1}$得到$k_0\mathcal{A}^{k-1}\xi = 0$,
  因此$k_0 = 0$,以此类推得到$k_0,k_1,\cdots,k_{n-1} = 0$
\end{proof}

~

\begin{theorem}[极大线性无关组的计算]
  给定一列向量$\alpha_1,\cdots,\alpha_n$,将它们作为矩阵的列向量拼成一个矩阵,
  对矩阵做初等行变换化为阶梯型矩阵,阶梯头所在列对应的向量为极大线性无关组。(其实按行也可以,按列更不容易出错,因为换行不影响结果)
\end{theorem}

~

\begin{exercise}[极大线性无关组的计算]
  计算下列向量组的极大线性无关组
  \begin{equation*}
    \begin{cases}
      \alpha_1 = (1,-1,2,4)\\
      \alpha_2 = (0,3,1,2)\\
      \alpha_3 = (3,0,7,14)\\
      \alpha_4 = (1,-1,2,0)\\
      \alpha_5 = (2,1,5,6)
    \end{cases}
  \end{equation*}
\end{exercise}

\begin{solution}
  拼成一个矩阵,并做初等行变换:
  \begin{equation*}
    \left[
      \begin{array}{ccccc}
        1&0&3&1&2\\ 
        -1&3&0&-1&1 \\
         2&1&7&2&5 \\
         4&2&14&0&6
      \end{array}
    \right] \rightarrow \left[
      \begin{array}{ccccc}
        1&0&3&1&-2 \\
         0&1&1&0&1 \\
         0&0&0&1&1 \\
         0&0&0&0&0
      \end{array}
    \right]
  \end{equation*}
  因此极大线性无关组为$\alpha_1, \alpha_2, \alpha_4$
\end{solution}

~

\begin{definition}[向量组的秩]
  $\alpha_1,\cdots,\alpha_n$为向量组,
  称其中极大线性无关组的向量个数为它们的秩。
\end{definition}

\begin{theorem}[源泉定理及其推论]
  设$\alpha_1,\cdots,\alpha_r$可以被$\beta_1,\cdots,\beta_s$线性表出,
  $r > s$,则$\alpha_1,\cdots,\alpha_r$线性相关。
\end{theorem}


\subsection{线性空间同构}

\begin{definition}[同构]
  若两个代数系统之间存在双射,且该双射保持代数系统的运算,则称两个代数系统同构。
\end{definition}

\begin{theorem}[有限维线性空间的同构映射]
  $V$为$\mathbb{P}$上线性空间,
  $\alpha_1,\cdots,\alpha_n$为$V$的一组基,
  构造线性映射$f$,将$\forall \alpha \in V$映射到其坐标,
  则$f$为$V \rightarrow \mathbb{P}^n$的同构映射
\end{theorem}

~

\begin{theorem}[用坐标研究向量组]
  $\alpha_1,\cdots,\alpha_n$是$V$的一组基,$A_{n \times s}$,向量组$\beta_1,\cdots,\beta_s$满足
  \begin{equation*}
    (\beta_1,\cdots,\beta_s) = (\alpha_1,\cdots,\alpha_n)A
  \end{equation*}
  则$\text{dim}(\beta_1,\cdots,\beta_s) = r(A)$
\end{theorem}

\begin{proof}
  $A = [X_1,\cdots,X_s]$即$\beta_1,\cdots,\beta_s$的坐标,
  因此$A$的列向量与$\beta_1,\cdots,\beta_s$等价,从而$r(A) = \mathrm{dim}(\beta_1,\cdots,\beta_s)$
\end{proof}

\begin{corollary}[线性无关情况]
  $\alpha_1,\cdots,\alpha_n$线性无关,
  $(\beta_1,\cdots,\beta_n) = (\alpha_1,\cdots,\alpha_n)A$,
  则$\beta_i$线性无关当且仅当$A$可逆。
\end{corollary}

~

\begin{exercise}[同构的应用]
  (1)$\beta_1 = \alpha_1 + \alpha_2, \beta_2 = \alpha_1 + \alpha_3, \beta_3 = \alpha_1 + \alpha_2$,
  证明:$\alpha_1,\alpha_2,\alpha_3$和$\beta_1,\beta_2,\beta_3$等价

  (2)证明$n$为奇数时,$\alpha_1,\cdots,\alpha_n$线性无关当且仅当$\alpha_1 + \alpha_2,\cdots,\alpha_{n-1}+\alpha_n, a_n+\alpha_1$线性无关

  (3)设$\beta_1 = \alpha_2 + \cdots + \alpha_n, \beta_2 = \alpha_1 + \alpha_3 + \cdots + \alpha_n, \beta_n = \alpha_1 + \cdots + \alpha_{n-1}$,
  证明$\alpha_1,\cdots,\alpha_n$和$\beta_1,\cdots,\beta_n$有相同的秩
\end{exercise}

\begin{proof}
  (1)如下同构映射至坐标,显然$A$可逆,因此等价
  \begin{equation*}
    (\beta_1,\beta_2,\beta_3) = (\alpha_1,\alpha_2,\alpha_3) \left[
      \begin{array}{ccc}
        1&1&1 \\
         1&&1 \\
         0&1&
      \end{array}
    \right] := (\alpha_1,\alpha_{2},\alpha_3)A
  \end{equation*}

  (2)将后者设为$\beta_1,\cdots,\beta_n$,如下线性表出
  \begin{equation*}
    (\beta_1,\cdots,\beta_n) = (\alpha_1,\cdots,\alpha_n) \left[
      \begin{array}{ccccc}
        1&&&&1 \\
         1&1&&& \\
         &1&\ddots&& \\
         &&\ddots&1& \\
         &&&1&1
      \end{array}
    \right]
  \end{equation*}
  用Gauss消元法可看出$n$为奇数时矩阵满秩,因此等价。

  (3)同理列矩阵,用加边法可算出行列式非零,因此$A$可逆,即等价。
\end{proof}


\subsection{基、坐标}

\begin{definition}[基与坐标]
  $V$是线性空间,
  $\epsilon_1,\cdots,\epsilon_n$是$V$中线性无关的向量组,
  若$\forall \alpha \in V$可被该向量组线性表出,则称该向量组为$V$的一组基。
  且若
  \begin{equation*}
    \alpha = (\epsilon_1, \cdots, \epsilon_n)X
  \end{equation*}
  则称$X$为$\alpha$在$\epsilon_1,\cdots,\epsilon_n$基下的坐标
\end{definition}

\begin{note}
  坐标的计算:直接根据$\alpha = (\epsilon_1,\cdots,\epsilon_n)X$解线性方程组(或者用初等变换求逆)
  \begin{equation*}
    X = (\epsilon_1,\cdots,\epsilon_n)^{-1}\alpha
  \end{equation*}
\end{note}


~

\begin{exercise}[证明基]
  给定$\alpha_1 = (1,1,\cdots,1,1), \alpha_2 = (1,1,\cdots,1,0),\cdots,\alpha_n = (1,0,\cdots,0)$,
  证明其是$\mathbb{P}^n$的一组基
\end{exercise}

\begin{proof}
  当成列向量排成一个矩阵,满秩即可
\end{proof}

~

\begin{exercise}[坐标的计算]
  (1)已知$\epsilon_1,\epsilon_2,\epsilon_3$是$\mathbb{P}^3$的一组基,计算$\alpha = (1,2,1)^T$在这组基下的坐标
  \begin{equation*}
    \epsilon_1 = (1,1,1)^T, \epsilon_2 = (1,1,-1)^T, \epsilon_3 = (1,-1,-1)^T
  \end{equation*}
\end{exercise}

\begin{solution}
  (1)只需要解方程组$\alpha = k_1\epsilon_1 + k_2\epsilon_2 + k_3\epsilon_3$,
  即$[\epsilon_1,\epsilon_2,\epsilon_3]X = \alpha$:
  \begin{equation*}
    \left[
      \begin{array}{ccc}
        1&1&1 \\
         1&1&-1 \\
         1&-1&-1
      \end{array}
    \right]X = \left[
      \begin{array}{c}
        1\\
        2\\
        1
      \end{array}
    \right]
  \end{equation*}
  解得$k_1 = 1, k_2 = \frac{1}{2}, k_3 = - \frac{1}{2}$
\end{solution}

~

\begin{exercise}[难度进阶]
  (1)求非零向量$\alpha \in \mathbb{P}^4$,使得其在下面两组基中有相同的坐标
  \begin{equation*}
    \begin{cases}
      \epsilon_1 = (1,0,0,0)^T\\
      \epsilon_2 = (0,1,0,0)^T\\
      \epsilon_3 = (0,0,1,0)^T\\
      \epsilon_4 = (0,0,0,1)^T\\
    \end{cases}, \quad
    \begin{cases}
      \eta_1 = (2,1,-1,1)^T\\
      \eta_2 = (0,3,1,0)^T\\
      \eta_3 = (5,3,2,1)^T\\
      \eta_4 = (6,6,1,3)^T
    \end{cases}
  \end{equation*}
\end{exercise}

\begin{solution}
  (1)设$\alpha$在第一组基下的坐标为$(x_1,x_2,x_3,x_4)^T$,
  则由于同坐标
  \begin{equation*}
    \alpha = x_1\eta_1 + x_2\eta_2 + x_3\eta_3 + x_4\eta_4
  \end{equation*}
  解线性方程组$[\eta_1,\eta_2,\eta_3,\eta_4]X = \alpha$即可(注意将$\alpha$减到左边):
  \begin{equation*}
    \left[
      \begin{array}{cccc}
        1&0&5&6 \\
         1&2&3&6 \\
         -1&1&1&1 \\
         1&0&1&2
      \end{array}
    \right]X = 0
  \end{equation*}
  解得$\alpha = k(1,1,1,-1)$
\end{solution}

\subsection{坐标变换}

\begin{definition}[过渡矩阵]
  给定$(\epsilon_1,\cdots,\epsilon_n),(\eta_1,\cdots,\eta_n)$两组基,
  若存在矩阵$T$满足:
  \begin{equation*}
    (\epsilon_1,\cdots,\epsilon_n) = (\eta_1,\cdots,\eta_n) T
  \end{equation*}
  则称$T$为$\eta_1,\cdots,\eta_n$到$\epsilon_1,\cdots,\epsilon_n$的过渡矩阵(不要搞反)
\end{definition}

\begin{note}
  过渡矩阵的计算:直接根据$(\epsilon_1,\cdots,\epsilon_n) = (\eta_1,\cdots,\eta_n)T$得到
  \begin{equation*}
    T = (\eta_1,\cdots,\eta_n)^{-1}(\epsilon_1,\cdots,\epsilon_n)
  \end{equation*}
  用初等变换求$A^{-1}B$的方式直接求$T$即可。
\end{note}


\begin{theorem}[坐标变换]
  $V$是$n$维线性空间,$\eta_1,\cdots,\eta_n$和$\epsilon_1,\cdots,\epsilon_n$是空间两组基,
  且$(\eta_1,\cdots,\eta_n) = (\epsilon_1,\cdots,\epsilon_n)T$,
  向量$x \in V$在$\epsilon_i,\eta_i$下的坐标分别为$X,Y$,
  则
  \begin{equation*}
    X = TY
  \end{equation*}
\end{theorem}

\begin{proof}
  $x = (\epsilon_1,\cdots,\epsilon_n)X = (\eta_1,\cdots, \eta_n)Y = (\epsilon_1,\cdots,\epsilon_n)TY$即可。
\end{proof}

\begin{note}
  坐标变换用荡秋千记忆,即$(\eta_1,\cdots,\eta_n) = (\epsilon_1,\cdots,\epsilon_n)T$,
  分别对应$Y,X$,则把$T$荡到另一边,即$TY = X$。
  这与线性变换在不同基下的矩阵一样都需要记住。
\end{note}

~

\begin{exercise}[过渡矩阵的计算]
  给定两组基,求它们之间的过渡矩阵,并求$\xi = (1,0,0,-1)$在$\alpha_1,\alpha_2,\alpha_3,\alpha_4$下的坐标
  \begin{equation*}
    \begin{array}{cccc}
      \alpha_1 = (1,1,1,1)&\alpha_2 = (1,1,-1,-1)&\alpha_3 = (1,-1,1,-1)&\alpha_4 = (1,-1,-1,1) \\
                          \beta_1 = (1,1,0,1)&\beta_2 = (2,1,3,1)&\beta_3 = (1,1,0,0)&\beta_4 = (0,1,-1,-1)
    \end{array}
  \end{equation*}
\end{exercise}

\begin{solution}
  (1)先取标准基$\epsilon_1,\epsilon_2,\epsilon_3,\epsilon_4$,
  设
  \begin{equation*}
    (\alpha_1,\alpha_2,\alpha_3,\alpha_4) = (\epsilon_1,\epsilon_2,\epsilon_3,\epsilon_4)A,
    (\beta_1,\beta_2,\beta_3,\beta_4) = (\epsilon_1,\epsilon_2,\epsilon_3,\epsilon_4)B
  \end{equation*}
  则
  \begin{equation*}
    A = \left[
      \begin{array}{cccc}
        1&1&1&1 \\
         1&1&-1&-1 \\
         1&-1&1&-1 \\
         1&-1&-1&1
      \end{array}
    \right],
    B = \left[
      \begin{array}{cccc}
        1&2&1&0 \\
         1&1&1&1 \\
         0&3&0&-1 \\
         1&1&0&-1
      \end{array}
    \right]
  \end{equation*}
  由此得到
  \begin{equation*}
    (\beta_1,\beta_2,\beta_3,\beta_4) = (\epsilon_1,\epsilon_2,\epsilon_3,\epsilon_4)B = (\alpha_1,\alpha_2,\alpha_3,\alpha_4)A^{-1}B
  \end{equation*}
  综上得到过渡矩阵为(具体计算时用行初等变换:$(A|B) \rightarrow (E|A^{-1}B)$)
  \begin{equation*}
    A^{-1}B = \frac{1}{4} \left[
      \begin{array}{cccc}
        3&7&2&-1 \\
         1&-1&2&3 \\
         -1&3&0&-1 \\
         1&-1&0&-1
      \end{array}
    \right]
  \end{equation*}

  (2)$\xi$在$(\epsilon_1,\epsilon_2,\epsilon_3,\epsilon_4)$下的坐标即本身,
  因此在$(\alpha_1,\alpha_2,\alpha_3,\alpha_4)$下坐标为$A^{-1}\xi$(具体计算时也和$(A|B)$摆在一起),
  结果为$(0, \frac{1}{2}, \frac{1}{2},0)^T$
\end{solution}

\subsection{和与交、维数定理}

\begin{definition}[线性空间和与交]
  $V_1,V_2$为线性空间,则
  \begin{equation*}
    V_1 \cap V_2 := \{\alpha: \alpha \in V_1, \alpha \in V_2\}, V_1 + V_2 := \{\alpha = \alpha_1 + \alpha_2: \alpha_1 \in V_1, \alpha_2 \in V_2\}
  \end{equation*}
  且显然$V_1 \cap V_2, V_1 + V_2$都是子空间。
\end{definition}

\begin{note}
  $V_1 \cup V_2$由于运算不封闭,一般而言不是子空间。
  注意分别$V_1 \cup V_2$和$V_1 + V_2$
\end{note}

~

\begin{exercise}[和与交基本计算]
  (1)已知$V_1 = L(\alpha_1,\alpha_2,\alpha_3), V_2 = L(\beta_1,\beta_2)$,其中
  \begin{equation*}
    \alpha_1 = \left(
      \begin{array}{c}
        1\\
        2\\
        1\\
        0
      \end{array}
    \right), \alpha_2 = \left(
      \begin{array}{c}
        -1\\
        1\\
        1\\
        1
      \end{array}
    \right), \alpha_3 = \left(
      \begin{array}{c}
        0\\
        3\\
        2\\
        1
      \end{array}
    \right), \beta_1 = \left(
      \begin{array}{c}
        2\\
        -1\\
        0\\
        1
      \end{array}
    \right), \beta_2 = \left(
      \begin{array}{c}
        1\\
        -1\\
        3\\
        7
      \end{array}
    \right)
  \end{equation*}
  求$V_1 + V_2, V_1 \cap V_2$的基与维数

  (2)已知$W_1 = L(\alpha_1,\alpha_2,\alpha_3), W_2 = L(\beta_1,\beta_2)$,求$W_1 \cap W_2, W_1 + W_2$的基与维数。
  \begin{equation*}
    \alpha_1 = \left(
      \begin{array}{c}
        1\\
        2\\
        -1\\
        -3
      \end{array}
    \right), \alpha_2 = \left(
      \begin{array}{c}
        -1\\
        -1\\
        2\\
        1
      \end{array}
    \right), \alpha_3 = \left(
      \begin{array}{c}
        -1\\
        -3\\
        0\\
        5
      \end{array}
    \right),
    \beta_1 = \left(
      \begin{array}{c}
        -1\\
        0\\
        4\\
        2
      \end{array}
    \right), \beta_2 = \left(
      \begin{array}{c}
        0\\
        5\\
        9\\
        -14
      \end{array}
    \right)
  \end{equation*}
\end{exercise}

\begin{solution}
  (1)$V_1 + V_2 = L(\alpha_1,\alpha_2,\alpha_3,\beta_1,\beta_2)$,
  因此只需要排成列,做初等变换:
  \begin{equation*}
    \left(
      \begin{array}{ccccc}
        1&-1&0&2&1 \\
         2&1&3&-1&-1 \\
         1&1&2&0&3 \\
         0&1&1&1&7
      \end{array}
    \right) \rightarrow \left(
      \begin{array}{ccccc}
        1&0&1&0&-1 \\
         0&1&1&0&4 \\
         0&0&0&1&3 \\
         0&0&0&0&0
      \end{array}
    \right):= A
  \end{equation*}
  因此维数为$3$,基为$\alpha_1,\alpha_2,\beta_1$

  任取$\alpha \in V_1 \cap V_2$,设$\alpha = x_1\alpha_1 + x_2\alpha_2 + x_3\alpha_3 = y_1\beta_1 + y_2\beta_2$,
  因此系数需要满足方程组:
  \begin{equation*}
    x_1\alpha_1 + x_2\alpha_2 + x_3\alpha_3 - y_1\beta_1 - y_2\beta_2 = 0
  \end{equation*}
  设$x_4 = -y_1, x_5 = y_2$,即求解$AX = [\alpha_1,\alpha_2,\alpha_3,\beta_1,\beta_2]X =  0$,
  解得基础解系为
  \begin{equation*}
    \eta_1 = \left(
      \begin{array}{c}
        -1\\
        -1\\
        1\\
        0\\
        0
      \end{array}
    \right), \eta_2 = \left(
      \begin{array}{c}
        1\\
        -4\\
        0\\
        -3\\
        1
      \end{array}
    \right)
  \end{equation*}
  对应$V_1 \cap V_2$的基础解系为:
  \begin{equation*}
    -\alpha_1 -\alpha_2 + \alpha_3 = (0,0,0,0)^T, \alpha_1 - 4\alpha_2 = (5,-2,-3,-4)^T
  \end{equation*}
  只取非零的,故只能为$k(5,-2,-3,-4)$,
  维数为$1$,基即$\alpha_1 - 4\alpha_2$。
\end{solution}

~

\begin{exercise}[和与并的区别]
  (1)设$V_1,V_2$是$\mathbb{P}$上线性空间$V$的两个非平凡子空间,
  证明:$\exists \alpha \in V$使得$\alpha \not\in V_1, \alpha \not\in V_2$

  (2)$V_1,\cdots,V_s$是$V$的$s$个非平凡子空间,证明:$V$中至少有一个向量不属于$V_1,\cdots,V_s$中的任意一个。

  (3)$V_1,V_2$是两个子空间,证明:$V_1 + V_2 = V_1 \cup V_2$当且仅当$V_1 \subseteq V_2$或者$V_2 \subseteq V_1$
\end{exercise}

\begin{proof}
  (1)由于$V_1$非平凡,$\exists \alpha \not \in V_1$,
  若$\alpha \not\in V_2$,则结论成立;
  否则$\alpha \in V_2$,由于$V_2$非平凡,因此$\exists \beta \not \in V_2$,
  若$\beta \not\in V_1$,则结论成立;
  否则$\beta \in V_1$,此时有
  \begin{equation*}
    \alpha \not \in V_1, \beta \in V_1, \alpha \in V_2, \beta \not\in V_2
  \end{equation*}
  因此$\gamma = \alpha + \beta \not \in V_1, \gamma \not \in V_2$

  (2)用归纳法

  (3)右推左显然。
  左推右:设$V_1 + V_2 = V_1 \cup V_2$,但$V_1 \not \subset V_2$,
  则$\exists \alpha_1 \in V_1, \alpha_1 \not\in V_2$,
  但
  \begin{equation*}
    \forall \alpha_2 \in V_2,  \alpha = \alpha_1 + \alpha_2 \in V_1 + V_2
  \end{equation*}
  由于$V_1 + V_2 = V_1 \cup V_2$,因此$\alpha$至少属于$V_1,V_2$中的一个,
  显然$\alpha \not\in V_2$(根据加法封闭),
  因此$\alpha \in V_1$,
  故
  \begin{equation*}
    \alpha_2 = \alpha - \alpha_1 \in V_1
  \end{equation*}
  综上得到$V_2 \subseteq V_1$
\end{proof}

~

\begin{theorem}[维数公式]
  设$V_1,V_2$是有限维线性空间$V$的两个子空间,则
  \begin{equation*}
    \mathrm{dim} V_1 + \mathrm{dim} V_2 = \mathrm{dim} (V_1 + V_2) + \mathrm{dim}(V_1 \cap V_2)
  \end{equation*}
\end{theorem}

\begin{proof}
  设小扩大。
  设$V_1 \cap V_2$的基为$\alpha_1,\cdots,\alpha_s$,
  $V_1,V_2$的基分别为:
  \begin{equation*}
    \begin{cases}
      \alpha_1,\cdots,\alpha_s,\beta_{s+1},\cdots,\beta_m\\
      \alpha_1,\cdots,\alpha_s,\gamma_{s+1},\cdots,\gamma_n
    \end{cases}
  \end{equation*}
  则$V_1+V_2 = L(\alpha_1,\cdots,\alpha_s,\beta_{s+1},\cdots,\beta_m,\gamma_{s+1},\cdots,\gamma_n)$,
  下面证明上面的向量全部线性无关。
  不妨设
  \begin{equation*}
    k_1\alpha_1+\cdots+k_s\alpha_s + k_{s+1}\beta_{s+1}+\cdots+k_n\beta_n + l_{s+1}\gamma_{s+1}+\cdots+l_n\gamma_n = 0
  \end{equation*}
  将$l_i\gamma_i$的部分移到右侧,此时左侧线性组合属于$V_1$,右侧属于$V_2$,
  因此都属于$V_1 \cap V_2$。
  根据左侧得到:
  \begin{equation*}
    k_1\alpha_1 + \cdots +k_s \alpha_s + k_{s+1}\beta_{s+1} + \cdots +  k_n\beta_n \in V_1 \cap V_2 \Rightarrow k_{s+1} = k_{s+2} = \cdots = k_n = 0 \Rightarrow k_1 = \cdots = k_s = 0
  \end{equation*}
  因此左侧系数全为$0$,
  同理右侧系数也全为$0$
\end{proof}

~

\begin{exercise}[维数公式相关]
  (1)经典:$V_1,V_2$是有限维线性空间$V$的子空间,
  $\mathrm{dim}(V_1 + V_2) = \mathrm{dim}(V_1 \cap V_2) + 1$,
  证明:要么$V_1 \subseteq V_2$,要么$V_2 \subseteq V_1$
\end{exercise}

\begin{proof}
  (1)根据维数公式得到
  \begin{equation*}
    \mathrm{dim}(V_1) + \mathrm{dim}(V_2) = \mathrm{dim}(V_1 + V_2) + \mathrm{dim}(V_1 \cap V_2) = 2 \mathrm{dim}(V_1 \cap V_2) + 1
  \end{equation*}
  等价于
  \begin{equation*}
    \left( \mathrm{dim}V_1 - \mathrm{dim}(V_1 \cap V_2) \right) + \left( \mathrm{dim}V_2 - \mathrm{dim}(V_1 \cap V_2) \right) = 1
  \end{equation*}
  由于都是整数,只能一个$0$,一个$1$,因此结论成立。
\end{proof}

\subsection{直和理论}

\begin{definition}[直和]
  $V_1,V_2$是线性空间$V$的子空间,
  若$V_1 \cap V_2 = \{0\}$,
  则记$V_1 \oplus V_2 := V_1 + V_2$
\end{definition}

\begin{definition}[补空间]
  若$V = V_1 \oplus V_2$,则称$V_2$是$V_1$的补空间。
\end{definition}

\begin{note}
  补空间一般是不唯一的,因为$\alpha_1,\cdots,\alpha_m$扩充为$\alpha_1,\cdots,\alpha_n$的方式不唯一。
\end{note}

\begin{theorem}[直和等价条件]
  $W = V_1 \oplus V_2$当且仅当
  \begin{itemize}
  \item $W$中任意向量$\alpha = \alpha_1 + \alpha_2, \alpha_1 \in V_1, \alpha_2 \in V_2$唯一表出
  \item $W$中零向量被$V_1, V_2$中向量唯一表出:$0 = 0+ 0$
  \item $\mathrm{dim}W = \mathrm{dim}V_1 + \mathrm{dim}V_2$
  \end{itemize}
\end{theorem}

~

\begin{exercise}[判断直和]
  (1)数域$P$上所有$n$阶矩阵组成线性空间$V = M_n(P)$,
  $V_1$表示所有对称矩阵组成的集合,$V_2$表示所有反对称矩阵组成的集合,
  证明$V_1,V_2$是子空间且$V = V_1 \oplus V_2$
\end{exercise}

\begin{proof}
  (1)显然关于加法和数乘封闭,因此是子空间。
  首先证明$V = V_1 + V_2$,
  由于矩阵$A$都可以写为:
  \begin{equation*}
    A = \frac{A + A^T}{2} + \frac{A^T - A}{2}
  \end{equation*}
  从而$V = V_1 + V_2$。

  证明$V = V_1 \oplus V_2$:
  可用$V_1 \cap V_2 = \{\O\}$,
  此时取$A \in V_1 \cap V_2$,
  则$A = A^T = - A^T$,得到$A^T = A = O$,因此为直和。
  也可以用$O$分解唯一,设$O = B + C = B^T - C^T$,
  而$O^T = B^T + C^T$,因此$B = C = O$。
\end{proof}

~

\begin{theorem}[多空间直和等价条件]
  $V_1,\cdots,V_s$是$W$的子空间,且$W = V_1 + V_2 + \cdots + V_s$,则$W = V_1 \oplus V_2 \oplus \cdots \oplus V_s$当且仅当
  \begin{itemize}
  \item $W$中零向量分解唯一
  \item $W$中任意向量$\alpha$分解唯一
  \item $V_i \cap \sum\limits_{j \neq i} V_j = \{0\}$
  \item $\mathrm{dim}W = \mathrm{dim}V_1 + \cdots + \mathrm{dim}V_s$
  \end{itemize}
\end{theorem}

\begin{proof}
  (3)只证明第三条。第三条推直和:若
  \begin{equation*}
    0 = \alpha_1 + \cdots + \alpha_n \Rightarrow \alpha_i = \alpha_1 + \cdots + \alpha_{i-1}+\alpha_{i+1}+\cdots + \alpha_n
  \end{equation*}
  由于$V_i \cap \sum\limits_{j \neq i} V_j = \{0\}$,
  因此$\alpha_k = 0, \forall k = 1,2,\cdots,s$,
  从而是直和

  直和推第三条:若$\alpha \in V_i \cap \sum\limits_{j \neq i} V_j$,则$\alpha \in V_i$,且
  \begin{equation*}
    \alpha = \alpha_1 + \cdots + \alpha_{i-1} + \alpha_{i+1}+\cdots +\alpha_n \Rightarrow
    \alpha_1 + \cdots + \alpha_{i-1} + (-\alpha) + \alpha_{i+1}+\cdots +\alpha_{n }= 0
  \end{equation*}
  由于$\sum\limits_{i = 1}^n V_i$为直和,因此$\alpha = 0$,即第三条结论成立。
\end{proof}

\begin{theorem}[直和合并定理]
  若$V = V_1 \oplus V_2$,$V_1 = V_{11} \oplus V_{12} \oplus \cdots \oplus V_{1s}, V_2 = V_{21} \oplus \cdots \oplus V_{2t}$,则
  \begin{equation*}
    V = V_{11}\oplus \cdots V_{1s} \oplus V_{21} \oplus \cdots \oplus V_{2t}
  \end{equation*}
\end{theorem}

\begin{proof}
  只证明$V = V_1 \oplus V_2$,$V_1 = V_{11} \oplus V_{12}$时,
  $V = V_{11} \oplus V_{12} \oplus V_2$。

  首先显然有$V = V_{11} + V_{12} + V_2$。
  设
  \begin{equation*}
    0 = \alpha_{11} + \alpha_{12} + \alpha_2 = (\alpha_{11} + \alpha_{12}) + \alpha_2
  \end{equation*}
  由于$V = V_1 \oplus V_2$,因此得到$\alpha_{11} + \alpha_{12} = \alpha _2 = 0$,
  同理根据$V_1 = V_{11}\oplus V_{12}$得到$\alpha_{11} = \alpha_{12} = 0$,
  综上$0$元素分解唯一,为直和。
\end{proof}



\subsection{商空间}

\begin{definition}[商空间的元素]
  $W$是线性空间$V$的子空间,
  $\alpha \in V$,
  定义
  \begin{equation*}
    \alpha + W := \{\alpha + \omega: \omega \in W\}
  \end{equation*}
\end{definition}

\begin{definition}[商空间]
  定义$V/W$为$\{\alpha + W: \alpha \in V\}$,
  即线性空间子集组成的集合。
  可以证明其是一个线性空间
\end{definition}

\begin{definition}[自然映射]
  定义$\eta : V \rightarrow V/W$为自然映射
\end{definition}






